%% Generated by Sphinx.
\def\sphinxdocclass{report}
\documentclass[letterpaper,10pt,english]{sphinxmanual}
\ifdefined\pdfpxdimen
   \let\sphinxpxdimen\pdfpxdimen\else\newdimen\sphinxpxdimen
\fi \sphinxpxdimen=.75bp\relax
%% turn off hyperref patch of \index as sphinx.xdy xindy module takes care of
%% suitable \hyperpage mark-up, working around hyperref-xindy incompatibility
\PassOptionsToPackage{hyperindex=false}{hyperref}

\PassOptionsToPackage{warn}{textcomp}

\catcode`^^^^00a0\active\protected\def^^^^00a0{\leavevmode\nobreak\ }
\usepackage{cmap}
\usepackage{fontspec}
\defaultfontfeatures[\rmfamily,\sffamily,\ttfamily]{}
\usepackage{amsmath,amssymb,amstext}
\usepackage{polyglossia}
\setmainlanguage{english}



\setmainfont{FreeSerif}[
  Extension      = .otf,
  UprightFont    = *,
  ItalicFont     = *Italic,
  BoldFont       = *Bold,
  BoldItalicFont = *BoldItalic
]
\setsansfont{FreeSans}[
  Extension      = .otf,
  UprightFont    = *,
  ItalicFont     = *Oblique,
  BoldFont       = *Bold,
  BoldItalicFont = *BoldOblique,
]
\setmonofont{FreeMono}[
  Extension      = .otf,
  UprightFont    = *,
  ItalicFont     = *Oblique,
  BoldFont       = *Bold,
  BoldItalicFont = *BoldOblique,
]


\usepackage[Bjarne]{fncychap}
\usepackage[,numfigreset=1,mathnumfig]{sphinx}

\fvset{fontsize=\small}
\usepackage{geometry}


% Include hyperref last.
\usepackage{hyperref}
% Fix anchor placement for figures with captions.
\usepackage{hypcap}% it must be loaded after hyperref.
% Set up styles of URL: it should be placed after hyperref.
\urlstyle{same}
\addto\captionsenglish{\renewcommand{\contentsname}{数学建模基础课}}

\usepackage{sphinxmessages}




\title{数学建模系列课程}
\date{Sep 07, 2020}
\release{}
\author{LinCol Education}
\newcommand{\sphinxlogo}{\vbox{}}
\renewcommand{\releasename}{}
\makeindex
\begin{document}

\pagestyle{empty}
\sphinxmaketitle
\pagestyle{plain}
\sphinxtableofcontents
\pagestyle{normal}
\phantomsection\label{\detokenize{docs/index::doc}}


这是LinCol教育开发的HiMCM备赛课程。

目前正在迭代过程汇总。

已经加入的课件有:
\begin{itemize}
\item {} 
规划模型

\item {} 
评价模型

\item {} 
预测模型

\end{itemize}


\chapter{规划模型}
\label{\detokenize{docs/LP:id1}}\label{\detokenize{docs/LP::doc}}

\section{线性规划简介}
\label{\detokenize{docs/LP:id2}}
在人们的生产实践中,经常会遇到如何利用现有资源来安排生产,以取得最大经济效益的问题。此类问题构成了运筹学的一个重要分支—数学规划,而\sphinxstylestrong{线性规划}(Linear Programming)则是数学规划的一个重要分支,也是一种十分常用的最优化模型。

而随着计算机的发展,线性规划的方法被应用于广泛的领域,已成为数学建模里最为经典,最为常用的模型之一。

线性规划模型可用于求解利润最大,成本最小,路径最短等最优化问题。


\subsection{概念引入:一个典型的线性规划问题}
\label{\detokenize{docs/LP:id3}}
\begin{sphinxadmonition}{note}{例题}

某机床厂生产甲、乙两种机床,每台销售后的利润分别为\sphinxstylestrong{4千元}与\sphinxstylestrong{3千元}。
\begin{itemize}
\item {} 
生产甲机床需用A、B机器加工,加工时间分别为每台2小时和1小时

\item {} 
生产乙机床需用A、B、C三种机器加工,加工时间为每台各一小时

\end{itemize}

若每天可用于加工的机器时数分别为\sphinxstylestrong{A机器10小时、B机器8小时和C机器7小时}

问该厂应生产甲、乙机床各几台,才能使总利润最大?
\end{sphinxadmonition}

这个问题是一个十分典型的线性规划问题,首先对问题提取出关键信息:
\begin{itemize}
\item {} 
\sphinxstylestrong{决策}:生产几台甲、乙机床

\item {} 
\sphinxstylestrong{优化目标}:总利润最大

\item {} 
\sphinxstylestrong{约束}:生产机床的使用时间有限

\end{itemize}

将上诉三个要素写成数学表达式,就是一个典型的线性规划模型:
\begin{equation*}
\begin{split}
\begin{aligned}
&\max \quad z= 4x_{1}+ 3x_{2}\\
&\text { s.t. }\left\{\begin{array}{l}
{2x_{1}+ x_{2} \leq 10} \\
{x_{1}+ x_{2} \leq 8} \\
{x_2  \leq 7}\\
{ x_{1}, x_{2} \geq 0}
\end{array}\right.
\end{aligned}
\end{split}
\end{equation*}

\subsection{线性规划模型的三要素}
\label{\detokenize{docs/LP:id4}}
线性规划模型主要包括三个部分:决策变量、目标函数、约束条件
\begin{itemize}
\item {} 
\sphinxstylestrong{决策变量}
决策变量是指问题中可以改变的量,例如生产多少货物,选择哪条路径等;线性规划的目标就是找到最优的决策变量。在线性规划中决策变量包括实数变量,整数变量,0\sphinxhyphen{}1变量等。

\item {} 
\sphinxstylestrong{目标函数}
目标函数就是把问题中的决策目标量化,一般分为最大化目标函数\(\max\)和最小化目标函数\(\min\)
在线性规划中,目标函数为一个包含决策变量的线性函数,例如\(\max x_1 + x_2\)

\item {} 
\sphinxstylestrong{约束条件}
约束条件是指问题中各种时间,空间,人力,物力等限制。
在线性规划中约束条件一般表示为一组包含决策变量的不等式,例如\(x_1 + 2x_2 \leq 10 \)或者\(4x_1 + 3x_2 \leq 24\)。此外,决策变量的取值范围称为符号约束,例如\(x_1 \geq 0, x_2 \geq 0\)。

\end{itemize}

\begin{sphinxadmonition}{note}{思考}

什么是线性?
\end{sphinxadmonition}


\subsection{线性规划模型的标准形式}
\label{\detokenize{docs/LP:id5}}
对于一个线性规划模型:
\begin{equation*}
\begin{split}
\begin{aligned}
&\max \quad z= x_{1}+ x_{2}\\
&\text { s.t. }\left\{\begin{array}{l}
{x_{1}+2 x_{2} \leq 10} \\
{4 x_{1}+3 x_{2} \leq 24} \\
{ x_{1}, x_{2} \geq 0}
\end{array}\right.
\end{aligned}
\end{split}
\end{equation*}
其中\(s.t.\)为subject to的缩写。上述模型可以写成如下的矩阵形式:
\begin{equation*}
\begin{split}
\begin{aligned}
&\min c^{T} x\\
&\text { s.t. }\left\{\begin{array}{l}
{A x \leq b} \\
{ x \geq 0}
\end{array}\right.
\end{aligned}
\end{split}
\end{equation*}
其中
\begin{equation*}
\begin{split}
c = [1,1]^T
\end{split}
\end{equation*}\begin{equation*}
\begin{split}
x= [x_1,x_2]^T
\end{split}
\end{equation*}\begin{equation*}
\begin{split}
A = \left[ \begin{matrix}
1 & 2\\
4&3
\end{matrix}\right]
\end{split}
\end{equation*}\begin{equation*}
\begin{split}
b = \left[\begin{matrix}10\\24 \end{matrix}   \right]
\end{split}
\end{equation*}
\begin{sphinxadmonition}{tip}{Tip:}
对于有 \(n\) 个决策变量,\(m\) 个约束的线性规划模型, \(c,x\) 为 \(n\) 维列向量,\(b\) 为 \(m\) 维列向量,\(A\) 为 \(m \times n\) 维矩阵。
\end{sphinxadmonition}

线性规划的目标函数可能是最大化,也可能是最小化,约束条件的符号可能是小于等于,也可能是大于等于。甚至还有等于。因此为了编程方便,一般统一为\sphinxstylestrong{最小化目标函数,小于等于约束}。

最大化目标函数可以添加负号变为最小化约束:
\begin{equation*}
\begin{split}
\max z = x_1 +  x_2 \implies \min -z = -x_1 - x_2
\end{split}
\end{equation*}
大于等于约束可以两边乘以 \(-1\) 变为小于等于约束:
\begin{equation*}
\begin{split}
x_1 + 2x_2 \geq 10 \implies -x_1 - 2x_2 \leq -10
\end{split}
\end{equation*}
等于约束可以变为一个大于等于约束和一个小于等于约束,\sphinxstylestrong{但在编程中一般支持直接写等式约束,可以不进行转换:}
\begin{equation*}
\begin{split}
x_1 + 2x_2 = 10 \implies x_1 + 2x_2 \leq 10, x_1 + 2x_2 \geq 10
\end{split}
\end{equation*}
综上,考虑了等式约束后的线性规划的标准形式可以写为:

\begin{sphinxadmonition}{note}{线性规划的标准形式}
\begin{equation*}
\begin{split}
\begin{aligned}
&\min c^{T} x\\
&\text { s.t. }\left\{\begin{array}{l}
{A x \leq b} \\
{ Aeq  \cdot x=b e q} \\
{l b \leq x \leq u b}
\end{array}\right.
\end{aligned}
\end{split}
\end{equation*}\end{sphinxadmonition}

\begin{sphinxadmonition}{note}{课堂练习}

请写出以下两个线性规划的标准型
\begin{equation*}
\begin{split}
\begin{aligned}
&\min \quad z= 5x_{1}+ 4x_{2} + 6x_3\\
&\text { s.t. }\left\{\begin{array}{l}
{3x_{1}+2 x_{2} - x_3 \geq 10} \\
{4 x_{1}+3 x_{2} - 2x_3 \geq 24} \\
{ x_{1}+ x_{2}  + x_3 \leq 8} \\
{5x_{1}+2 x_{2}  =  10} \\
{2x_{2}+3 x_{3}  =  6} \\
{ x_{1}, x_{2},x_3 \geq 0}
\end{array}\right.
\end{aligned}
\end{split}
\end{equation*}
\(\quad\)
\begin{equation*}
\begin{split}
\begin{aligned}
&\max \quad z= x_{1}+ x_{2} + 3x_3 + 2x_4\\
&\text { s.t. }\left\{\begin{array}{l}
{3x_{1}+2 x_{2} - x_3 \geq 5} \\
{4 x_{1}+3 x_{2} - 2x_3 + x_4 \geq 12} \\
{ x_{1}+ x_{2}  + x_3 + x_4 \leq 7} \\
{5x_{1}+2 x_{2}  =  5} \\
{2x_{2}+3 x_{4}  =  4} \\
{ x_{1}, x_{2},x_3,x_4 \geq 0}
\end{array}\right.
\end{aligned}
\end{split}
\end{equation*}\end{sphinxadmonition}

\begin{sphinxadmonition}{note}{ 点击查看问题1答案}
\begin{equation*}
\begin{split}
c = [5,4,6]^T
\end{split}
\end{equation*}\begin{equation*}
\begin{split}
x = [x_1,x_2,x_3]^T
\end{split}
\end{equation*}\begin{equation*}
\begin{split}
A = \left[\begin{matrix} -3 & -2 & 1 \\ 
-4 & -3 & 2\\
1 & 1 & 1
\end{matrix} \right]
\end{split}
\end{equation*}\begin{equation*}
\begin{split}
b = \left[\begin{matrix} -10 \\ 
-24 \\
8
\end{matrix} \right] 
\end{split}
\end{equation*}\begin{equation*}
\begin{split}
Aeq = \left[\begin{matrix} 5 & 2 & 0 \\ 
0 & 2 & 3\\
\end{matrix} \right]
\end{split}
\end{equation*}\begin{equation*}
\begin{split}
beq = \left[\begin{matrix} 5 \\ 
4 
\end{matrix} \right] 
\end{split}
\end{equation*}
\begin{sphinxadmonition}{note}{}

问题2请大家课上自行完成
\end{sphinxadmonition}
\end{sphinxadmonition}


\section{图解法和单纯形法}
\label{\detokenize{docs/LP:id6}}

\subsection{图解法}
\label{\detokenize{docs/LP:id7}}
对于较为简单且只有两个决策变量的线性规划问题可以使用图解法。

例如考虑如下线性规划模型:
\begin{equation*}
\begin{split}
\begin{aligned}
&z= x_{1}+ x_{2}\\
&\text { s.t. }\left\{\begin{array}{l}
{x_{1}+2 x_{2} \leq 10} \\
{4 x_{1}+3 x_{2} \leq 24} \\
{x_{1}, x_{2} \geq 0}
\end{array}\right.
\end{aligned}
\end{split}
\end{equation*}
以决策变量 \(x_1\) 为 \(x\) 轴,决策变量 \(x_2\) 为 \(y\) 轴,可以将约束条件表示为如下所示的多边形,其中多边形的每一个边界即为一个约束条件,目标函数则为一条直线,优化目标为使该条直线在 \(y\) 轴上的截距最大。

\begin{figure}[htbp]
\centering

\noindent\sphinxincludegraphics[height=450\sphinxpxdimen]{{p1}.svg}
\end{figure}

从图中可以看出,当目标函数经过多边形的顶点A(即表示两个约束条件的直线交点)时, \(y\) 轴截距取得最大值。
即:\sphinxstylestrong{当\(x_1 = 3.6, x_2 = 3.2\) 时,目标函数取得最大值为 \(z = x_1 + x_2 = 3.6 +3.2 = 6.8\)}


\subsection{单纯形法}
\label{\detokenize{docs/LP:id8}}
\sphinxstylestrong{对于决策变量比较多的线性规划模型,图解法不再适用。}
单纯形法是1947 年G. B. Dantzig提出的一种十分有效的求解方法,极大地推广了线性规划的应用,
直到今日也在一些线性规划的求解器中使用。

从图解法的例子中,我们可以看出,约束条件所围成的区域为一个\sphinxstylestrong{凸多边形},当决策变量多于两个时,约束条件围城的区域为一个\sphinxstylestrong{凸多面体},称之为\sphinxstylestrong{可行域}。其中每一个面(称之为\sphinxstylestrong{超平面})即代表一个约束条件。

\begin{sphinxadmonition}{hint}{Hint:}
可以证明:线性规划的最优解一定在可行域的边界上
\end{sphinxadmonition}

\begin{figure}[htbp]
\centering

\noindent\sphinxincludegraphics[height=400\sphinxpxdimen]{{c2d87f7c9a2dd128cb96a627b477242157ccc6a6}.jpg}
\end{figure}

单纯形法的思路就是在可行域的一个顶点处找到一个\sphinxstylestrong{初始可行解},\sphinxstylestrong{判断该解是不是最优},若不是,则\sphinxstylestrong{迭代到下一个顶点处}进行重复判断。因为最优解的搜索范围从整个可行域缩小到了可行域的有限个顶点,算法的效率得到了极大的提升。

\begin{figure}[htbp]
\centering

\noindent\sphinxincludegraphics[height=400\sphinxpxdimen]{{0}.png}
\end{figure}

\sphinxstylestrong{具体的找初始可行解的方法,判断解是否最优的条件,如何进行迭代这里不做详细展开,有兴趣的同学可以查阅相关资料。}

\begin{sphinxadmonition}{tip}{Tip:}
此外,求解线性规划的方法还有椭球法、卡玛卡算法、内点法等。
其中内点法因为求解效率更高,在决策变量多,约束多的情况下能取得更好的效果,目前主流线性规划求解器都是使用的内点法。
\end{sphinxadmonition}


\section{使用Python求解简单线性规划模型}
\label{\detokenize{docs/LP:python}}
使用python进行线性规划求解的 编程思路为
\begin{itemize}
\item {} 
\sphinxstylestrong{选择适当的决策变量}
在解决实际问题时,把问题归结成一个线性规划数学模型是很重要的一步,但往往也是困难的一步,模型建立得是否恰当,直接影响到求解。
而选适当的决策变量,是我们建立有效模型的关键之一。

\item {} 
\sphinxstylestrong{将求解目标简化为求一个目标函数的最大/最小值}
能把要求解的问题简化为一个最值问题是能否使用线性规划模型的关键,如果这一点不能达到,之后的工作都有没有意义的。

\item {} 
\sphinxstylestrong{根据实际要求写出约束条件(正负性,资源约束等)}
线性规划的约束条件针对不同的问题有不同的形式,总结来说有以下三种:等式约束、不等式约束、符号约束

\end{itemize}

\begin{sphinxadmonition}{note}{例题1}

考虑如下线性规划问题
\begin{equation*}
\begin{split}
\begin{aligned}
&\max z=2 x_{1}+3 x_{2}-5 x_{3}\\
&\text { s.t. }\left\{\begin{array}{l}
{x_{1}+x_{2}+x_{3}=7} \\
{2 x_{1}-5 x_{2}+x_{3} \geq 10} \\
{x_{1}+3 x_{2}+x_{3} \leq 12}\\
{x_{1}, x_{2}, x_{3} \geq 0}
\end{array}\right.
\end{aligned}
\end{split}
\end{equation*}\end{sphinxadmonition}

\sphinxstylestrong{Step1: 导入相关库}

\begin{sphinxVerbatim}[commandchars=\\\{\}]
\PYG{k+kn}{import} \PYG{n+nn}{numpy} \PYG{k}{as} \PYG{n+nn}{np} 
\PYG{k+kn}{from} \PYG{n+nn}{scipy} \PYG{k+kn}{import} \PYG{n}{optimize} \PYG{k}{as} \PYG{n}{op} 
\end{sphinxVerbatim}

\sphinxstylestrong{Step2: 定义决策变量}

\begin{sphinxVerbatim}[commandchars=\\\{\}]
\PYG{c+c1}{\PYGZsh{} 给出变量取值范围}
\PYG{n}{x1}\PYG{o}{=}\PYG{p}{(}\PYG{l+m+mi}{0}\PYG{p}{,}\PYG{l+m+mi}{7}\PYG{p}{)}  
\PYG{n}{x2}\PYG{o}{=}\PYG{p}{(}\PYG{l+m+mi}{0}\PYG{p}{,}\PYG{l+m+mi}{7}\PYG{p}{)}
\PYG{n}{x3}\PYG{o}{=}\PYG{p}{(}\PYG{l+m+mi}{0}\PYG{p}{,}\PYG{l+m+mi}{7}\PYG{p}{)}
\end{sphinxVerbatim}

\sphinxstylestrong{Step3: 将原问题化为标准形式}
\begin{equation*}
\begin{split}
\begin{aligned}
&{\min -z=-2 x_{1}-3 x_{2}+5 x_{3}}\\
&\text { s.t. }\left\{\begin{array}{l}
{x_{1}+x_{2}+x_{3}=7} \\
{-2 x_{1}+5 x_{2}-x_{3} \leq -10} \\
{x_{1}+3 x_{2}+x_{3} \leq 12}\\
{x_{1}, x_{2}, x_{3} \geq 0}
\end{array}\right.
\end{aligned}
\end{split}
\end{equation*}
\begin{sphinxadmonition}{warning}{Warning:}
注意:编程时默认为最小化目标函数,因此这里改为 \(\min-z\);第二个约束为大于等于约束,这里化为小于等于约束
\end{sphinxadmonition}

\sphinxstylestrong{Step4: 定义目标函数系数和约束条件系数}

\begin{sphinxVerbatim}[commandchars=\\\{\}]
\PYG{n}{c}\PYG{o}{=}\PYG{n}{np}\PYG{o}{.}\PYG{n}{array}\PYG{p}{(}\PYG{p}{[}\PYG{o}{\PYGZhy{}}\PYG{l+m+mi}{2}\PYG{p}{,}\PYG{o}{\PYGZhy{}}\PYG{l+m+mi}{3}\PYG{p}{,}\PYG{l+m+mi}{5}\PYG{p}{]}\PYG{p}{)}   \PYG{c+c1}{\PYGZsh{} 目标函数系数,3x1列向量}
\PYG{n}{A}\PYG{o}{=}\PYG{n}{np}\PYG{o}{.}\PYG{n}{array}\PYG{p}{(}\PYG{p}{[}\PYG{p}{[}\PYG{o}{\PYGZhy{}}\PYG{l+m+mi}{2}\PYG{p}{,}\PYG{l+m+mi}{5}\PYG{p}{,}\PYG{o}{\PYGZhy{}}\PYG{l+m+mi}{1}\PYG{p}{]}\PYG{p}{,}\PYG{p}{[}\PYG{l+m+mi}{1}\PYG{p}{,}\PYG{l+m+mi}{3}\PYG{p}{,}\PYG{l+m+mi}{1}\PYG{p}{]}\PYG{p}{]}\PYG{p}{)} \PYG{c+c1}{\PYGZsh{} 不等式约束系数A,2x3维矩阵}
\PYG{n}{b}\PYG{o}{=}\PYG{n}{np}\PYG{o}{.}\PYG{n}{array}\PYG{p}{(}\PYG{p}{[}\PYG{o}{\PYGZhy{}}\PYG{l+m+mi}{10}\PYG{p}{,}\PYG{l+m+mi}{12}\PYG{p}{]}\PYG{p}{)}  \PYG{c+c1}{\PYGZsh{} 等式约束系数b, 2x1维列向量}
\PYG{n}{A\PYGZus{}eq}\PYG{o}{=}\PYG{n}{np}\PYG{o}{.}\PYG{n}{array}\PYG{p}{(}\PYG{p}{[}\PYG{p}{[}\PYG{l+m+mi}{1}\PYG{p}{,}\PYG{l+m+mi}{1}\PYG{p}{,}\PYG{l+m+mi}{1}\PYG{p}{]}\PYG{p}{]}\PYG{p}{)}  \PYG{c+c1}{\PYGZsh{} 等式约束系数Aeq,3x1维列向量}
\PYG{n}{b\PYGZus{}eq}\PYG{o}{=}\PYG{n}{np}\PYG{o}{.}\PYG{n}{array}\PYG{p}{(}\PYG{p}{[}\PYG{l+m+mi}{7}\PYG{p}{]}\PYG{p}{)}   \PYG{c+c1}{\PYGZsh{} 等式约束系数beq,1x1数值}
\end{sphinxVerbatim}

\sphinxstylestrong{Step5: 求解}

\begin{sphinxVerbatim}[commandchars=\\\{\}]
\PYG{n}{res}\PYG{o}{=}\PYG{n}{op}\PYG{o}{.}\PYG{n}{linprog}\PYG{p}{(}\PYG{n}{c}\PYG{p}{,}\PYG{n}{A}\PYG{p}{,}\PYG{n}{b}\PYG{p}{,}\PYG{n}{A\PYGZus{}eq}\PYG{p}{,}\PYG{n}{b\PYGZus{}eq}\PYG{p}{,}\PYG{n}{bounds}\PYG{o}{=}\PYG{p}{(}\PYG{n}{x1}\PYG{p}{,}\PYG{n}{x2}\PYG{p}{,}\PYG{n}{x3}\PYG{p}{)}\PYG{p}{)} \PYG{c+c1}{\PYGZsh{}调用函数进行求解}
\PYG{n}{res}
\end{sphinxVerbatim}

\begin{sphinxVerbatim}[commandchars=\\\{\}]
     con: array([1.19830323e\PYGZhy{}08])
     fun: \PYGZhy{}14.571428542312137
 message: \PYGZsq{}Optimization terminated successfully.\PYGZsq{}
     nit: 5
   slack: array([\PYGZhy{}3.70230904e\PYGZhy{}08,  3.85714287e+00])
  status: 0
 success: True
       x: array([6.42857141e+00, 5.71428573e\PYGZhy{}01, 9.82192085e\PYGZhy{}10])
\end{sphinxVerbatim}

即当 \(x_1 = 6.43, x_2 = 0.57, x_3 = 0\) 时,目标函数取得最大值 \(z = 14.57\)

\begin{sphinxadmonition}{note}{例题2}
\begin{equation*}
\begin{split}
\begin{array}{l}
&{\min z=2 x_{1}+3 x_{2}+x_{3}} \\
&\text { s.t. }{\quad\left\{\begin{array}{l}{x_{1}+4 x_{2}+2 x_{3} \geq 8} \\ {3 x_{1}+2 x_{2} \geq 6} \\ {x_{1}, x_{2}, x_{3} \geq 0}\end{array}\right.}\end{array}
\end{split}
\end{equation*}\end{sphinxadmonition}

\begin{sphinxVerbatim}[commandchars=\\\{\}]
\PYG{c+c1}{\PYGZsh{}导入相关库}
\PYG{k+kn}{import} \PYG{n+nn}{numpy} \PYG{k}{as} \PYG{n+nn}{np}
\PYG{k+kn}{from} \PYG{n+nn}{scipy} \PYG{k+kn}{import} \PYG{n}{optimize} \PYG{k}{as} \PYG{n}{op}

\PYG{c+c1}{\PYGZsh{}定义决策变量范围}
\PYG{n}{x1}\PYG{o}{=}\PYG{p}{(}\PYG{l+m+mi}{0}\PYG{p}{,}\PYG{k+kc}{None}\PYG{p}{)}
\PYG{n}{x2}\PYG{o}{=}\PYG{p}{(}\PYG{l+m+mi}{0}\PYG{p}{,}\PYG{k+kc}{None}\PYG{p}{)}
\PYG{n}{x3}\PYG{o}{=}\PYG{p}{(}\PYG{l+m+mi}{0}\PYG{p}{,}\PYG{k+kc}{None}\PYG{p}{)}

\PYG{c+c1}{\PYGZsh{}定义目标函数系数}
\PYG{n}{c}\PYG{o}{=}\PYG{n}{np}\PYG{o}{.}\PYG{n}{array}\PYG{p}{(}\PYG{p}{[}\PYG{l+m+mi}{2}\PYG{p}{,}\PYG{l+m+mi}{3}\PYG{p}{,}\PYG{l+m+mi}{1}\PYG{p}{]}\PYG{p}{)} 

\PYG{c+c1}{\PYGZsh{}定义约束条件系数}
\PYG{n}{A\PYGZus{}ub}\PYG{o}{=}\PYG{n}{np}\PYG{o}{.}\PYG{n}{array}\PYG{p}{(}\PYG{p}{[}\PYG{p}{[}\PYG{o}{\PYGZhy{}}\PYG{l+m+mi}{1}\PYG{p}{,}\PYG{o}{\PYGZhy{}}\PYG{l+m+mi}{4}\PYG{p}{,}\PYG{o}{\PYGZhy{}}\PYG{l+m+mi}{2}\PYG{p}{]}\PYG{p}{,}\PYG{p}{[}\PYG{o}{\PYGZhy{}}\PYG{l+m+mi}{3}\PYG{p}{,}\PYG{o}{\PYGZhy{}}\PYG{l+m+mi}{2}\PYG{p}{,}\PYG{l+m+mi}{0}\PYG{p}{]}\PYG{p}{]}\PYG{p}{)}
\PYG{n}{B\PYGZus{}ub}\PYG{o}{=}\PYG{n}{np}\PYG{o}{.}\PYG{n}{array}\PYG{p}{(}\PYG{p}{[}\PYG{o}{\PYGZhy{}}\PYG{l+m+mi}{10}\PYG{p}{,}\PYG{o}{\PYGZhy{}}\PYG{l+m+mi}{6}\PYG{p}{]}\PYG{p}{)}

\PYG{c+c1}{\PYGZsh{}求解}
\PYG{n}{res}\PYG{o}{=}\PYG{n}{op}\PYG{o}{.}\PYG{n}{linprog}\PYG{p}{(}\PYG{n}{c}\PYG{p}{,}\PYG{n}{A\PYGZus{}ub}\PYG{p}{,}\PYG{n}{B\PYGZus{}ub}\PYG{p}{,}\PYG{n}{bounds}\PYG{o}{=}\PYG{p}{(}\PYG{n}{x1}\PYG{p}{,}\PYG{n}{x2}\PYG{p}{,}\PYG{n}{x3}\PYG{p}{)}\PYG{p}{)}
\PYG{n}{res}
\end{sphinxVerbatim}

\begin{sphinxVerbatim}[commandchars=\\\{\}]
     con: array([], dtype=float64)
     fun: 8.000000000000306
 message: \PYGZsq{}Optimization terminated successfully.\PYGZsq{}
     nit: 4
   slack: array([ 8.81072992e\PYGZhy{}13, \PYGZhy{}2.70006240e\PYGZhy{}13])
  status: 0
 success: True
       x: array([0.91155217, 1.63267174, 1.27888043])
\end{sphinxVerbatim}

即当 \(x_1 = 0.8, x_2 = 1.8, x_3 = 0\) 时,目标函数取得最小值 \(z = 7\)


\section{使用Python求解包含非线性项的规划问题}
\label{\detokenize{docs/LP:id9}}
\begin{sphinxadmonition}{tip}{Tip:}\begin{itemize}
\item {} 
规划问题的分类
\begin{itemize}
\item {} 
\sphinxstylestrong{线性规划}: 在一组线性约束条件的限制下,求一线性目标函数最大或最小的问题;

\item {} 
\sphinxstylestrong{整数规划}:当约束条件加强,要求所有的自变量必须是整数时,成为整数规划(特别地,自变量只能为0或1时称为0\sphinxhyphen{}1规划);

\item {} 
\sphinxstylestrong{非线性规划}:无论是约束条件还是目标函数出现非线性项,那么规划问题就变成了非线性规划;

\item {} 
\sphinxstylestrong{多目标规划}:在一组约束条件的限制下,求多个目标函数最大或最小的问题;

\item {} 
\sphinxstylestrong{动态规划}:将优化目标函数分多阶段,利用阶段间的关系逐一进行求解的方法;

\end{itemize}

\end{itemize}
\end{sphinxadmonition}

有的时候,我们还需要求解包含了非线性项的非线性规划问题。事实上,针对非线性规划,没有一个统一的理论。这里仅提供一种可行的求解思路。

今后在讲优化算法的时候,会介绍如何使用现代优化算法求解非线性规划问题。

\begin{sphinxadmonition}{note}{例题3}

求解如下规划问题
\begin{equation*}
\begin{split}
\begin{array}{l}
&{\min f(x)=x_{1}^{2}+x_{2}^{2}+x_{3}^{2}+8} \\
&\text { s.t. }{\quad\left\{\begin{array}{l}{x_{1}^{2}-x_{2}+x_{3}^{2} \geq 0} \\ {x_{1}+x_{2}^{2}+x_{3}^{3} \leq 20} \\ {-x_{1}-x_{2}^{2}+2=0}\\
 {x_{2}+2 x_{3}^{2}=3}\\
 {x_{1}, x_{2}, x_{3} \geq 0}
\end{array}\right.}\end{array}
\end{split}
\end{equation*}\end{sphinxadmonition}

由于存在非线性项,不能沿用例一中的linprog函数求解,这里使用自定义函数的方法编写目标函数和约束条件,并使用scipy.optimize中的minimize函数求解。

\sphinxstylestrong{Step1:导入相关库}

\begin{sphinxVerbatim}[commandchars=\\\{\}]
\PYG{k+kn}{import} \PYG{n+nn}{numpy} \PYG{k}{as} \PYG{n+nn}{np}
\PYG{k+kn}{from} \PYG{n+nn}{scipy}\PYG{n+nn}{.}\PYG{n+nn}{optimize} \PYG{k+kn}{import} \PYG{n}{minimize}
\end{sphinxVerbatim}

\sphinxstylestrong{Step2:使用函数的形式表示目标和约束}

\begin{sphinxVerbatim}[commandchars=\\\{\}]
\PYG{c+c1}{\PYGZsh{} 定义目标函数}
\PYG{k}{def} \PYG{n+nf}{objective}\PYG{p}{(}\PYG{n}{x}\PYG{p}{)}\PYG{p}{:}
    \PYG{k}{return} \PYG{n}{x}\PYG{p}{[}\PYG{l+m+mi}{0}\PYG{p}{]} \PYG{o}{*}\PYG{o}{*} \PYG{l+m+mi}{2} \PYG{o}{+} \PYG{n}{x}\PYG{p}{[}\PYG{l+m+mi}{1}\PYG{p}{]}\PYG{o}{*}\PYG{o}{*}\PYG{l+m+mi}{2} \PYG{o}{+} \PYG{n}{x}\PYG{p}{[}\PYG{l+m+mi}{2}\PYG{p}{]}\PYG{o}{*}\PYG{o}{*}\PYG{l+m+mi}{2} \PYG{o}{+}\PYG{l+m+mi}{8}

\PYG{c+c1}{\PYGZsh{} 定义约束条件}
\PYG{k}{def} \PYG{n+nf}{constraint1}\PYG{p}{(}\PYG{n}{x}\PYG{p}{)}\PYG{p}{:}
    \PYG{k}{return} \PYG{n}{x}\PYG{p}{[}\PYG{l+m+mi}{0}\PYG{p}{]} \PYG{o}{*}\PYG{o}{*} \PYG{l+m+mi}{2} \PYG{o}{\PYGZhy{}} \PYG{n}{x}\PYG{p}{[}\PYG{l+m+mi}{1}\PYG{p}{]} \PYG{o}{+} \PYG{n}{x}\PYG{p}{[}\PYG{l+m+mi}{2}\PYG{p}{]}\PYG{o}{*}\PYG{o}{*}\PYG{l+m+mi}{2}  \PYG{c+c1}{\PYGZsh{} 不等约束}

\PYG{k}{def} \PYG{n+nf}{constraint2}\PYG{p}{(}\PYG{n}{x}\PYG{p}{)}\PYG{p}{:}
    \PYG{k}{return} \PYG{o}{\PYGZhy{}}\PYG{p}{(}\PYG{n}{x}\PYG{p}{[}\PYG{l+m+mi}{0}\PYG{p}{]} \PYG{o}{+} \PYG{n}{x}\PYG{p}{[}\PYG{l+m+mi}{1}\PYG{p}{]}\PYG{o}{*}\PYG{o}{*}\PYG{l+m+mi}{2} \PYG{o}{+} \PYG{n}{x}\PYG{p}{[}\PYG{l+m+mi}{2}\PYG{p}{]}\PYG{o}{*}\PYG{o}{*}\PYG{l+m+mi}{2}\PYG{o}{\PYGZhy{}}\PYG{l+m+mi}{20}\PYG{p}{)}  \PYG{c+c1}{\PYGZsh{} 不等约束}

\PYG{k}{def} \PYG{n+nf}{constraint3}\PYG{p}{(}\PYG{n}{x}\PYG{p}{)}\PYG{p}{:}
    \PYG{k}{return} \PYG{o}{\PYGZhy{}}\PYG{n}{x}\PYG{p}{[}\PYG{l+m+mi}{0}\PYG{p}{]} \PYG{o}{\PYGZhy{}} \PYG{n}{x}\PYG{p}{[}\PYG{l+m+mi}{1}\PYG{p}{]}\PYG{o}{*}\PYG{o}{*}\PYG{l+m+mi}{2} \PYG{o}{+} \PYG{l+m+mi}{2}        \PYG{c+c1}{\PYGZsh{} 等式约束}

\PYG{k}{def} \PYG{n+nf}{constraint4}\PYG{p}{(}\PYG{n}{x}\PYG{p}{)}\PYG{p}{:}
    \PYG{k}{return} \PYG{n}{x}\PYG{p}{[}\PYG{l+m+mi}{1}\PYG{p}{]} \PYG{o}{+} \PYG{l+m+mi}{2}\PYG{o}{*}\PYG{n}{x}\PYG{p}{[}\PYG{l+m+mi}{2}\PYG{p}{]}\PYG{o}{*}\PYG{o}{*}\PYG{l+m+mi}{2} \PYG{o}{\PYGZhy{}}\PYG{l+m+mi}{3}           \PYG{c+c1}{\PYGZsh{} 等式约束}
\end{sphinxVerbatim}

注意:每一个函数的输入为一个 \(n\)维列向量 \(x\),其中 \(x[0]\)表示该列向量的第一个元素,即 \(x_1\)。

\sphinxstylestrong{Step3:定义约束条件}

\begin{sphinxVerbatim}[commandchars=\\\{\}]
\PYG{n}{con1} \PYG{o}{=} \PYG{p}{\PYGZob{}}\PYG{l+s+s1}{\PYGZsq{}}\PYG{l+s+s1}{type}\PYG{l+s+s1}{\PYGZsq{}}\PYG{p}{:} \PYG{l+s+s1}{\PYGZsq{}}\PYG{l+s+s1}{ineq}\PYG{l+s+s1}{\PYGZsq{}}\PYG{p}{,} \PYG{l+s+s1}{\PYGZsq{}}\PYG{l+s+s1}{fun}\PYG{l+s+s1}{\PYGZsq{}}\PYG{p}{:} \PYG{n}{constraint1}\PYG{p}{\PYGZcb{}}
\PYG{n}{con2} \PYG{o}{=} \PYG{p}{\PYGZob{}}\PYG{l+s+s1}{\PYGZsq{}}\PYG{l+s+s1}{type}\PYG{l+s+s1}{\PYGZsq{}}\PYG{p}{:} \PYG{l+s+s1}{\PYGZsq{}}\PYG{l+s+s1}{ineq}\PYG{l+s+s1}{\PYGZsq{}}\PYG{p}{,} \PYG{l+s+s1}{\PYGZsq{}}\PYG{l+s+s1}{fun}\PYG{l+s+s1}{\PYGZsq{}}\PYG{p}{:} \PYG{n}{constraint2}\PYG{p}{\PYGZcb{}}
\PYG{n}{con3} \PYG{o}{=} \PYG{p}{\PYGZob{}}\PYG{l+s+s1}{\PYGZsq{}}\PYG{l+s+s1}{type}\PYG{l+s+s1}{\PYGZsq{}}\PYG{p}{:} \PYG{l+s+s1}{\PYGZsq{}}\PYG{l+s+s1}{eq}\PYG{l+s+s1}{\PYGZsq{}}\PYG{p}{,} \PYG{l+s+s1}{\PYGZsq{}}\PYG{l+s+s1}{fun}\PYG{l+s+s1}{\PYGZsq{}}\PYG{p}{:} \PYG{n}{constraint3}\PYG{p}{\PYGZcb{}}
\PYG{n}{con4} \PYG{o}{=} \PYG{p}{\PYGZob{}}\PYG{l+s+s1}{\PYGZsq{}}\PYG{l+s+s1}{type}\PYG{l+s+s1}{\PYGZsq{}}\PYG{p}{:} \PYG{l+s+s1}{\PYGZsq{}}\PYG{l+s+s1}{eq}\PYG{l+s+s1}{\PYGZsq{}}\PYG{p}{,} \PYG{l+s+s1}{\PYGZsq{}}\PYG{l+s+s1}{fun}\PYG{l+s+s1}{\PYGZsq{}}\PYG{p}{:} \PYG{n}{constraint4}\PYG{p}{\PYGZcb{}}

\PYG{c+c1}{\PYGZsh{} 4个约束条件}
\PYG{n}{cons} \PYG{o}{=} \PYG{p}{(}\PYG{p}{[}\PYG{n}{con1}\PYG{p}{,} \PYG{n}{con2}\PYG{p}{,} \PYG{n}{con3}\PYG{p}{,}\PYG{n}{con4}\PYG{p}{]}\PYG{p}{)}  

\PYG{c+c1}{\PYGZsh{} 决策变量的符号约束}
\PYG{n}{b} \PYG{o}{=} \PYG{p}{(}\PYG{l+m+mf}{0.0}\PYG{p}{,} \PYG{k+kc}{None}\PYG{p}{)} \PYG{c+c1}{\PYGZsh{}即决策变量的取值范围为大于等于0}
\PYG{n}{bnds} \PYG{o}{=} \PYG{p}{(}\PYG{n}{b}\PYG{p}{,} \PYG{n}{b} \PYG{p}{,}\PYG{n}{b}\PYG{p}{)} 
\end{sphinxVerbatim}

注意:每一个约束为一个字典,其中 \sphinxcode{\sphinxupquote{type}} 表示约束类型:\sphinxcode{\sphinxupquote{ineq}}为大于等于,\sphinxcode{\sphinxupquote{eq}}为等于; \sphinxcode{\sphinxupquote{fun}} 表示约束函数表达式,即step2中的自定义函数。

\sphinxstylestrong{Step4:求解}

\begin{sphinxVerbatim}[commandchars=\\\{\}]
\PYG{n}{x0}\PYG{o}{=}\PYG{n}{np}\PYG{o}{.}\PYG{n}{array}\PYG{p}{(}\PYG{p}{[}\PYG{l+m+mi}{0}\PYG{p}{,} \PYG{l+m+mi}{0}\PYG{p}{,} \PYG{l+m+mi}{0}\PYG{p}{]}\PYG{p}{)} \PYG{c+c1}{\PYGZsh{}定义初始值}
\PYG{n}{solution} \PYG{o}{=} \PYG{n}{minimize}\PYG{p}{(}\PYG{n}{objective}\PYG{p}{,} \PYG{n}{x0}\PYG{p}{,} \PYG{n}{method}\PYG{o}{=}\PYG{l+s+s1}{\PYGZsq{}}\PYG{l+s+s1}{SLSQP}\PYG{l+s+s1}{\PYGZsq{}}\PYG{p}{,} \PYGZbs{}
                    \PYG{n}{bounds}\PYG{o}{=}\PYG{n}{bnds}\PYG{p}{,} \PYG{n}{constraints}\PYG{o}{=}\PYG{n}{cons}\PYG{p}{)}
\end{sphinxVerbatim}

注意:minimize为最小化目标函数,且约束条件中默认为大于等于约束。

\sphinxstylestrong{Step5:打印求解结果}

\begin{sphinxVerbatim}[commandchars=\\\{\}]
\PYG{n}{x} \PYG{o}{=} \PYG{n}{solution}\PYG{o}{.}\PYG{n}{x}

\PYG{n+nb}{print}\PYG{p}{(}\PYG{l+s+s1}{\PYGZsq{}}\PYG{l+s+s1}{目标值: }\PYG{l+s+s1}{\PYGZsq{}} \PYG{o}{+} \PYG{n+nb}{str}\PYG{p}{(}\PYG{n}{objective}\PYG{p}{(}\PYG{n}{x}\PYG{p}{)}\PYG{p}{)}\PYG{p}{)}
\PYG{n+nb}{print}\PYG{p}{(}\PYG{l+s+s1}{\PYGZsq{}}\PYG{l+s+s1}{最优解为}\PYG{l+s+s1}{\PYGZsq{}}\PYG{p}{)}
\PYG{n+nb}{print}\PYG{p}{(}\PYG{l+s+s1}{\PYGZsq{}}\PYG{l+s+s1}{x1 = }\PYG{l+s+s1}{\PYGZsq{}} \PYG{o}{+} \PYG{n+nb}{str}\PYG{p}{(}\PYG{n+nb}{round}\PYG{p}{(}\PYG{n}{x}\PYG{p}{[}\PYG{l+m+mi}{0}\PYG{p}{]}\PYG{p}{,}\PYG{l+m+mi}{2}\PYG{p}{)}\PYG{p}{)}\PYG{p}{)}
\PYG{n+nb}{print}\PYG{p}{(}\PYG{l+s+s1}{\PYGZsq{}}\PYG{l+s+s1}{x2 = }\PYG{l+s+s1}{\PYGZsq{}} \PYG{o}{+} \PYG{n+nb}{str}\PYG{p}{(}\PYG{n+nb}{round}\PYG{p}{(}\PYG{n}{x}\PYG{p}{[}\PYG{l+m+mi}{1}\PYG{p}{]}\PYG{p}{,}\PYG{l+m+mi}{2}\PYG{p}{)}\PYG{p}{)}\PYG{p}{)}
\PYG{n+nb}{print}\PYG{p}{(}\PYG{l+s+s1}{\PYGZsq{}}\PYG{l+s+s1}{x3 = }\PYG{l+s+s1}{\PYGZsq{}} \PYG{o}{+} \PYG{n+nb}{str}\PYG{p}{(}\PYG{n+nb}{round}\PYG{p}{(}\PYG{n}{x}\PYG{p}{[}\PYG{l+m+mi}{2}\PYG{p}{]}\PYG{p}{,}\PYG{l+m+mi}{2}\PYG{p}{)}\PYG{p}{)}\PYG{p}{)}
\PYG{n}{solution}
\end{sphinxVerbatim}

\begin{sphinxVerbatim}[commandchars=\\\{\}]
目标值: 10.651091840572583
最优解为
x1 = 0.55
x2 = 1.2
x3 = 0.95
\end{sphinxVerbatim}

\begin{sphinxVerbatim}[commandchars=\\\{\}]
     fun: 10.651091840572583
     jac: array([1.10433471, 2.40651834, 1.89564812])
 message: \PYGZsq{}Optimization terminated successfully\PYGZsq{}
    nfev: 71
     nit: 15
    njev: 15
  status: 0
 success: True
       x: array([0.55216734, 1.20325918, 0.94782404])
\end{sphinxVerbatim}

即当 \(x_1 = 0.55, x_2 = 1.2, x_3 = 0.95\) 时,目标函数取得最小值 \(z = 10.65\)


\section{从整数规划到0\sphinxhyphen{}1规划}
\label{\detokenize{docs/LP:id10}}
\begin{sphinxadmonition}{attention}{Attention:}
本节为拓展内容,仅做了解,不要求掌握
\end{sphinxadmonition}


\subsection{整数规划模型}
\label{\detokenize{docs/LP:id11}}
规划中的变量(部分或全部)限制为整数时,称为整数规划。若在线性规划模型中,变量限制为整数,则称为整数线性规划。

\sphinxstylestrong{当决策变量均为整数时,称纯整数规划;}

\sphinxstylestrong{当决策变量中部分为整数,部分为实数时,称混合整数规划;}

一般用 \(x , integer\) 表示 \(x\) 为整数

将第一节中的线性规划图解法的例子添加整数约束,则可行域变为了多边形内的整点,如下图所示:

\sphinxincludegraphics{{500}.png}

可以看出,可行域变成了离散的点,这也使得整数规划问题比线性规划问题要更难求解,但现实中的许多决策变量都只能取整数,因此混合整数规划问题也成为了了研究最多的线性规划问题。

\sphinxstylestrong{注意:整数规划最优解不能按照实数最优解简单取整而获得},但简单取整后,再进行邻域搜索不失为一种有用的解法。

整数规划的两个常用求解方法:
\begin{itemize}
\item {} 
分支定界算法

\item {} 
割平面法

\end{itemize}

\sphinxstylestrong{分枝定界法}
\begin{itemize}
\item {} 
\sphinxstylestrong{step1}不考虑整数约束的情况下求解得到最优解 \(x^*\)(一般不是整数);

\item {} 
\sphinxstylestrong{step2}以该解的上下整数界限建立新的约束,将原整数规划问题变为两个问题(\sphinxstylestrong{分枝});

\item {} 
\sphinxstylestrong{step3}分别对两个子问题求解(不考虑整数约束),若解刚好为整数解则结束;若不为整数解则继续进行分枝;

\item {} 
\sphinxstylestrong{step4}以最开始的目标函数值作为上界,子问题求解中得到的任一整数解为下界(\sphinxstylestrong{定界}),对子问题进行剪枝,减小问题规模;

\item {} 
\sphinxstylestrong{step5}重复以上步骤直到得到最优解

\end{itemize}

\sphinxstylestrong{割平面法}
\begin{itemize}
\item {} 
\sphinxstylestrong{step1}不考虑整数约束的情况下求解得到最优解 \(x^*\)(一般不是整数);

\item {} 
\sphinxstylestrong{step2}过该解做一个割平面(二维情况下为一条直线),缩小可行域;

\item {} 
\sphinxstylestrong{step3}在缩小后的可行域中求最优解(不考虑整数约束)

\item {} 
\sphinxstylestrong{step4}重复步骤2和步骤3,直到最优解满足整数约束

\end{itemize}

\sphinxstylestrong{由于模型求解可以直接使用求解器(主要是Lingo),这里仅提供求解思路,如果在比赛中遇到该类问题,可以再查阅相关资料。}


\subsection{0\sphinxhyphen{}1规划模型}
\label{\detokenize{docs/LP:id12}}
当整数规划问题中的整数型决策变量限制为只能取0或1时,称为0\sphinxhyphen{}1整数规划,简称为0\sphinxhyphen{}1规划。

因为0\sphinxhyphen{}1规划问题的解空间比一般的整数规划问题较少,求解起来较为容易,且所有的整数规划问题都可以化为0\sphinxhyphen{}1规划问题,所以在建立混合整数规划模型求解实际问题时,应尽量使用0\sphinxhyphen{}1决策变量进行建模。

\sphinxstylestrong{例如}:有十个工厂可供决策时,可以使用10个0\sphinxhyphen{}1变量,当取值为0时时代表不使用这个工厂,取值为1时使用该工厂。

0\sphinxhyphen{}1规划的常用求解方法:\sphinxstylestrong{分支定界算法、割平面法、隐枚举法}

\begin{sphinxadmonition}{note}{\sphinxstylestrong{0\sphinxhyphen{}1规划应用举例} : 指派问题}

拟分配\(n\)个人去做\(n\)项工作,每人干且仅干一项工作,若分配第\(i\)人去干第\(j\)项工作,需花费\(c_{ij}\)时间,问应该如何分配工作才能使工人们总的花费时间最少?
\end{sphinxadmonition}

\sphinxstylestrong{解:} 引入变量\(x_{ij}\),若分配\(i\)做第\(j\)项工作,则取\(x_{ij} =1\),否则\(x_{ij} = 0\),上述指派问题的数学模型为:
\begin{equation*}
\begin{split}
\begin{aligned}
&{\min \quad \sum_{i=1}^{n} \sum_{j=1}^{n} c_{i j} x_{i j}}\\
s.t.&\left\{\begin{array}{ll}
{\displaystyle\sum_{j=1}^{n} x_{i j}=1} \\
{\displaystyle\sum_{i=1}^{n} x_{i j}=1} \\
{x_{i j}=0 或 1}
\end{array}\right.
\end{aligned}
\end{split}
\end{equation*}

\section{案例分析}
\label{\detokenize{docs/LP:id13}}
\begin{sphinxadmonition}{note}{案例1:投资的收益和风险}

市场上有\(n\)种资产\(s_i(i = 1,2,\cdots,n)\)可以选择,现用数额为\(M\)的相当大的资金作一个时期的投资。购买这\(n\)种资产的收益率为\(r_i\),风险损失率为\(q_i\),投资越分散,总的风险越少,总体风险可以用投资\(s_i\)中最大的一个风险来度量。

此外,购买\(s_i\)时还要付交易费(费率为\(p_i\)),当购买额不超过给定值\(u_i\)时,交易费按购买额\(u_i\)计算。另外,假定同期银行存款利率为\(r_0\),既无交易费又无风险(\(r_0 = 5\%\))。已知\(n=4\)时相关数据如下表
\begin{equation*}
\begin{split}
\begin{array}{|c|c|c|c|c|}\hline s_{i} & {r_{i}(\%)} & {q_{i}} & {p_{i}(\%)} & {u_{i}(\text{RMB})} \\ 
\hline s_{1} & {28} & {2.5} & {1} & {103} \\ 
\hline s_{2} & {21} & {1.5} & {2} & {198} \\ 
\hline s_{3} & {23} & {5.5} & {4.5} & {52} \\ 
\hline s_{4} & {25} & {2.6} & {6.5} & {40} \\ 
\hline\end{array}
\end{split}
\end{equation*}
试给该公司设计一种投资组合方案,用给定的资金\(M\),有选择地购买若干种资产或者银行生息,\sphinxstylestrong{使净收益最大,总体风险最小}。
\end{sphinxadmonition}

符号规定
\begin{itemize}
\item {} 
\(s_i\)  :第\(i\)种投资项目,如股票、债券

\item {} 
\(r_i,p_i,q_i\):分别为\(s_i\)的平均收益率,交易费率,风险损失率

\item {} 
\(u_i\) : \(s_i\)的交易定额

\item {} 
\(r_0\) :同期银行利率

\item {} 
\(x_i\) : 投资项目\(s_i\)的资金

\item {} 
\(a\) : 投资风险度

\item {} 
\(Q\) : 总体收益

\end{itemize}

模型假设:
\begin{enumerate}
\sphinxsetlistlabels{\arabic}{enumi}{enumii}{}{.}%
\item {} 
投资数额M相当大,为了便于计算,假设\(M=1\)

\item {} 
投资越分散,总风险越小,总体风险用投资项目\(s_i\)中最大的一个风险来度量

\item {} 
\(n\)种资产\(s_i\)之间是相互独立的

\item {} 
在投资的这一时期内,\(r_i,p_i,q_i,r_0\)为定值

\end{enumerate}


\subsection{建立与简化模型}
\label{\detokenize{docs/LP:id14}}
根据模型假设和符号规定,我们可以写出模型的第一个优化目标为总体风险尽可能小,而总体风险是所有投资中风险最大的一个
\begin{equation*}
\begin{split}
\min \max \{q_ix_i | i=1,2,...n\}
\end{split}
\end{equation*}
第二个优化目标为净收益尽可能大。根据题意,交易费用为一个分段函数(非线性函数),因此需要进行简化:由于题目给定的定值 \(u_i\) 相对于总投资额 \(M\) 很小,可以忽略不计,因此将交易费简化为 \(p_i x_i\),所以目标函数为
\begin{equation*}
\begin{split}
\max \sum_{i=1}^n (r_i - p_i) x_i 
\end{split}
\end{equation*}
\sphinxstylestrong{对于一个多目标优化模型,常用的考虑方式为先固定其中一个目标,再优化另一个目标。}

在本题中,可以给定一个投资者能够承受的风险界限 \(a\),使得最大投资风险下损失比例小于 \(a\),即 \(\dfrac{q_i x_i}{M} < a\),将其作为新的约束,就可以把多目标优化转化为单目标优化,即:
\begin{equation*}
\begin{split}
\begin{aligned}
&{\max \sum_{i=1}^n (r_i - p_i) x_i }\\
s.t.&\left\{\begin{array}{ll}
{\dfrac{q_i x_i}{M} < a } \\
{\displaystyle\sum_{i=1}^n (1+p_i)x_i = M} \\
{x_i \geq 0, i=1,2,...n}
\end{array}\right.
\end{aligned}
\end{split}
\end{equation*}

\subsection{使用python scipy库求解}
\label{\detokenize{docs/LP:python-scipy}}
\(a\) 反映了投资者对风险的偏好程度,从 \(a = 0\) 开始,以步长为0.001进行循环搜索,使用python编写代码如下

\begin{sphinxVerbatim}[commandchars=\\\{\}]
\PYG{c+c1}{\PYGZsh{}导入相关库}
\PYG{k+kn}{import} \PYG{n+nn}{numpy} \PYG{k}{as} \PYG{n+nn}{np}
\PYG{k+kn}{import} \PYG{n+nn}{matplotlib}\PYG{n+nn}{.}\PYG{n+nn}{pyplot} \PYG{k}{as} \PYG{n+nn}{plt}
\PYG{o}{\PYGZpc{}}\PYG{k}{matplotlib} inline
\PYG{k+kn}{import} \PYG{n+nn}{scipy}\PYG{n+nn}{.}\PYG{n+nn}{optimize} \PYG{k}{as} \PYG{n+nn}{op}

\PYG{c+c1}{\PYGZsh{}定义a的取值}
\PYG{n}{a} \PYG{o}{=} \PYG{l+m+mi}{0}
\PYG{n}{profit\PYGZus{}list} \PYG{o}{=} \PYG{p}{[}\PYG{p}{]} \PYG{c+c1}{\PYGZsh{}记录最大收益}
\PYG{n}{a\PYGZus{}list} \PYG{o}{=} \PYG{p}{[}\PYG{p}{]} \PYG{c+c1}{\PYGZsh{}记录a的取值}


\PYG{k}{while} \PYG{n}{a}\PYG{o}{\PYGZlt{}}\PYG{l+m+mf}{0.05}\PYG{p}{:}
    \PYG{c+c1}{\PYGZsh{}定义决策变量取值范围}
    \PYG{n}{x1}\PYG{o}{=}\PYG{p}{(}\PYG{l+m+mi}{0}\PYG{p}{,}\PYG{k+kc}{None}\PYG{p}{)}

    \PYG{c+c1}{\PYGZsh{}定义目标函数系数}
    \PYG{n}{c}\PYG{o}{=}\PYG{n}{np}\PYG{o}{.}\PYG{n}{array}\PYG{p}{(}\PYG{p}{[}\PYG{o}{\PYGZhy{}}\PYG{l+m+mf}{0.05}\PYG{p}{,}\PYG{o}{\PYGZhy{}}\PYG{l+m+mf}{0.27}\PYG{p}{,}\PYG{o}{\PYGZhy{}}\PYG{l+m+mf}{0.19}\PYG{p}{,}\PYG{o}{\PYGZhy{}}\PYG{l+m+mf}{0.185}\PYG{p}{,}\PYG{o}{\PYGZhy{}}\PYG{l+m+mf}{0.185}\PYG{p}{]}\PYG{p}{)} 
    \PYG{c+c1}{\PYGZsh{}定义不等式约束条件左边系数}
    \PYG{n}{A} \PYG{o}{=} \PYG{n}{np}\PYG{o}{.}\PYG{n}{hstack}\PYG{p}{(}\PYG{p}{(}\PYG{n}{np}\PYG{o}{.}\PYG{n}{zeros}\PYG{p}{(}\PYG{p}{(}\PYG{l+m+mi}{4}\PYG{p}{,}\PYG{l+m+mi}{1}\PYG{p}{)}\PYG{p}{)}\PYG{p}{,}\PYG{n}{np}\PYG{o}{.}\PYG{n}{diag}\PYG{p}{(}\PYG{p}{[}\PYG{l+m+mf}{0.025}\PYG{p}{,}\PYG{l+m+mf}{0.015}\PYG{p}{,}\PYG{l+m+mf}{0.055}\PYG{p}{,}\PYG{l+m+mf}{0.026}\PYG{p}{]}\PYG{p}{)}\PYG{p}{)}\PYG{p}{)}
    \PYG{c+c1}{\PYGZsh{}定义不等式约束条件右边系数}
    \PYG{n}{b}\PYG{o}{=}\PYG{n}{a}\PYG{o}{*}\PYG{n}{np}\PYG{o}{.}\PYG{n}{ones}\PYG{p}{(}\PYG{p}{(}\PYG{l+m+mi}{4}\PYG{p}{,}\PYG{l+m+mi}{1}\PYG{p}{)}\PYG{p}{)}\PYG{p}{;}
    \PYG{c+c1}{\PYGZsh{}定义等式约束条件左边系数}
    \PYG{n}{Aeq}\PYG{o}{=}\PYG{n}{np}\PYG{o}{.}\PYG{n}{array}\PYG{p}{(}\PYG{p}{[}\PYG{p}{[}\PYG{l+m+mi}{1}\PYG{p}{,}\PYG{l+m+mf}{1.01}\PYG{p}{,}\PYG{l+m+mf}{1.02}\PYG{p}{,}\PYG{l+m+mf}{1.045}\PYG{p}{,}\PYG{l+m+mf}{1.065}\PYG{p}{]}\PYG{p}{]}\PYG{p}{)}
    \PYG{c+c1}{\PYGZsh{}定义等式约束条件右边系数}
    \PYG{n}{beq}\PYG{o}{=}\PYG{n}{np}\PYG{o}{.}\PYG{n}{array}\PYG{p}{(}\PYG{p}{[}\PYG{l+m+mi}{1}\PYG{p}{]}\PYG{p}{)}\PYG{p}{;}
    \PYG{c+c1}{\PYGZsh{}求解}
    \PYG{n}{res}\PYG{o}{=}\PYG{n}{op}\PYG{o}{.}\PYG{n}{linprog}\PYG{p}{(}\PYG{n}{c}\PYG{p}{,}\PYG{n}{A}\PYG{p}{,}\PYG{n}{b}\PYG{p}{,}\PYG{n}{Aeq}\PYG{p}{,}\PYG{n}{beq}\PYG{p}{,}\PYG{n}{bounds}\PYG{o}{=}\PYG{p}{(}\PYG{n}{x1}\PYG{p}{,}\PYG{n}{x1}\PYG{p}{,}\PYG{n}{x1}\PYG{p}{,}\PYG{n}{x1}\PYG{p}{,}\PYG{n}{x1}\PYG{p}{)}\PYG{p}{)}
    \PYG{n}{profit} \PYG{o}{=} \PYG{o}{\PYGZhy{}}\PYG{n}{res}\PYG{o}{.}\PYG{n}{fun}
    \PYG{n}{profit\PYGZus{}list}\PYG{o}{.}\PYG{n}{append}\PYG{p}{(}\PYG{n}{profit}\PYG{p}{)}
    \PYG{n}{a\PYGZus{}list}\PYG{o}{.}\PYG{n}{append}\PYG{p}{(}\PYG{n}{a}\PYG{p}{)}
    \PYG{n}{a} \PYG{o}{=} \PYG{n}{a}\PYG{o}{+}\PYG{l+m+mf}{0.001}

\PYG{c+c1}{\PYGZsh{}绘制风险偏好a与最大收益的曲线图    }
\PYG{n}{plt}\PYG{o}{.}\PYG{n}{figure}\PYG{p}{(}\PYG{n}{figsize}\PYG{o}{=}\PYG{p}{(}\PYG{l+m+mi}{10}\PYG{p}{,}\PYG{l+m+mi}{7}\PYG{p}{)}\PYG{p}{)}
\PYG{n}{plt}\PYG{o}{.}\PYG{n}{plot}\PYG{p}{(}\PYG{n}{a\PYGZus{}list}\PYG{p}{,}\PYG{n}{profit\PYGZus{}list}\PYG{p}{)}
\PYG{n}{plt}\PYG{o}{.}\PYG{n}{xlabel}\PYG{p}{(}\PYG{l+s+s1}{\PYGZsq{}}\PYG{l+s+s1}{a}\PYG{l+s+s1}{\PYGZsq{}}\PYG{p}{)}\PYG{p}{;}\PYG{n}{plt}\PYG{o}{.}\PYG{n}{ylabel}\PYG{p}{(}\PYG{l+s+s1}{\PYGZsq{}}\PYG{l+s+s1}{Profit}\PYG{l+s+s1}{\PYGZsq{}}\PYG{p}{)}
\end{sphinxVerbatim}

\begin{sphinxVerbatim}[commandchars=\\\{\}]
Text(0, 0.5, \PYGZsq{}Profit\PYGZsq{})
\end{sphinxVerbatim}

\noindent\sphinxincludegraphics{{LP_51_1}.png}

\sphinxstylestrong{从上图中可以看出}
\begin{enumerate}
\sphinxsetlistlabels{\arabic}{enumi}{enumii}{}{.}%
\item {} 
风险越大,收益也就越大;

\item {} 
当投资越分散时,投资者承担的风险越小,这与题意一致。即:
冒险的投资者会出现集中投资的情况,保守的投资者则尽量分散投资。

\item {} 
在 \(a=0.006\) 附近有一个转折点,在这点左边,风险增加很少时,利润增长很快。在这一点右边,风险增加很大时,利润增长很缓慢,所以对于风险和收益没有特殊偏好的投资者来说,应该选择曲线的拐点作为最优投资组合,大约是 \(a=0.6\%\),总体收益为 \(Q=20\%\),所对应投资方案为:风险度 \(a = 0.006\),收益 \(Q = 0.2019\),\(x_0=0,x_1=0.24,x_2=0.4,x_3=0.1091, x_4=0.2212\)

\end{enumerate}

\sphinxstylestrong{模型求解的其他思路}

在上面的例子中,我们使用固定风险水平来最大化收益的方法来将多目标转化为单目标,也可考虑其他思路:

1.在总盈利在水平 \(k\) 以上的情况下,寻找风险最低的投资方案,即:
\begin{equation*}
\begin{split}
\begin{aligned}
&{\min \left\{ \max {q_i x_i} \right\}}\\
\text{s.t.} & {\left\{ \begin{array}{l}
{\displaystyle\sum_{i=1}^n (r_i - p_i) x_i \geq k}\\
{ \displaystyle\sum_{i=1}^n (1+p_i)x_i = M}\\
{ x_i \geq 0, i=1,2,...n}
\end{array} \right.}
\end{aligned}
\end{split}
\end{equation*}
2.对风险和收益赋予权重 \(s(0 \leq s \leq 1)\) 和 \(1-s\), \(s\)成为投资偏好系数,即
\begin{equation*}
\begin{split}
\begin{aligned}
& {\min \quad s\max \{q_ix_i\} - (1-s)\sum_{i=1}^n (r_i - p_i) x_i } \\
\text{s.t.}&\left\{\begin{array}{l}
{\displaystyle \sum_{i=1}^n (1+p_i)x_i = M }\\
{  x_i \geq 0, i=1,2,...n}
\end{array}\right.
\end{aligned}
\end{split}
\end{equation*}
\begin{sphinxadmonition}{note}{案例2:运输问题(产销平衡)}

某商品有\(m\)个产地,\(n\)个销地,各产地的产量分别为\(a_1,a_2,\cdots,a_m\),各销地的需求量分别为\(b_1,b_2,\cdots,b_n\)。若该商品由\(i\)地运到\(j\)地的单位运价为\(c_{ij}\),问应该如何调运才能使总运费最省?
\end{sphinxadmonition}

\sphinxstylestrong{解:} 引入变量\(x_{ij}\),其取值为由\(i\)地运往\(j\)地的商品数量,数学模型为
\begin{equation*}
\begin{split}
\begin{aligned}
&\min \sum_{i=1}^{m} \sum_{j=1}^{n} c_{i j} x_{i j}\\
s.t.&\left\{\begin{array}{ll}
{\displaystyle \sum_{j=1}^{n} x_{i j}=a_{i},} & {i=1, \cdots, m} \\
{\displaystyle \sum_{i=1}^{m} x_{i j}=b_{j},} & {j=1,2, \cdots, n} \\
{x_{i j} \geq 0}
\end{array}\right.
\end{aligned}
\end{split}
\end{equation*}

\section{练习作业}
\label{\detokenize{docs/LP:id15}}
\begin{sphinxadmonition}{note}{作业1:}
\begin{itemize}
\item {} 
请使用Python \sphinxcode{\sphinxupquote{scipy}}库 的\sphinxcode{\sphinxupquote{optimize.linprog}}方法,求解以下线性规划问题,并通过图解法验证。

\end{itemize}
\begin{equation*}
\begin{split}
\begin{array}{l}
&{\max z= 4x_{1}+ 3x_{2}} \\
&\text { s.t. }{\quad\left\{\begin{array}{l}
{2x_{1}+ x_{2} \leq 10} \\ 
{x_{1}+ x_{2} \leq 8} \\ 
{x_{1}, x_{2} \geq 0}
\end{array}\right.}\end{array}
\end{split}
\end{equation*}\end{sphinxadmonition}

\begin{sphinxadmonition}{note}{作业2:}
\begin{itemize}
\item {} 
请使用Python \sphinxcode{\sphinxupquote{scipy}}库 的\sphinxcode{\sphinxupquote{optimize.minimize}}方法,求解以下非线性规划问题

\end{itemize}
\begin{equation*}
\begin{split}
\begin{array}{l}
&{\min z= x_{1}^2 + x_{2}^2 +x_{3}^2} \\
&\text { s.t. }{\quad\left\{\begin{array}{l}
{x_1+x_2 + x_3\geq9 } \\ 
{ x_{1}, x_{2},x_3 \geq 0}
\end{array}\right.}\end{array}
\end{split}
\end{equation*}\end{sphinxadmonition}

\begin{sphinxadmonition}{note}{作业3:}
\begin{itemize}
\item {} 
某农场 I,II,III 等耕地的面积分别为 \(100 hm^2\)、\(300 hm^2\) 和 \(200 hm^2\),计划种植水稻、大豆和玉米,要求三种作物的最低收获量分别为\(190000kg\)、\(130000kg\)和\(350000kg\)。I,II,III 等耕地种植三种作物的单产如下表所示。
若三种作物的售价分别为水稻1.20元/kg,大豆1.50元/kg,玉米0.80元/kg。那么,
\begin{itemize}
\item {} 
如何制订种植计划才能使总产量最大?

\item {} 
如何制订种植计划才能使总产值最大?
\sphinxstylestrong{要求:写出规划问题的标准型,并合理采用本课程学到的知识,进行求解。}

\end{itemize}

\end{itemize}


\begin{savenotes}\sphinxattablestart
\centering
\begin{tabulary}{\linewidth}[t]{|T|T|T|T|}
\hline


&\sphinxstyletheadfamily 
I等耕地
&\sphinxstyletheadfamily 
II等耕地
&\sphinxstyletheadfamily 
III等耕地
\\
\hline
水稻
&
11000
&
9500
&
9000
\\
\hline
大豆
&
8000
&
6800
&
6000
\\
\hline
玉米
&
14000
&
12000
&
10000
\\
\hline
\end{tabulary}
\par
\sphinxattableend\end{savenotes}
\end{sphinxadmonition}


\chapter{评价模型}
\label{\detokenize{docs/evaluation_model:id1}}\label{\detokenize{docs/evaluation_model::doc}}

\section{评价模型介绍}
\label{\detokenize{docs/evaluation_model:id2}}
我们常常会遇到如下的综合评价问题:\sphinxstylestrong{在若干个(同类)对象中,从多个维度对其进行评分,将这些评分综合后,给定一个最终排名}。如好大学排名,运动员排名,城市排名等。

\begin{figure}[htbp]
\centering

\noindent\sphinxincludegraphics[height=300\sphinxpxdimen]{{0}.jpg}
\end{figure}

这时候我们就需要用到评价模型,评价模型是数学建模比赛中最基础也是最常用的模型, 例如\sphinxhref{https://www.comap.com/highschool/contests/himcm/2018problems.html}{2018年HiMCM A题} 就专门考察了评价模型。

\begin{sphinxadmonition}{note}{ 2018HiMCM\sphinxhyphen{}A}

There are several Roller Coaster rating/ranking sites online that, while taking some objective measures into account, heavily rely on subjective input to determine the rating or ranking of a particular roller coaster (e.g., an “excitement”or “experience” score of an “expert” rider to measure “thrill”).

In addressing this HiMCM problem, consider only roller coasters currently in operation. We have provided data for a subset of operating roller coasters whose height, speed, and/or drop are above the average of worldwide operating coasters. Therefore, we have not included family or kiddie coasters, nor have we included bobsled or mountain type coasters.
\begin{enumerate}
\sphinxsetlistlabels{\arabic}{enumi}{enumii}{}{.}%
\item {} 
Create an objective quantitative algorithm or set of algorithms to develop a descriptive roller coaster rating/ranking system based only on roller coaster numerical and descriptive specification data (e.g., speed, duration of ride, steel or wood, drop).

\item {} 
Use your algorithm(s) to develop your “Top 10 Roller Coasters in the World” list. Compare and discuss the rating/ranking results and descriptions from your team’s algorithm(s) with at least two other rating/ranking systems found online.

\item {} 
Describe the concept and design for a user\sphinxhyphen{}friendly app that uses your algorithm(s) to help a potential roller coaster rider find a roller coaster that she or he would want to ride. NOTE: You DO NOT need to program and/or write code for the app. You are developing the concept and design for the app only.

\item {} 
Write a one\sphinxhyphen{}page non\sphinxhyphen{}technical News Release describing your new algorithm, results, and app.

\end{enumerate}

\begin{sphinxadmonition}{note}{ 数据下载地址}

\sphinxhref{https://www.comap.com/highschool/contests/himcm/COMAP\_RollerCoasterData\_2018.xlsx}{点我下载}
\end{sphinxadmonition}
\end{sphinxadmonition}

如何去思考综合评价问题的建模呢?通常有如下五个角度
\begin{itemize}
\item {} 
\sphinxstylestrong{评价对象}:评价对象就是综合评价问题中所研究的对象,或称为系统。通常情况下,在一个问题中评价对象是属于同一类的,且个数要大于1,不妨假设一个综合评价问题中有\(n\)个评价对象,分别记为

\end{itemize}
\begin{equation*}
\begin{split}
S_{1}, S_{2}, \cdots, S_{n}(n>1)
\end{split}
\end{equation*}\begin{itemize}
\item {} 
\sphinxstylestrong{评价指标}:评价指标是反映评价对象的运行(或发展)状况的基本要素。通常的问题都是有多项指标构成,每一项指标都是从不同的侧面刻画系统所具有某种特征大小的一个度量。一个综合评价问题的评价指标一般可用一个向量\(x\)表示,称为评价指标问题,其中每一个分量就是从一个侧面反映系统的状态,即称为综合评价的指标体系。不失一般性,设系统有\(m\)个评价指标,分别记为

\end{itemize}
\begin{equation*}
\begin{split}
x_{1}, x_{2}, \cdots, x_{m}(m>1)
\end{split}
\end{equation*}\begin{itemize}
\item {} 
\sphinxstylestrong{权重系数}: 每一个综合评价问题都有相应的评价目的,针对某种评价目的,各评价指标之间的相对重要性是不同的,评价指标之间的这种相对重要性的大小,可用权重系数来刻画。当各评价对象和评价指标值都确定以后,综合评价结果就依赖于权重系数的取值了,即\sphinxstylestrong{权重系数确定的合理与否,直接关系到综合评价结果的可信度,甚至影响到最后决策的正确性}。因此,权重系数的确定要特别谨慎,应按一定的方法和原则来确定。如果用\(w_{j}(j=1,2, \cdots, m)\)来表示评价指标\(x_j\)的权重系数,一般应满足

\end{itemize}
\begin{equation*}
\begin{split}
w_{j} \geq 0, j=1,2, \cdots, m
\end{split}
\end{equation*}\begin{equation*}
\begin{split}
\sum_{j=1}^{m} w_{j}=1
\end{split}
\end{equation*}\begin{itemize}
\item {} 
\sphinxstylestrong{综合模型} 对于多指标(或多因素)的综合评价问题,就是要通过建立一定的数学模型将多个评价指标值综合成为一个整体的综合评价值,作为综合评价的依据,从而得到相应的评价结果。

\item {} 
\sphinxstylestrong{评价者} 评价者是直接参与评价的人,可以是一个人,也可以是一个团体。对于评价目的选择、评价指标体系确定、权重系数的确定和评价模型的建立都与评价者有关。因此,评价者在评价过程中的作用是不可小视的。

\end{itemize}

目前国内外综合评价方法有数十种之多,其中主要使用的评价方法有
\begin{itemize}
\item {} 
主成分分析法

\item {} 
因子分析

\item {} 
\sphinxstylestrong{TOPSIS}

\item {} 
秩和比法

\item {} 
灰色关联法

\item {} 
\sphinxstylestrong{熵权法}

\item {} 
\sphinxstylestrong{层次分析法}

\item {} 
模糊评价法

\item {} 
物元分析法

\item {} 
聚类分析法

\item {} 
价值工程法

\item {} 
神经网络法等

\end{itemize}

方法多样,各自有其适用场景。本次课程,我们重点展开讲其中较为常用的\sphinxstylestrong{TOPSIS、熵权法和层次分析法}。

在展开这三个模型之前,我们首先来看一下数据的预处理方法。


\section{数据预处理方法}
\label{\detokenize{docs/evaluation_model:id3}}
一般情况下,在综合评价指标中,有的指标比较重要,有的影响微乎其微,另外各指标值可能属于不同类型、不同单位或不同数量级,从而使得各指标之间存在着不可公度性,给综合评价带来了诸多不便。为了尽可能地反映实际情况,消除由于各项指标间的这些差别带来的影响,避免出现不合理的评价结果,就需要对评价指标进行一定的预处理,包括
\begin{enumerate}
\sphinxsetlistlabels{\arabic}{enumi}{enumii}{}{.}%
\item {} 
{\hyperref[\detokenize{docs/evaluation_model:content-choose-1}]{\sphinxcrossref{\DUrole{std,std-ref}{\sphinxstylestrong{指标的筛选}}}}}

\item {} 
{\hyperref[\detokenize{docs/evaluation_model:content-choose-2}]{\sphinxcrossref{\DUrole{std,std-ref}{\sphinxstylestrong{指标的一致化处理}}}}}

\item {} 
{\hyperref[\detokenize{docs/evaluation_model:content-choose-3}]{\sphinxcrossref{\DUrole{std,std-ref}{\sphinxstylestrong{无量纲化处理}}}}}

\item {} 
{\hyperref[\detokenize{docs/evaluation_model:content-choose-4}]{\sphinxcrossref{\DUrole{std,std-ref}{\sphinxstylestrong{定性数据定量化}}}}}

\end{enumerate}

下面分别介绍。


\subsection{评价指标的筛选}
\label{\detokenize{docs/evaluation_model:content-choose-1}}\label{\detokenize{docs/evaluation_model:id4}}
要根据综合评价的目的,针对具体的评价对象、评价内容收集有关指标信息,采用适当的筛选方法对指标进行筛选,合理地选取主要指标,剔除次要指标,以简化评价指标体系。常用的评价指标筛选方法主要有专家调研法、\sphinxstylestrong{最小均方差法、极大极小离差法}等。我们重点来看后两种方法。


\subsubsection{最小均方差法}
\label{\detokenize{docs/evaluation_model:id5}}
对于\(n\)个评价对象\(S_{1}, S_{2}, \cdots, S_{n}\),每个评价对象有\(m\)个指标,其观测值分别为
\begin{equation*}
\begin{split}
a_{i j}(i=1,2, \cdots, n ; j=1,2, \cdots, m)
\end{split}
\end{equation*}
\begin{sphinxadmonition}{hint}{Hint:}
最小均方差法的出发点是: 如果\(n\)个评价对象关于某项指标的观测值都差不多,那么不管这个评价指标重要与否,对于这\(n\)个评价对象的评价结果所起的作用将是很小的。因此,在评价过程中就可以删除这样的评价指标。
\end{sphinxadmonition}

最小均方差法的筛选过程如下:
\begin{itemize}
\item {} 
首先求出第\(j\)项指标的平均值和均方差

\end{itemize}
\begin{equation*}
\begin{split}
\mu_{j}=\frac{1}{n} \sum_{i=1}^{n} a_{i j}
\end{split}
\end{equation*}\begin{equation*}
\begin{split}
s_{j}=\sqrt{\frac{1}{n} \sum_{i=1}^{n}\left(a_{i j}-\mu_{j}\right)^{2}}, \quad j=1,2, \cdots, m
\end{split}
\end{equation*}\begin{itemize}
\item {} 
求出最小均方差

\end{itemize}
\begin{equation*}
\begin{split}
{S}_{j_{0}}=\min _{1 \leq j \leq m}\left\{{s}_{j}\right\}
\end{split}
\end{equation*}\begin{itemize}
\item {} 
如果最小均方差\(S_{j_{0}} \approx 0\),则可删除与\(S_{j_{0}}\)对应的指标 。考察完所有指标,即可得到最终的评价指标体系。

\end{itemize}

\begin{sphinxadmonition}{warning}{Warning:}
注意:最小均方差法只考虑了指标的差异程度,也有可能将重要的指标删除。你能否举一个例子?
\end{sphinxadmonition}


\subsubsection{极大极小离差法}
\label{\detokenize{docs/evaluation_model:id6}}
对于\(n\)个评价对象\(S_{1}, S_{2}, \cdots, S_{n}\),每个评价对象有\(m\)个指标,其观测值分别为
\begin{equation*}
\begin{split}
a_{i j}(i=1,2, \cdots, n ; j=1,2, \cdots, m)
\end{split}
\end{equation*}
极大极小离差法的筛选过程如下:
\begin{itemize}
\item {} 
求出第\(j\)项指标的最大离差

\end{itemize}
\begin{equation*}
\begin{split}
d_{j}=\max _{1 \leq i,k \leq n}\left\{\left|a_{i j}-a_{k j}\right|\right\}, j=1,2, \cdots, m
\end{split}
\end{equation*}\begin{itemize}
\item {} 
求出最小离差

\end{itemize}
\begin{equation*}
\begin{split}
d_{j_{0}}=\min _{1 \leq j \leq n}\left\{d_{j}\right\}
\end{split}
\end{equation*}\begin{itemize}
\item {} 
如果最小离差\(d_{j_{0}} \approx 0\),则可删除与\(d_{j_{0}}\)对应的指标\(x_{j_{0}}\),考察完所有指标,即可得到最终的评价指标体系。

\end{itemize}

其他几个常用的评价指标筛选方法还有条件广义方差极小法、极大不相关法等,这里限于篇幅不再展开。


\subsection{指标的一致化处理}
\label{\detokenize{docs/evaluation_model:content-choose-2}}\label{\detokenize{docs/evaluation_model:id7}}
所谓一致化处理就是将评价指标的类型进行统一。

一般来说,在评价指标体系中,可能会同时存在极大型指标、极小型指标、居中型指标和区间型指标,它们都具有不同的特点。

若指标体系中存在不同类型的指标,必须在综合评价之前将评价指标的类型做一致化处理。

例如,将各类指标都转化为极大型指标,或极小型指标。一般的做法是将非极大型指标转化为极大型指标。


\subsubsection{极小型指标化为极大型指标}
\label{\detokenize{docs/evaluation_model:id8}}
对极小型指标\(x_j\),将其转化为极大型指标时,只需对指标\(x_j\)取倒数:
\begin{equation*}
\begin{split}
x_{j}^{\prime}=\frac{1}{x_{j}}
\end{split}
\end{equation*}
或做平移变换:
\begin{equation*}
\begin{split}
x_{j}^{\prime}=M_{j}-x_{j}
\end{split}
\end{equation*}
其中,\(M_{j}=\max _{1 \leq i \leq n}\left\{a_{i j}\right\}\),即\(n\)个评价对象第\(j\)项指标值\(a_{ij}\)最大者。

当然,其他能改变单调性的转换方法也是可行的。


\subsubsection{居中型指标化为极大型指标}
\label{\detokenize{docs/evaluation_model:id9}}
对居中型指标\(x_j\),令\(M_{j}=\max _{1 \leq i \leq n}\left\{a_{i j}\right\}, \quad m_{j}=\min _{1 \leq i \leq n}\left\{a_{i j}\right\}\),取
\begin{equation*}
\begin{split}
x_{j}^{\prime}=\left\{\begin{array}{ll}
{\frac{2\left(x_{j}-m_{j}\right)}{M_{j}-m_{j}},} & {m_{j} \leq x_{j} \leq \frac{M_{j}+m_{j}}{2}} \\
{\frac{2\left(M_{j}-x_{j}\right)}{M_{j}-m_{j}},} & {\frac{M_{j}+m_{j}}{2} \leq x_{j} \leq M_{j}}
\end{array}\right.
\end{split}
\end{equation*}
就可以将\(x_j\)转化为极大型指标。


\subsubsection{区间型指标化为极大型指标}
\label{\detokenize{docs/evaluation_model:id10}}
对区间型指标\(x_j\), \(x_j\)是取值介于区间\(\left[b_{j}^{(1)}, b_{j}^{(2)}\right]\),内时为最好,指标值离该区间越远就越差。令
\begin{equation*}
\begin{split}
M_{j}=\max _{1 \leq i \leq n}\left\{a_{i j}\right\}, \quad m_{j}=\min _{\| \leq i \leq n}\left\{a_{i j}\right\}, \quad c_{j}=\max \left\{b_{j}^{(1)}-m_{j}, M_{j}-b_{j}^{(2)}\right\}
\end{split}
\end{equation*}
就可以将区间型指标\(x_j\)转化为极大型指标。
\begin{equation*}
\begin{split}
x_{j}'=\left\{\begin{array}{lll}
{1-\dfrac{b_{j}^{(1)}-x_{j}}{c_{j}},} & {x_{j} < b_{j}^{(1)}} \\
{1,} & {b_{j}^{(1)} \leq x_{j} \leq b_{j}^{(2)}} \\
{1-\dfrac{x_{j}-b_{j}^{(2)}}{c_{j}},} & {x_{j} > b_{j}^{(2)}}
\end{array}\right.
\end{split}
\end{equation*}

\subsection{指标的无量纲化处理}
\label{\detokenize{docs/evaluation_model:content-choose-3}}\label{\detokenize{docs/evaluation_model:id11}}
\sphinxstylestrong{所谓无量纲化,也称为指标的规范化,是通过数学变换来消除原始指标的单位及其数值数量级影响的过程。}

因此,就有指标的实际值和评价值之分。一般地,将指标无量纲化处理以后的值称为指标评价值。

无量纲化过程就是将指标实际值转化为指标评价值的过程。

对于\(n\)个评价对象\(S_{1}, S_{2}, \cdots, S_{n}\),每个评价对象有\(m\)个指标,其观测值分别为
\begin{equation*}
\begin{split}
a_{i j}(i=1,2, \cdots, n ; j=1,2, \cdots, m)
\end{split}
\end{equation*}

\subsubsection{标准样本变换法}
\label{\detokenize{docs/evaluation_model:id12}}
令
\begin{equation*}
\begin{split}
a_{i j}^{*}=\frac{a_{i j}-\mu_{j}}{s_{j}}(1 \leq i \leq n, 1 \leq j \leq m)
\end{split}
\end{equation*}
其中样本均值\(\mu_{j}=\frac{1}{n} \sum_{i=1}^{n} a_{i j}\),样本均方差\(s_{j}=\sqrt{\frac{1}{n} \sum_{i=1}^{n}\left(a_{i j}-\mu_{j}\right)^{2}}\),称为标准观测值。


\subsubsection{比例变换法}
\label{\detokenize{docs/evaluation_model:id13}}
对于极大型指标,令
\begin{equation*}
\begin{split}
a_{i j}^{*}=\frac{a_{i j}}{\max _{1 \leq i \leq n} a_{i j}}\left(\max _{1 \leq i \leq n} a_{i j} \neq 0,1 \leq i \leq n, 1 \leq j \leq m\right)
\end{split}
\end{equation*}
对极小型指标,令
\begin{equation*}
\begin{split}
a_{i j}^{*}=\frac{\min a_{i j}}{a_{i j}}(1 \leq i \leq n, 1 \leq j \leq m)
\end{split}
\end{equation*}
或
\begin{equation*}
\begin{split}
a_{i j}^{*}=1-\frac{a_{i j}}{\max _{1 \leq i \leq n} a_{i j}}\left(\max _{1 \leq i \leq n} a_{i j} \neq 0,1 \leq i \leq n, 1 \leq j \leq m\right)
\end{split}
\end{equation*}
该方法的优点是这些变换前后的属性值成比例。但对任一指标来说,变换后的\(a_{i j}^{*} = 1\)和\(a_{i j}^{*}= 0\)不一定同时出现。


\subsubsection{向量归一化法}
\label{\detokenize{docs/evaluation_model:id14}}
对于极大型指标,令
\begin{equation*}
\begin{split}
a_{i j}^{*}=\frac{a_{i j}}{\sqrt{\sum_{i=1}^{n} a_{i j}^{2}}}(i=1,2, \cdots, n, 1 \leq j \leq m)
\end{split}
\end{equation*}
对于极小型指标,令
\begin{equation*}
\begin{split}
a_{i j}^{*}=1-\frac{a_{i j}}{\sqrt{\sum_{i=1}^{n} a_{i j}^{2}}}(i=1,2, \cdots, n, 1 \leq j \leq m)
\end{split}
\end{equation*}

\subsubsection{极差变换法}
\label{\detokenize{docs/evaluation_model:id15}}
对于极大型指标
\begin{equation*}
\begin{split}
a_{i j}^{*}=\frac{a_{i j}-\min _{1 \leq i \leq n} a_{i j}}{\max _{1 \leq i \leq n} a_{i j}-\min _{1 \leq i \leq n} a_{i j}}(1 \leq i \leq n, 1 \leq j \leq m)
\end{split}
\end{equation*}
对于极小型直指标
\begin{equation*}
\begin{split}
a_{i j}^{*}=\frac{\max _{1 \leq i \leq n} a_{i j}-a_{i j}}{\max _{1 \leq i \leq n} a_{i j}-\min _{1 \leq i \leq n} a_{i j}}(1 \leq i \leq n, 1 \leq j \leq m)
\end{split}
\end{equation*}
其特点为经过极差变换后,均有\(0 \leq a_{i j}^{*} \leq 1\),且最优指标值\(a_{i j}^{*}=1\),最劣指标值\(a_{i j}^{*}=0\)。该方法的缺点是变换前后的各指标值不成比例。


\subsubsection{功效系数法}
\label{\detokenize{docs/evaluation_model:id16}}
令,
\begin{equation*}
\begin{split}
a_{i j}^{*}=c+\frac{a_{i j}-\min _{1 \leq i \leq n} a_{i j}}{\max _{1 \leq i \leq n} a_{i j}-\min _{1 \leq i \leq n} a_{i j}} \times d(1 \leq i \leq n, 1 \leq j \leq m)
\end{split}
\end{equation*}
其\(c,d\)均为确定的常数,\(c\)表示“平移量”,表示指标实际基础值, \(d\)表示“旋转量”,即表示“放大”或“缩小”倍数


\subsection{定性指标的定量化}
\label{\detokenize{docs/evaluation_model:content-choose-4}}\label{\detokenize{docs/evaluation_model:id17}}
在综合评价工作中,有些评价指标是定性指标,即只给出定性的描述,例如,质量很好、性能一般、可靠性高等。对于这些指标,在进行综合评价时,必须先通过适当的方式进行赋值,使其量化。一般来说,对于指标最优值可赋值1,对于指标最劣值可赋值0。对极大型定性指标常按以下方式赋值。

对于极大型定性指标而言,如果指标能够分为很低、低、一般、高和很高五个等级,则可以分别取量化值为0,0.1,0.3,0.5,0.7,1,对应关系如下表所示。介于两个等级之间的可以取两个分值之间的适当数值作为量化值。极小型指标同理。


\begin{savenotes}\sphinxattablestart
\centering
\begin{tabulary}{\linewidth}[t]{|T|T|T|T|T|T|}
\hline
\sphinxstyletheadfamily 
等级
&\sphinxstyletheadfamily 
很低
&\sphinxstyletheadfamily 
低
&\sphinxstyletheadfamily 
一般
&\sphinxstyletheadfamily 
高
&\sphinxstyletheadfamily 
很高
\\
\hline
量化值
&
0
&
0.3
&
0.5
&
0.7
&
0.9
\\
\hline
\end{tabulary}
\par
\sphinxattableend\end{savenotes}

下面我们通过几个例子来学习三种常用的评价模型。分别是TOPSIS方法,熵权法和层次分析法。


\section{TOPSIS方法}
\label{\detokenize{docs/evaluation_model:topsis}}
C.L.Hwang 和 K.Yoon 于1981年首次提出 TOPSIS (全称:Technique for Order Preference by Similarity to an Ideal Solution)。

TOPSIS 法是一种常用的组内综合评价方法,能充分利用原始数据的信息,其结果能精确地反映各评价方案之间的差距。

基本过程为基于归一化后的原始数据矩阵,采用余弦法找出有限方案中的\sphinxstylestrong{最优方案和最劣方案},然后分别计算各评价对象与最优方案和最劣方案间的距离,获得各评价对象与最优方案的相对接近程度,以此作为评价优劣的依据。

该方法对数据分布及样本含量没有严格限制,数据计算简单易行。

为了客观地评价我国研究生教育的实际状况和各研究生院的教学质量,国务院学位委员会办公室组织过一次研究生院的评估。为了取得经验,先选5所研究生院,收集有关数据资料进行了试评估,下表是所给出的部分数据:

\sphinxincludegraphics{{5001}.png}

TOPSIS 法使用距离尺度来度量样本差距,使用距离尺度就需要对指标属性进行同向化处理(若一个维度的数据越大越好,另一个维度的数据越小越好,会造成尺度混乱)。通常采用成本型指标向效益型指标转化(即数值越大评价越高,事实上几乎所有的评价方法都需要进行转化)

通过分析,我们知道:
\begin{itemize}
\item {} 
人均专著,越多越好(极大型指标)

\item {} 
科研经费,越多越好(极大型指标)

\item {} 
逾期毕业率,越小越好(极小型指标)

\item {} 
生师比,过大过小都不好(区间型指标)

\end{itemize}

设研究生院的生师比最佳区间为\([5,6]\),在最佳区间内生师比得分为1 ,如果生师比小于2或者大于12都是0分,在其他的区间都按照线性关系进行变换。

因此,我们把两个极大型指标保持不变,对极小型指标采用取倒数操作,对区间型指标使用上面介绍的处理方法。处理结果见下图。

\sphinxincludegraphics{{400}.png}

接下来,我们进行无量纲处理,以 “人均专著” 属性为例,我们使用向量归一化方法:
\begin{equation*}
\begin{split}
\begin{aligned}
&0.1 / \sqrt{0.1^{2}+0.2^{2}+0.4^{2}+0.9^{2}+1.2^{2}}=0.0637576713063384\\
&0.2 / \sqrt{0.1^{2}+0.2^{2}+0.4^{2}+0.9^{2}+1.2^{2}}=0.12751534261266767\\
&0.4 / \sqrt{0.1^{2}+0.2^{2}+0.4^{2}+0.9^{2}+1.2^{2}}=0.2550306852253334\\
&\begin{array}{l}
{0.9 / \sqrt{0.1^{2}+0.2^{2}+0.4^{2}+0.9^{2}+1.2^{2}}=0.5738190417570045} \\
{1.2 / \sqrt{0.1^{2}+0.2^{2}+0.4^{2}+0.9^{2}+1.2^{2}}=0.7650920556760059}
\end{array}
\end{aligned}
\end{split}
\end{equation*}
使用同样的向量归一化方法,我们可以对其他三个指标也进行无量纲化处理,得到如下表所示的结果

\sphinxincludegraphics{{4001}.png}

接着,我们选出其中的最优方案和最劣方案。

\sphinxincludegraphics{{4002}.png}

然后,我们开始计算每一个学校,与最优方案以及最劣方案之间的距离
\begin{equation*}
\begin{split}
\begin{array}{l}
{D_{i}^{+}=\sqrt{\sum_{j=1}^{m} w_{j}\left(Z_{j}^{+}-z_{i j}\right)^{2}}} \\
{D_{i}^{-}=\sqrt{\sum_{j=1}^{m} w_{j}\left(Z_{j}^{-}-z_{i j}\right)^{2}}}
\end{array}
\end{split}
\end{equation*}
然后使用如下的评价函数将其综合起来
\begin{equation*}
\begin{split}
C_{i}=\frac{D_{i}^{-}}{D_{i}^{+}+D_{i}^{-}}
\end{split}
\end{equation*}
\begin{sphinxadmonition}{note}{思考}

如果理想中最好的大学是真实存在的,其得分\(C_i\)应该等于几,为什么?如果是理想中最差的大学真实存在呢?
\end{sphinxadmonition}

最终评价结果如下

\sphinxincludegraphics{{640}.png}

当然,直接看结果可能不够直观,我们来通过一张雷达图解释这个评价的结果。

上面是通过举例来进行TOPSIS方法的实现,关于TOPSIS方法的更详细理论推导,参考\sphinxhref{https://wiki.mbalib.com/wiki/TOPSIS\%E6\%B3\%95}{这里}。

在实际论文撰写的过程中,最好理论性强一些,而不是像我们上面一样简单地代入数据计算。

\begin{sphinxadmonition}{note}{思考}

上面的TOPSIS方法的权重已经给出,请思考,如果实际建模的时候没有给出权重,应该如何选取?
\end{sphinxadmonition}


\section{熵权法}
\label{\detokenize{docs/evaluation_model:id18}}
下面我们来学习一种客观赋权的方法:熵权法,它是一种突出局部差异的客观赋权方法。因为它的权重选取仅依赖于数据本身的离散性。

我们通过一个例子来看。

某医院为了提高自身的护理水平,对拥有的11个科室进行了考核,考核标准包括9项整体护理,并对护理水平较好的科室进行奖励。下表是对各个科室指标考核后的评分结果。

\sphinxincludegraphics{{6401}.png}

但是由于各项护理的难易程度不同,因此需要对9项护理进行赋权,以便能够更加合理的对各个科室的护理水平进行评价。根据原始评分表,对数据进行标准化后可以得到下列数据标准化表

\sphinxincludegraphics{{6402}.png}

\begin{sphinxadmonition}{note}{思考}

这里我们用了什么归一化的方法?
\end{sphinxadmonition}

计算第\(j\)项指标下第\(i\)个样本值所占比重
\begin{equation*}
\begin{split}
p_{i j}=\frac{x_{i j}}{\displaystyle \sum_{i=1}^{n} x_{i j}}, \quad i=1, \cdots, n, j=1, \cdots, m
\end{split}
\end{equation*}
计算第\(j\)个指标的熵值(熵值的计算方法是信息论中的定义,这里我们直接采用)
\begin{equation*}
\begin{split}
e_{j}=-k \sum_{i=1}^{n} p_{i j} \ln \left(p_{i j}\right), \quad j=1, \cdots, m
\end{split}
\end{equation*}
其中,
\begin{equation*}
\begin{split}
k=1 / \ln (n)>0
\end{split}
\end{equation*}
\sphinxincludegraphics{{6403}.png}

可以发现,熵值越小的变量,离散程度越大。接下来,我们计算信息熵冗余度,并将其归一化得到权重
\begin{equation*}
\begin{split}
d_{j}=1-e_{j}, \quad j=1, \cdots, m
\end{split}
\end{equation*}\begin{equation*}
\begin{split}
w_{j}=\frac{d_{j}}{\displaystyle\sum_{j=1}^{m} d_{j}}, \quad j=1, \cdots, m
\end{split}
\end{equation*}
\sphinxincludegraphics{{6404}.png}

加权求和计算指标综合评分
\begin{equation*}
\begin{split}
s_{i}=\sum_{j=1}^{m} w_{j} x_{i j}, \quad i=1, \cdots, n
\end{split}
\end{equation*}
\sphinxincludegraphics{{6405}.png}

上面是一个特殊的案例,接下来我们来总结熵权法的一般步骤。

\sphinxstylestrong{熵权法步骤}
\begin{itemize}
\item {} 
对\(n\)个样本,\(m\)个指标的数据集

\end{itemize}
\begin{equation*}
\begin{split}
\{x_{ij}|i = 1,2,\cdots,n, \quad j = 1,2,\cdots,m\}
\end{split}
\end{equation*}\begin{itemize}
\item {} 
对指标进行归一化处理:异质指标同质化

\item {} 
计算第\(j\)项指标下第\(i\)个样本值所占比重

\end{itemize}
\begin{equation*}
\begin{split}
p_{i j}=\frac{x_{i j}}{\displaystyle\sum_{i=1}^{n} x_{i j}}, \quad i=1, \cdots, n, j=1, \cdots, m
\end{split}
\end{equation*}\begin{itemize}
\item {} 
计算第\(j\)个指标的熵值

\end{itemize}
\begin{equation*}
\begin{split}
e_{j}=-k \sum_{i=1}^{n} p_{i j} \ln \left(p_{i j}\right), \quad j=1, \cdots, m
\end{split}
\end{equation*}
其中,
\begin{equation*}
\begin{split}
k=1 / \ln (n)>0
\end{split}
\end{equation*}\begin{itemize}
\item {} 
计算信息熵冗余度,并将其归一化得到权重

\end{itemize}
\begin{equation*}
\begin{split}
d_{j}=1-e_{j}, \quad j=1, \cdots, m
\end{split}
\end{equation*}\begin{equation*}
\begin{split}
w_{j}=\frac{d_{j}}{\displaystyle\sum_{j=1}^{m} d_{j}}, \quad j=1, \cdots, m
\end{split}
\end{equation*}\begin{itemize}
\item {} 
计算指标综合评分

\end{itemize}
\begin{equation*}
\begin{split}
s_{i}=\sum_{j=1}^{m} w_{j} x_{i j}, \quad i=1, \cdots, n
\end{split}
\end{equation*}
这里的\(x_{ij}\)是标准化以后的数据。

\begin{sphinxadmonition}{note}{思考}

熵权法是客观赋权,这里的\sphinxstylestrong{客观}就一定是优于主观的吗?经过熵权法计算得到的权重,应用中会不会有什么问题?
\end{sphinxadmonition}


\section{层次分析法}
\label{\detokenize{docs/evaluation_model:id19}}
AHP (Analytic Hierarchy Process)层次分析法是美国运筹学家Saaty教授于二十世纪80年代提出的一种实用的多方案或多目标的决策方法。其主要特征是,它合理地将定性与定量的决策结合起来,按照思维、心理的规律把决策过程层次化、数量化。 该方法自1982年被介绍到我国以来,以其定性与定量相结合地处理各种决策因素的特点,以及其系统灵活简洁的优点,迅速地在我国社会经济各个领域内,如能源系统分析、城市规划、经济管理、科研评价等,得到了广泛的重视和应用。

层次分析法的基本思路:先分解后综合首先将所要分析的问题层次化,根据问题的性质和要达到的总目标,将问题分解成不同的组成因素,按照因素间的相互关系及隶属关系,将因素按不同层次聚集组合,形成一个多层分析结构模型,最终归结为最低层(方案、措施、指标等)相对于最高层(总目标)相对重要程度的权值或相对优劣次序的问题。 运用层次分析法建模,大体上可按下面四个步骤进行:
\begin{itemize}
\item {} 
建立递阶层次结构模型;

\item {} 
构造出各层次中的所有判断矩阵;

\item {} 
层次单排序及一致性检验;

\item {} 
层次总排序及一致性检验。

\end{itemize}

我们一样通过一个案例来学习层次分析法。

人们在日常生活中经常会碰到多目标决策问题,例如假期某人想要出去旅游,现有三个目的地(方案):风光绮丽的杭州(P1)、迷人的北戴河(P2)和山水甲天下的桂林(P3)。假如选择的标准和依据(行动方案准则)有5个:景色,费用,饮食,居住和旅途。则常规思维的方式如下:

\sphinxincludegraphics{{5002}.png}

通过相互比较确定各准则对于目标的权重,即构造判断矩阵。在层次分析法中,为使矩阵中的各要素的重要性能够进行定量显示,引进了矩阵判断标度(1~9标度法) :

\sphinxincludegraphics{{700}.png}

构造判断矩阵

\sphinxincludegraphics{{960}.png}


\subsection{层次单排序和总排序}
\label{\detokenize{docs/evaluation_model:id20}}
所谓层次单排序是指,对于上层某因素而言,本层次因素重要性的排序。
具体计算方法为:对于判断矩阵\(B\),计算满足
\begin{equation*}
\begin{split}
BW=\lambda_{max} W
\end{split}
\end{equation*}
的特征根与特征向量。

式中,\(\lambda_{max}\) 为矩阵\(B\)的最大特征跟根,\(W\)为对应于\(\lambda_{max}\)的特征向量,\(W\)的分量\(w_i\)即为相应元素单排序的权值。

\begin{sphinxVerbatim}[commandchars=\\\{\}]
\PYG{n}{A} \PYG{o}{=} \PYG{p}{[}\PYG{p}{[}\PYG{l+m+mi}{1}\PYG{p}{,}\PYG{l+m+mi}{2}\PYG{p}{,}\PYG{l+m+mi}{9}\PYG{p}{]}\PYG{p}{,}\PYG{p}{[}\PYG{l+m+mi}{1}\PYG{o}{/}\PYG{l+m+mi}{2}\PYG{p}{,}\PYG{l+m+mi}{1}\PYG{p}{,}\PYG{l+m+mi}{1}\PYG{o}{/}\PYG{l+m+mi}{2}\PYG{p}{]}\PYG{p}{,}\PYG{p}{[}\PYG{l+m+mi}{1}\PYG{o}{/}\PYG{l+m+mi}{9}\PYG{p}{,}\PYG{l+m+mi}{2}\PYG{p}{,}\PYG{l+m+mi}{1}\PYG{p}{]}\PYG{p}{]}
\PYG{n}{A}
\end{sphinxVerbatim}

\begin{sphinxVerbatim}[commandchars=\\\{\}]
[[1, 2, 9], [0.5, 1, 0.5], [0.1111111111111111, 2, 1]]
\end{sphinxVerbatim}

\begin{sphinxVerbatim}[commandchars=\\\{\}]
\PYG{c+c1}{\PYGZsh{} 调用 np.linalg.eig方法计算矩阵的特征值和特征向量,其中lamb是特征值,v是特征向量}
\PYG{k+kn}{import} \PYG{n+nn}{numpy} \PYG{k}{as} \PYG{n+nn}{np}
\PYG{n}{lamb}\PYG{p}{,}\PYG{n}{v} \PYG{o}{=} \PYG{n}{np}\PYG{o}{.}\PYG{n}{linalg}\PYG{o}{.}\PYG{n}{eig}\PYG{p}{(}\PYG{n}{A}\PYG{p}{)}      
\PYG{n+nb}{print}\PYG{p}{(}\PYG{n}{lamb}\PYG{p}{)}
\end{sphinxVerbatim}

\begin{sphinxVerbatim}[commandchars=\\\{\}]
[ 3.56083368+0.j         \PYGZhy{}0.28041684+1.38506384j \PYGZhy{}0.28041684\PYGZhy{}1.38506384j]
\end{sphinxVerbatim}

\begin{sphinxVerbatim}[commandchars=\\\{\}]
\PYG{n}{lambda\PYGZus{}max} \PYG{o}{=} \PYG{n+nb}{max}\PYG{p}{(}\PYG{n+nb}{abs}\PYG{p}{(}\PYG{n}{lamb}\PYG{p}{)}\PYG{p}{)}                    \PYG{c+c1}{\PYGZsh{} 提取最大的特征值}
\PYG{n}{loc} \PYG{o}{=} \PYG{n}{np}\PYG{o}{.}\PYG{n}{where}\PYG{p}{(}\PYG{n}{lamb} \PYG{o}{==} \PYG{n}{lambda\PYGZus{}max}\PYG{p}{)}             \PYG{c+c1}{\PYGZsh{} 获取最大特征值的索引}
\end{sphinxVerbatim}

\begin{sphinxVerbatim}[commandchars=\\\{\}]
\PYG{n}{weight} \PYG{o}{=} \PYG{n+nb}{abs}\PYG{p}{(}\PYG{n}{v}\PYG{p}{[}\PYG{l+m+mi}{0}\PYG{p}{:}\PYG{n+nb}{len}\PYG{p}{(}\PYG{n}{A}\PYG{p}{)}\PYG{p}{,}\PYG{n}{loc}\PYG{p}{[}\PYG{l+m+mi}{0}\PYG{p}{]}\PYG{p}{[}\PYG{l+m+mi}{0}\PYG{p}{]}\PYG{p}{]}\PYG{p}{)}            \PYG{c+c1}{\PYGZsh{} 获取最大特征值对应的特征向量}
\PYG{n}{weight}
\end{sphinxVerbatim}

\begin{sphinxVerbatim}[commandchars=\\\{\}]
array([0.94864674, 0.22803089, 0.21925164])
\end{sphinxVerbatim}

\sphinxincludegraphics{{9601}.png}
\begin{equation*}
\begin{split}
W=W^{(3)} W^{(2)}=\left(\begin{array}{ccccc}
{0.595} & {0.082} & {0.429} & {0.633} & {0.166} \\
{0.277} & {0.236} & {0.429} & {0.193} & {0.166} \\
{0.129} & {0.682} & {0.142} & {0.175} & {0.668}
\end{array}\right)\left(\begin{array}{c}
{0.263} \\
{0.475} \\
{0.055} \\
{0.099} \\
{0.110}
\end{array}\right)=\left(\begin{array}{c}
{0.300} \\
{0.246} \\
{0.456}
\end{array}\right)
\end{split}
\end{equation*}
决策结果是首选旅游地为\(P_3\) ,其次为\(P_1\),再次\(P_2\)

一般地,若层次结构由\(k\)个层次(目标层算第一层),则方案的优先程度的排序向量为:
\begin{equation*}
\begin{split}
W=W^{(k)} W^{(k-1)} \dots W^{(2)}
\end{split}
\end{equation*}

\subsection{判断一致性}
\label{\detokenize{docs/evaluation_model:id21}}
判断矩阵通常是不一致的,但是为了能用它的对应于特征根的特征向量作为被比较因素的权向量,其不一致程度应在容许的范围内.如何确定这个范围?

一致性指标:
\begin{equation*}
\begin{split}
C I=\frac{\lambda-n}{n-1}
\end{split}
\end{equation*}\begin{equation*}
\begin{split}
C R=\frac{C I}{R I}
\end{split}
\end{equation*}
\sphinxincludegraphics{{9602}.png}

当\(CR<0.1\)时,认为层次排序是具有满意的一致性的,我们可以接受该分析结果。

\begin{sphinxVerbatim}[commandchars=\\\{\}]
\PYG{n}{A} \PYG{o}{=} \PYG{p}{[}\PYG{p}{[}\PYG{l+m+mi}{1}\PYG{p}{,}\PYG{l+m+mi}{2}\PYG{p}{,}\PYG{l+m+mi}{6}\PYG{p}{]}\PYG{p}{,}\PYG{p}{[}\PYG{l+m+mi}{1}\PYG{o}{/}\PYG{l+m+mi}{2}\PYG{p}{,}\PYG{l+m+mi}{1}\PYG{p}{,}\PYG{l+m+mi}{2}\PYG{p}{]}\PYG{p}{,}\PYG{p}{[}\PYG{l+m+mi}{1}\PYG{o}{/}\PYG{l+m+mi}{5}\PYG{p}{,}\PYG{l+m+mi}{1}\PYG{o}{/}\PYG{l+m+mi}{2}\PYG{p}{,}\PYG{l+m+mi}{1}\PYG{p}{]}\PYG{p}{]}
\PYG{c+c1}{\PYGZsh{} 调用 np.linalg.eig方法计算矩阵的特征值和特征向量,其中lamb是特征值,v是特征向量}
\PYG{n}{lamb}\PYG{p}{,}\PYG{n}{v} \PYG{o}{=} \PYG{n}{np}\PYG{o}{.}\PYG{n}{linalg}\PYG{o}{.}\PYG{n}{eig}\PYG{p}{(}\PYG{n}{A}\PYG{p}{)}      
\PYG{n}{lambda\PYGZus{}max} \PYG{o}{=} \PYG{n+nb}{max}\PYG{p}{(}\PYG{n+nb}{abs}\PYG{p}{(}\PYG{n}{lamb}\PYG{p}{)}\PYG{p}{)}                    \PYG{c+c1}{\PYGZsh{} 提取最大的特征值}
\PYG{n}{loc} \PYG{o}{=} \PYG{n}{np}\PYG{o}{.}\PYG{n}{where}\PYG{p}{(}\PYG{n}{lamb} \PYG{o}{==} \PYG{n}{lambda\PYGZus{}max}\PYG{p}{)}             \PYG{c+c1}{\PYGZsh{} 获取最大特征值的索引}
\PYG{n}{weight} \PYG{o}{=} \PYG{n+nb}{abs}\PYG{p}{(}\PYG{n}{v}\PYG{p}{[}\PYG{l+m+mi}{0}\PYG{p}{:}\PYG{n+nb}{len}\PYG{p}{(}\PYG{n}{A}\PYG{p}{)}\PYG{p}{,}\PYG{n}{loc}\PYG{p}{[}\PYG{l+m+mi}{0}\PYG{p}{]}\PYG{p}{[}\PYG{l+m+mi}{0}\PYG{p}{]}\PYG{p}{]}\PYG{p}{)}            \PYG{c+c1}{\PYGZsh{} 获取最大特征值对应的特征向量}
\PYG{n}{RI\PYGZus{}list} \PYG{o}{=} \PYG{p}{[}\PYG{l+m+mi}{0} \PYG{p}{,}\PYG{l+m+mi}{0} \PYG{p}{,}\PYG{l+m+mf}{0.58}\PYG{p}{,}\PYG{l+m+mf}{0.9}\PYG{p}{,}\PYG{l+m+mf}{1.12}\PYG{p}{,}\PYG{l+m+mf}{1.24}\PYG{p}{,}\PYG{l+m+mf}{1.32}\PYG{p}{,}\PYG{l+m+mf}{1.41}\PYG{p}{,}\PYG{l+m+mf}{1.45}\PYG{p}{]}   
\PYG{n}{RI} \PYG{o}{=} \PYG{n}{RI\PYGZus{}list}\PYG{p}{[}\PYG{n+nb}{len}\PYG{p}{(}\PYG{n}{A}\PYG{p}{)}\PYG{o}{\PYGZhy{}}\PYG{l+m+mi}{1}\PYG{p}{]}                        \PYG{c+c1}{\PYGZsh{} 计算RI}
\PYG{n}{CI} \PYG{o}{=} \PYG{p}{(}\PYG{n}{lambda\PYGZus{}max} \PYG{o}{\PYGZhy{}} \PYG{n+nb}{len}\PYG{p}{(}\PYG{n}{A}\PYG{p}{)}\PYG{p}{)}\PYG{o}{/}\PYG{p}{(}\PYG{n+nb}{len}\PYG{p}{(}\PYG{n}{A}\PYG{p}{)}\PYG{o}{\PYGZhy{}}\PYG{l+m+mi}{1}\PYG{p}{)}         \PYG{c+c1}{\PYGZsh{} 计算CI}
\PYG{n}{CR} \PYG{o}{=} \PYG{n}{CI} \PYG{o}{/} \PYG{n}{RI}                                   \PYG{c+c1}{\PYGZsh{} 计算CR}
\PYG{n+nb}{print}\PYG{p}{(}\PYG{l+s+s1}{\PYGZsq{}}\PYG{l+s+s1}{最大特征值 lambda\PYGZus{}max=}\PYG{l+s+s1}{\PYGZsq{}}\PYG{p}{,}\PYG{n}{lambda\PYGZus{}max}\PYG{p}{)}
\PYG{n+nb}{print}\PYG{p}{(}\PYG{l+s+s1}{\PYGZsq{}}\PYG{l+s+s1}{最大特征值对应的特征向量 w=}\PYG{l+s+s1}{\PYGZsq{}}\PYG{p}{,}\PYG{n}{weight}\PYG{p}{)}
\PYG{n+nb}{print}\PYG{p}{(}\PYG{l+s+s1}{\PYGZsq{}}\PYG{l+s+s1}{CI=}\PYG{l+s+s1}{\PYGZsq{}}\PYG{p}{,}\PYG{n}{CI}\PYG{p}{)}
\PYG{n+nb}{print}\PYG{p}{(}\PYG{l+s+s1}{\PYGZsq{}}\PYG{l+s+s1}{RI=}\PYG{l+s+s1}{\PYGZsq{}}\PYG{p}{,}\PYG{n}{RI}\PYG{p}{)}
\PYG{n+nb}{print}\PYG{p}{(}\PYG{l+s+s1}{\PYGZsq{}}\PYG{l+s+s1}{CR=}\PYG{l+s+s1}{\PYGZsq{}}\PYG{p}{,}\PYG{n}{CR}\PYG{p}{)}
\end{sphinxVerbatim}

\begin{sphinxVerbatim}[commandchars=\\\{\}]
最大特征值 lambda\PYGZus{}max= 3.075599563855794
最大特征值对应的特征向量 w= [0.90191687 0.3919986  0.18133684]
CI= 0.037799781927897014
RI= 0.58
CR= 0.065172037806719
\end{sphinxVerbatim}


\subsection{案例:企业的资金使用}
\label{\detokenize{docs/evaluation_model:id22}}
\sphinxincludegraphics{{9603}.png}

构造判断矩阵
\begin{equation*}
\begin{split}
A=\left(\begin{array}{ccc}{1} & {1 / 5} & {1 / 3} \\ {5} & {1} & {3} \\ {3} & {1 / 3} & {1}\end{array}\right)
\end{split}
\end{equation*}\begin{equation*}
\begin{split}
B_{1}=\left(\begin{array}{ccccc}{1} & {2} & {3} & {4} & {7} \\ {1 / 3} & {1} & {3} & {2} & {5} \\ {1 / 5} & {1 / 3} & {1} & {1 / 2} & {1} \\ {1 / 4} & {1 / 2} & {2} & {1} & {3} \\ {1 / 7} & {1 / 5} & {1 / 2} & {1 / 3} & {1}\end{array}\right) \quad B_{2}=\left(\begin{array}{cccc}{1} & {1 / 7} & {1 / 3} & {1 / 5} \\ {7} & {1} & {5} & {3} \\ {3} & {1 / 5} & {1} & {1 / 3} \\ {5} & {1 / 2} & {3} & {1}\end{array}\right) \quad B_{3}=\left(\begin{array}{cccc}{1} & {1} & {3} & {3} \\ {1} & {1} & {3} & {3} \\ {1 / 3} & {1 / 3} & {1} & {1} \\ {1 / 7} & {1 / 3} & {1} & {1}\end{array}\right)
\end{split}
\end{equation*}
\begin{sphinxVerbatim}[commandchars=\\\{\}]
\PYG{c+c1}{\PYGZsh{} 层次分析法函数}
\PYG{k+kn}{import} \PYG{n+nn}{numpy} \PYG{k}{as} \PYG{n+nn}{np}
\PYG{k}{def} \PYG{n+nf}{AHP}\PYG{p}{(}\PYG{n}{A}\PYG{p}{)}\PYG{p}{:}
    \PYG{n}{lamb}\PYG{p}{,}\PYG{n}{v} \PYG{o}{=} \PYG{n}{np}\PYG{o}{.}\PYG{n}{linalg}\PYG{o}{.}\PYG{n}{eig}\PYG{p}{(}\PYG{n}{A}\PYG{p}{)}                      \PYG{c+c1}{\PYGZsh{} 调用 np.linalg.eig方法计算矩阵的特征值和特征向量,其中lamb是特征值,v是特征向量}
    \PYG{n}{lambda\PYGZus{}max} \PYG{o}{=} \PYG{n+nb}{max}\PYG{p}{(}\PYG{n+nb}{abs}\PYG{p}{(}\PYG{n}{lamb}\PYG{p}{)}\PYG{p}{)}                    \PYG{c+c1}{\PYGZsh{} 提取最大的特征值}
    \PYG{n}{loc} \PYG{o}{=} \PYG{n}{np}\PYG{o}{.}\PYG{n}{where}\PYG{p}{(}\PYG{n}{lamb} \PYG{o}{==} \PYG{n}{lambda\PYGZus{}max}\PYG{p}{)}             \PYG{c+c1}{\PYGZsh{} 获取最大特征值的索引}
    \PYG{n}{weight} \PYG{o}{=} \PYG{n+nb}{abs}\PYG{p}{(}\PYG{n}{v}\PYG{p}{[}\PYG{l+m+mi}{0}\PYG{p}{:}\PYG{n+nb}{len}\PYG{p}{(}\PYG{n}{A}\PYG{p}{)}\PYG{p}{,}\PYG{n}{loc}\PYG{p}{[}\PYG{l+m+mi}{0}\PYG{p}{]}\PYG{p}{[}\PYG{l+m+mi}{0}\PYG{p}{]}\PYG{p}{]}\PYG{p}{)}            \PYG{c+c1}{\PYGZsh{} 获取最大特征值对应的特征向量}
    \PYG{n}{RI\PYGZus{}list} \PYG{o}{=} \PYG{p}{[}\PYG{l+m+mi}{0} \PYG{p}{,}\PYG{l+m+mi}{0} \PYG{p}{,}\PYG{l+m+mf}{0.58}\PYG{p}{,}\PYG{l+m+mf}{0.9}\PYG{p}{,}\PYG{l+m+mf}{1.12}\PYG{p}{,}\PYG{l+m+mf}{1.24}\PYG{p}{,}\PYG{l+m+mf}{1.32}\PYG{p}{,}\PYG{l+m+mf}{1.41}\PYG{p}{,}\PYG{l+m+mf}{1.45}\PYG{p}{]}   
    \PYG{n}{RI} \PYG{o}{=} \PYG{n}{RI\PYGZus{}list}\PYG{p}{[}\PYG{n+nb}{len}\PYG{p}{(}\PYG{n}{A}\PYG{p}{)}\PYG{o}{\PYGZhy{}}\PYG{l+m+mi}{1}\PYG{p}{]}                        \PYG{c+c1}{\PYGZsh{} 计算RI}
    \PYG{n}{CI} \PYG{o}{=} \PYG{p}{(}\PYG{n}{lambda\PYGZus{}max} \PYG{o}{\PYGZhy{}} \PYG{n+nb}{len}\PYG{p}{(}\PYG{n}{A}\PYG{p}{)}\PYG{p}{)}\PYG{o}{/}\PYG{p}{(}\PYG{n+nb}{len}\PYG{p}{(}\PYG{n}{A}\PYG{p}{)}\PYG{o}{\PYGZhy{}}\PYG{l+m+mi}{1}\PYG{p}{)}         \PYG{c+c1}{\PYGZsh{} 计算CI}
    \PYG{n}{CR} \PYG{o}{=} \PYG{n}{CI} \PYG{o}{/} \PYG{n}{RI}                                   \PYG{c+c1}{\PYGZsh{} 计算CR}
    \PYG{n+nb}{print}\PYG{p}{(}\PYG{l+s+s1}{\PYGZsq{}}\PYG{l+s+s1}{最大特征值 lambda\PYGZus{}max=}\PYG{l+s+s1}{\PYGZsq{}}\PYG{p}{,}\PYG{n}{lambda\PYGZus{}max}\PYG{p}{)}
    \PYG{n+nb}{print}\PYG{p}{(}\PYG{l+s+s1}{\PYGZsq{}}\PYG{l+s+s1}{最大特征值对应的特征向量 w=}\PYG{l+s+s1}{\PYGZsq{}}\PYG{p}{,}\PYG{n}{weight}\PYG{p}{)}
    \PYG{n+nb}{print}\PYG{p}{(}\PYG{l+s+s1}{\PYGZsq{}}\PYG{l+s+s1}{CI=}\PYG{l+s+s1}{\PYGZsq{}}\PYG{p}{,}\PYG{n}{CI}\PYG{p}{)}
    \PYG{n+nb}{print}\PYG{p}{(}\PYG{l+s+s1}{\PYGZsq{}}\PYG{l+s+s1}{RI=}\PYG{l+s+s1}{\PYGZsq{}}\PYG{p}{,}\PYG{n}{RI}\PYG{p}{)}
    \PYG{n+nb}{print}\PYG{p}{(}\PYG{l+s+s1}{\PYGZsq{}}\PYG{l+s+s1}{CR=}\PYG{l+s+s1}{\PYGZsq{}}\PYG{p}{,}\PYG{n}{CR}\PYG{p}{)}
    \PYG{k}{return} \PYG{n}{weight}\PYG{p}{,}\PYG{n}{CI}\PYG{p}{,}\PYG{n}{RI}\PYG{p}{,}\PYG{n}{CR}
\end{sphinxVerbatim}

\begin{sphinxVerbatim}[commandchars=\\\{\}]
\PYG{n}{A} \PYG{o}{=} \PYG{n}{np}\PYG{o}{.}\PYG{n}{array}\PYG{p}{(}\PYG{p}{[}\PYG{p}{[}\PYG{l+m+mi}{1}\PYG{p}{,} \PYG{l+m+mi}{1}\PYG{o}{/}\PYG{l+m+mi}{2}\PYG{p}{,} \PYG{l+m+mi}{4}\PYG{p}{,}\PYG{l+m+mi}{3}\PYG{p}{,}\PYG{l+m+mi}{3}\PYG{p}{]}\PYG{p}{,}
             \PYG{p}{[}\PYG{l+m+mi}{2}\PYG{p}{,}  \PYG{l+m+mi}{1}\PYG{p}{,}    \PYG{l+m+mi}{7}\PYG{p}{,}\PYG{l+m+mi}{5}\PYG{p}{,}\PYG{l+m+mi}{5}\PYG{p}{]}\PYG{p}{,}
             \PYG{p}{[}\PYG{l+m+mi}{1}\PYG{o}{/}\PYG{l+m+mi}{4}\PYG{p}{,} \PYG{l+m+mi}{1}\PYG{o}{/}\PYG{l+m+mi}{7}\PYG{p}{,}   \PYG{l+m+mi}{1}\PYG{p}{,}\PYG{l+m+mi}{1}\PYG{o}{/}\PYG{l+m+mi}{2}\PYG{p}{,}\PYG{l+m+mi}{1}\PYG{o}{/}\PYG{l+m+mi}{3}\PYG{p}{]}\PYG{p}{,}
             \PYG{p}{[}\PYG{l+m+mi}{1}\PYG{o}{/}\PYG{l+m+mi}{3}\PYG{p}{,}\PYG{l+m+mi}{1}\PYG{o}{/}\PYG{l+m+mi}{5}\PYG{p}{,}\PYG{l+m+mi}{2}\PYG{p}{,}\PYG{l+m+mi}{1}\PYG{p}{,}\PYG{l+m+mi}{1}\PYG{p}{]}\PYG{p}{,}
             \PYG{p}{[}\PYG{l+m+mi}{1}\PYG{o}{/}\PYG{l+m+mi}{3}\PYG{p}{,}\PYG{l+m+mi}{1}\PYG{o}{/}\PYG{l+m+mi}{5}\PYG{p}{,}\PYG{l+m+mi}{3}\PYG{p}{,}\PYG{l+m+mi}{1}\PYG{p}{,}\PYG{l+m+mi}{1}\PYG{p}{]}\PYG{p}{]}\PYG{p}{)}
\PYG{c+c1}{\PYGZsh{} A = np.array([[1, 1/5, 1/3],}
\PYG{c+c1}{\PYGZsh{}              [5,  1,    3],}
\PYG{c+c1}{\PYGZsh{}              [3, 1/3,   1]])            \PYGZsh{} 输入判断矩阵}
\PYG{n}{weight}\PYG{p}{,}\PYG{n}{CI}\PYG{p}{,}\PYG{n}{RI}\PYG{p}{,}\PYG{n}{CR} \PYG{o}{=} \PYG{n}{AHP}\PYG{p}{(}\PYG{n}{A}\PYG{p}{)}
\end{sphinxVerbatim}

\begin{sphinxVerbatim}[commandchars=\\\{\}]
最大特征值 lambda\PYGZus{}max= 5.072084408570216
最大特征值对应的特征向量 w= [0.46582183 0.84086331 0.09509743 0.17329948 0.19204866]
CI= 0.018021102142554035
RI= 1.12
CR= 0.01609026977013753
\end{sphinxVerbatim}

\begin{sphinxVerbatim}[commandchars=\\\{\}]
\PYG{n}{B1} \PYG{o}{=} \PYG{n}{np}\PYG{o}{.}\PYG{n}{array}\PYG{p}{(}\PYG{p}{[}\PYG{p}{[}\PYG{l+m+mi}{1}\PYG{p}{,} \PYG{l+m+mi}{2}\PYG{p}{,} \PYG{l+m+mi}{3}\PYG{p}{,} \PYG{l+m+mi}{4}\PYG{p}{,} \PYG{l+m+mi}{7}\PYG{p}{]}\PYG{p}{,}
             \PYG{p}{[}\PYG{l+m+mi}{1}\PYG{o}{/}\PYG{l+m+mi}{3}\PYG{p}{,} \PYG{l+m+mi}{1}\PYG{p}{,} \PYG{l+m+mi}{3}\PYG{p}{,} \PYG{l+m+mi}{2}\PYG{p}{,} \PYG{l+m+mi}{5}\PYG{p}{]}\PYG{p}{,}
             \PYG{p}{[}\PYG{l+m+mi}{1}\PYG{o}{/}\PYG{l+m+mi}{5}\PYG{p}{,} \PYG{l+m+mi}{1}\PYG{o}{/}\PYG{l+m+mi}{3}\PYG{p}{,} \PYG{l+m+mi}{1}\PYG{p}{,} \PYG{l+m+mi}{1}\PYG{o}{/}\PYG{l+m+mi}{2}\PYG{p}{,} \PYG{l+m+mi}{1}\PYG{p}{]}\PYG{p}{,}
             \PYG{p}{[}\PYG{l+m+mi}{1}\PYG{o}{/}\PYG{l+m+mi}{4}\PYG{p}{,} \PYG{l+m+mi}{1}\PYG{o}{/}\PYG{l+m+mi}{2}\PYG{p}{,} \PYG{l+m+mi}{2}\PYG{p}{,} \PYG{l+m+mi}{1}\PYG{p}{,} \PYG{l+m+mi}{3}\PYG{p}{]}\PYG{p}{,}
             \PYG{p}{[}\PYG{l+m+mi}{1}\PYG{o}{/}\PYG{l+m+mi}{7}\PYG{p}{,} \PYG{l+m+mi}{1}\PYG{o}{/}\PYG{l+m+mi}{5}\PYG{p}{,} \PYG{l+m+mi}{1}\PYG{o}{/}\PYG{l+m+mi}{2}\PYG{p}{,} \PYG{l+m+mi}{1}\PYG{o}{/}\PYG{l+m+mi}{3}\PYG{p}{,} \PYG{l+m+mi}{1}\PYG{p}{]}\PYG{p}{]}\PYG{p}{)}     \PYG{c+c1}{\PYGZsh{} 输入判断矩阵}
\PYG{n}{weight}\PYG{p}{,}\PYG{n}{CI}\PYG{p}{,}\PYG{n}{RI}\PYG{p}{,}\PYG{n}{CR} \PYG{o}{=} \PYG{n}{AHP}\PYG{p}{(}\PYG{n}{B1}\PYG{p}{)}
\end{sphinxVerbatim}

\begin{sphinxVerbatim}[commandchars=\\\{\}]
最大特征值 lambda\PYGZus{}max= 4.788701831199925
最大特征值对应的特征向量 w= [0.82658145 0.46109858 0.14590859 0.27045896 0.09855884]
CI= \PYGZhy{}0.05282454220001864
RI= 1.12
CR= \PYGZhy{}0.04716476982144521
\end{sphinxVerbatim}

\begin{sphinxVerbatim}[commandchars=\\\{\}]
\PYG{n}{B2} \PYG{o}{=} \PYG{n}{np}\PYG{o}{.}\PYG{n}{array}\PYG{p}{(}\PYG{p}{[}\PYG{p}{[}\PYG{l+m+mi}{1}\PYG{p}{,} \PYG{l+m+mi}{1}\PYG{o}{/}\PYG{l+m+mi}{7}\PYG{p}{,} \PYG{l+m+mi}{1}\PYG{o}{/}\PYG{l+m+mi}{3}\PYG{p}{,} \PYG{l+m+mi}{1}\PYG{o}{/}\PYG{l+m+mi}{5}\PYG{p}{]}\PYG{p}{,}
              \PYG{p}{[}\PYG{l+m+mi}{7}\PYG{p}{,}  \PYG{l+m+mi}{1}\PYG{p}{,} \PYG{l+m+mi}{5} \PYG{p}{,} \PYG{l+m+mi}{3}\PYG{p}{]}\PYG{p}{,}
              \PYG{p}{[}\PYG{l+m+mi}{3}\PYG{p}{,} \PYG{l+m+mi}{1}\PYG{o}{/}\PYG{l+m+mi}{5}\PYG{p}{,} \PYG{l+m+mi}{1}\PYG{p}{,} \PYG{l+m+mi}{1}\PYG{o}{/}\PYG{l+m+mi}{3}\PYG{p}{]}\PYG{p}{,}
              \PYG{p}{[}\PYG{l+m+mi}{5}\PYG{p}{,} \PYG{l+m+mi}{1}\PYG{o}{/}\PYG{l+m+mi}{2}\PYG{p}{,} \PYG{l+m+mi}{3} \PYG{p}{,} \PYG{l+m+mi}{1}\PYG{p}{]}\PYG{p}{]}\PYG{p}{)}             \PYG{c+c1}{\PYGZsh{} 输入判断矩阵}
\PYG{n}{weight}\PYG{p}{,}\PYG{n}{CI}\PYG{p}{,}\PYG{n}{RI}\PYG{p}{,}\PYG{n}{CR} \PYG{o}{=} \PYG{n}{AHP}\PYG{p}{(}\PYG{n}{B2}\PYG{p}{)}
\end{sphinxVerbatim}

\begin{sphinxVerbatim}[commandchars=\\\{\}]
最大特征值 lambda\PYGZus{}max= 4.204930164418752
最大特征值对应的特征向量 w= [0.08512486 0.8766546  0.17994402 0.43800756]
CI= 0.06831005480625052
RI= 0.9
CR= 0.07590006089583391
\end{sphinxVerbatim}

\begin{sphinxVerbatim}[commandchars=\\\{\}]
\PYG{n}{B3} \PYG{o}{=} \PYG{n}{np}\PYG{o}{.}\PYG{n}{array}\PYG{p}{(}\PYG{p}{[}\PYG{p}{[}\PYG{l+m+mi}{1}\PYG{p}{,}\PYG{l+m+mi}{1}\PYG{p}{,}\PYG{l+m+mi}{3}\PYG{p}{,}\PYG{l+m+mi}{3}\PYG{p}{]}\PYG{p}{,}
             \PYG{p}{[}\PYG{l+m+mi}{1}\PYG{p}{,}\PYG{l+m+mi}{1}\PYG{p}{,}\PYG{l+m+mi}{3}\PYG{p}{,}\PYG{l+m+mi}{3}\PYG{p}{]}\PYG{p}{,}
             \PYG{p}{[}\PYG{l+m+mi}{1}\PYG{o}{/}\PYG{l+m+mi}{3}\PYG{p}{,} \PYG{l+m+mi}{1}\PYG{o}{/}\PYG{l+m+mi}{3}\PYG{p}{,} \PYG{l+m+mi}{1}\PYG{p}{,} \PYG{l+m+mi}{1}\PYG{p}{]}\PYG{p}{,}
              \PYG{p}{[}\PYG{l+m+mi}{1}\PYG{o}{/}\PYG{l+m+mi}{3}\PYG{p}{,}\PYG{l+m+mi}{1}\PYG{o}{/}\PYG{l+m+mi}{3}\PYG{p}{,}\PYG{l+m+mi}{1}\PYG{p}{,}\PYG{l+m+mi}{1}\PYG{p}{]}\PYG{p}{]}\PYG{p}{)}     \PYG{c+c1}{\PYGZsh{} 输入判断矩阵}
\PYG{n}{weight}\PYG{p}{,}\PYG{n}{CI}\PYG{p}{,}\PYG{n}{RI}\PYG{p}{,}\PYG{n}{CR} \PYG{o}{=} \PYG{n}{AHP}\PYG{p}{(}\PYG{n}{B3}\PYG{p}{)}
\end{sphinxVerbatim}

\begin{sphinxVerbatim}[commandchars=\\\{\}]
最大特征值 lambda\PYGZus{}max= 3.9999999999999996
最大特征值对应的特征向量 w= [0.67082039 0.67082039 0.2236068  0.2236068 ]
CI= \PYGZhy{}1.4802973661668753e\PYGZhy{}16
RI= 0.9
CR= \PYGZhy{}1.644774851296528e\PYGZhy{}16
\end{sphinxVerbatim}

\sphinxincludegraphics{{9604}.png}


\section{评价模型总结}
\label{\detokenize{docs/evaluation_model:id23}}
评价问题的流程:
\begin{enumerate}
\sphinxsetlistlabels{\arabic}{enumi}{enumii}{}{.}%
\item {} 
筛选指标

\item {} 
指标一致化和无量纲化

\item {} 
确定权重

\item {} 
使用合适的综合评价方法,加权综合求评分,得到结果

\end{enumerate}

\begin{sphinxadmonition}{tip}{Tip:}\begin{itemize}
\item {} 
优先选择客观方法,但也具有其局限性

\item {} 
如果选用了主观方法(AHP),一定不要忘记一致性检验和敏感性分析

\end{itemize}
\end{sphinxadmonition}


\chapter{预测模型}
\label{\detokenize{docs/prediction_model:id1}}\label{\detokenize{docs/prediction_model::doc}}

\section{预测模型介绍}
\label{\detokenize{docs/prediction_model:id2}}
\sphinxstylestrong{预测}(Prediction)是根据事物的历史资料及现状,运用一定的理论和方法,探求事物演变规律,对其未来发展状况作出的一种科学推测。生活中使用预测模型的场景比如:
\begin{itemize}
\item {} 
天气预报

\item {} 
飞机晚点预测

\item {} 
房价预测

\item {} 
股市

\end{itemize}

随着科学技术的发展和各种预测实践经验的积累,预测作为一门综合性科学已发展成为一门比较完善的学科,现在预测方法已近二百种(最常用的有十几种),但每种方法都有一定的使用范围,它们往往是相互补充的,实际预测时也常常是几种方法一起用。

预测方法虽然很多,但到目前为止,还没有一个统一的、普遍适用的分类体系,若根据预测的性质,大体可将预测分为\sphinxstylestrong{定性预测}和\sphinxstylestrong{定量预测}两类。

\begin{figure}[htbp]
\centering

\noindent\sphinxincludegraphics[height=450\sphinxpxdimen]{{01}.png}
\end{figure}

定量预测的特点是偏重于利用统计资料,借助于数学方法建立数学模型进行预测,是以数学模型为主的预测方法。又可分为\sphinxstylestrong{因果预测}和\sphinxstylestrong{时间序列预测}。
\begin{itemize}
\item {} 
\sphinxstylestrong{因果预测}是以相关原理来分析预测对象与有关因素的相互关系,并以此关系构造模型进行预测。例如,要预测人的血压与年龄的关系、贫血与缺铁的关系等都可以用此预测方法。常用的因果预测模型有\sphinxstylestrong{回归分析、数量经济模型、灰色系统模型、生命周期分析}等。

\item {} 
\sphinxstylestrong{时间序列预测}是根据预测对象时间序列的变化特征,来研究事物自身的发展规律和探讨未来发展趋势的。时间序列与因果预测的最重要区别就在于其研究的是“自身”的变化规律而不是因素之间的关系。

\end{itemize}


\section{线性回归}
\label{\detokenize{docs/prediction_model:id4}}
\sphinxstylestrong{确定性关系和相关关系:}
\begin{itemize}
\item {} 
\sphinxstylestrong{参数之间的确定性关系是指某参数可以完全由其他参数所决定},例如正方形的面积与边长的关系,总价格与单价和数量的关系等。

\item {} 
\sphinxstylestrong{相关关系是指一个或多个变量之间有一定的联系,但是这个联系不是完全决定性的}。例如,人的身高与体重有一定的关系,一般来讲身高高的人体重相对大一些。但是它们之间不能用一个确定的表达式表示出来。这次变量之间的关系。我们称之为相关关系。又如环境因素与农作物的产量也有相关关系,因为在相同环境条件下农作物的产量也有区别,这也就是说农作物的产量是一个随机变量。

\end{itemize}

**回归分析:**回归分析就是研究相关关系的一种数学方法,是寻找不完全确定的变量间的数学关系式并进行统计推断的一种方法。它能帮助我们从一个变量取得的值去估计另一个变量的值。

**线性回归:**回归分析中最简单的是线性回归。


\subsection{一元线性回归模型}
\label{\detokenize{docs/prediction_model:id5}}
单变量线性回归,又称简单线性回归(simple linear regression,SLR),是最简单但用途很广的回归模型。其回归式为:
\begin{equation*}
\begin{split}
y=\alpha+\beta x
\end{split}
\end{equation*}
为了从一组样本\(\{\left(x_{i}, y_{i}\right)|i = 1,2,\cdots,n\}\)之中估计最合适的参数\(\alpha\)和\(\beta\),通常采用\sphinxstylestrong{最小二乘法},其计算目标为\sphinxstylestrong{最小化残差平方和}:
\begin{equation*}
\begin{split}
\min \sum_{i=1}^{n} \varepsilon_{i}^{2}=\min \sum_{i=1}^{n}\left(y_{i}-\alpha-\beta x_{i}\right)^{2}
\end{split}
\end{equation*}
\sphinxstylestrong{本课程我们暂不关心最小二乘法的推导过程,只需了解其原理及应用即可。} 这里直接给出最小二乘估计的结果
\begin{equation*}
\begin{split}
\begin{aligned}
\hat{\alpha}&=\bar{y}-\bar{x} \hat{\beta}\\
\hat{\beta}&=S_{x y} / S_{x x}
\end{aligned}
\end{split}
\end{equation*}
其中,
\begin{equation*}
\begin{split}
\bar{y}=\frac{1}{n} \sum_{i} y_{i}
\end{split}
\end{equation*}\begin{equation*}
\begin{split}
\bar{x}=\frac{1}{n} \sum_{i} x_{i}
\end{split}
\end{equation*}\begin{equation*}
\begin{split}
S_{x x}=\sum_{i}\left(x_{i}-\bar{x}\right)^{2}
\end{split}
\end{equation*}\begin{equation*}
\begin{split}
S_{y y}=\sum_{i}\left(y_{i}-\bar{y}\right)^{2}
\end{split}
\end{equation*}\begin{equation*}
\begin{split}
S_{x y}=\sum_{i}\left(x_{i}-\bar{x}\right)\left(y_{i}-\bar{y}\right)
\end{split}
\end{equation*}
现在有一组高中生的身高和腿长的数据,用一元线性回归研究他们之间的关系,然后对新的同学进行预测。先来看看数据的\sphinxstylestrong{散点图}

\begin{sphinxVerbatim}[commandchars=\\\{\}]
\PYG{k+kn}{import} \PYG{n+nn}{numpy} \PYG{k}{as} \PYG{n+nn}{np}    \PYG{c+c1}{\PYGZsh{} 导入 numpy库,用于科学计算}
\PYG{k+kn}{import} \PYG{n+nn}{pandas} \PYG{k}{as} \PYG{n+nn}{pd}    \PYG{c+c1}{\PYGZsh{} 导入 pandas库 ,用于数据分析}
\PYG{k+kn}{import} \PYG{n+nn}{matplotlib}\PYG{n+nn}{.}\PYG{n+nn}{pyplot} \PYG{k}{as} \PYG{n+nn}{plt}   \PYG{c+c1}{\PYGZsh{} 导入 pandas库 ,用于数据可视化}

\PYG{o}{\PYGZpc{}}\PYG{k}{matplotlib} inline 
\PYG{n}{plt}\PYG{o}{.}\PYG{n}{style}\PYG{o}{.}\PYG{n}{use}\PYG{p}{(}\PYG{l+s+s2}{\PYGZdq{}}\PYG{l+s+s2}{ggplot}\PYG{l+s+s2}{\PYGZdq{}}\PYG{p}{)}  \PYG{c+c1}{\PYGZsh{} 使用ggplot绘图风格}
\PYG{k+kn}{from} \PYG{n+nn}{sklearn}\PYG{n+nn}{.}\PYG{n+nn}{linear\PYGZus{}model} \PYG{k+kn}{import} \PYG{n}{LinearRegression}  \PYG{c+c1}{\PYGZsh{} 导入线性回归工具函数 LinearRegression}
\PYG{n}{x} \PYG{o}{=} \PYG{n}{np}\PYG{o}{.}\PYG{n}{array}\PYG{p}{(}\PYG{p}{[}\PYG{l+m+mi}{143}\PYG{p}{,} \PYG{l+m+mi}{145}\PYG{p}{,} \PYG{l+m+mi}{146}\PYG{p}{,} \PYG{l+m+mi}{147}\PYG{p}{,} \PYG{l+m+mi}{149}\PYG{p}{,} \PYG{l+m+mi}{150}\PYG{p}{,} \PYG{l+m+mi}{153}\PYG{p}{,} \PYG{l+m+mi}{154}\PYG{p}{,} \PYG{l+m+mi}{155}\PYG{p}{,}
              \PYG{l+m+mi}{156}\PYG{p}{,} \PYG{l+m+mi}{157}\PYG{p}{,} \PYG{l+m+mi}{158}\PYG{p}{,} \PYG{l+m+mi}{159}\PYG{p}{,} \PYG{l+m+mi}{160}\PYG{p}{,} \PYG{l+m+mi}{162}\PYG{p}{,} \PYG{l+m+mi}{164}\PYG{p}{]}\PYG{p}{)}  \PYG{c+c1}{\PYGZsh{} 输入x数据}
\PYG{n}{x} \PYG{o}{=} \PYG{n}{x}\PYG{o}{.}\PYG{n}{reshape}\PYG{p}{(}\PYG{l+m+mi}{16}\PYG{p}{,} \PYG{l+m+mi}{1}\PYG{p}{)}    \PYG{c+c1}{\PYGZsh{} 修改数据的格式,从行向量转换为列向量}
\PYG{n}{y} \PYG{o}{=} \PYG{n}{np}\PYG{o}{.}\PYG{n}{array}\PYG{p}{(}\PYG{p}{[}\PYG{l+m+mi}{88}\PYG{p}{,} \PYG{l+m+mi}{85}\PYG{p}{,} \PYG{l+m+mi}{88}\PYG{p}{,} \PYG{l+m+mi}{91}\PYG{p}{,} \PYG{l+m+mi}{92}\PYG{p}{,} \PYG{l+m+mi}{93}\PYG{p}{,} \PYG{l+m+mi}{93}\PYG{p}{,} \PYG{l+m+mi}{95}\PYG{p}{,} \PYG{l+m+mi}{96}\PYG{p}{,}
              \PYG{l+m+mi}{98}\PYG{p}{,} \PYG{l+m+mi}{97}\PYG{p}{,} \PYG{l+m+mi}{96}\PYG{p}{,} \PYG{l+m+mi}{98}\PYG{p}{,} \PYG{l+m+mi}{99}\PYG{p}{,} \PYG{l+m+mi}{100}\PYG{p}{,} \PYG{l+m+mi}{102}\PYG{p}{]}\PYG{p}{)}  \PYG{c+c1}{\PYGZsh{} 输入y数据}
\PYG{n}{plt}\PYG{o}{.}\PYG{n}{scatter}\PYG{p}{(}\PYG{n}{x}\PYG{p}{,} \PYG{n}{y}\PYG{p}{)}    \PYG{c+c1}{\PYGZsh{} 绘制散点图}
\PYG{n}{plt}\PYG{o}{.}\PYG{n}{xlabel}\PYG{p}{(}\PYG{l+s+sa}{r}\PYG{l+s+s1}{\PYGZsq{}}\PYG{l+s+s1}{\PYGZdl{}H\PYGZdl{}}\PYG{l+s+s1}{\PYGZsq{}}\PYG{p}{)}  \PYG{c+c1}{\PYGZsh{} 添加xlabel}
\PYG{n}{plt}\PYG{o}{.}\PYG{n}{ylabel}\PYG{p}{(}\PYG{l+s+sa}{r}\PYG{l+s+s1}{\PYGZsq{}}\PYG{l+s+s1}{\PYGZdl{}L\PYGZdl{}}\PYG{l+s+s1}{\PYGZsq{}}\PYG{p}{)}  \PYG{c+c1}{\PYGZsh{} 绘制ylabel}
\end{sphinxVerbatim}

\begin{sphinxVerbatim}[commandchars=\\\{\}]
Text(0, 0.5, \PYGZsq{}\PYGZdl{}L\PYGZdl{}\PYGZsq{})
\end{sphinxVerbatim}

\noindent\sphinxincludegraphics{{prediction_model_7_1}.png}

很明显,升身高和体重明显\sphinxstylestrong{符合线性关系}。下面我们使用\sphinxcode{\sphinxupquote{sklearn.linear\_model}}中的\sphinxcode{\sphinxupquote{LinearRegression}}方法进行线性回归

\begin{sphinxVerbatim}[commandchars=\\\{\}]
\PYG{c+c1}{\PYGZsh{} 1. 初始化线性回归函数,命名为lrModel}
\PYG{n}{lrModel} \PYG{o}{=} \PYG{n}{LinearRegression}\PYG{p}{(}\PYG{p}{)}
\PYG{c+c1}{\PYGZsh{} 2. 使用lrModel对数据x,y进行拟合}
\PYG{n}{lrModel}\PYG{o}{.}\PYG{n}{fit}\PYG{p}{(}\PYG{n}{x}\PYG{p}{,}\PYG{n}{y}\PYG{p}{)}
\end{sphinxVerbatim}

\begin{sphinxVerbatim}[commandchars=\\\{\}]
LinearRegression()
\end{sphinxVerbatim}

\begin{sphinxVerbatim}[commandchars=\\\{\}]
\PYG{c+c1}{\PYGZsh{} 3. 输出截距和斜率}
\PYG{n+nb}{print}\PYG{p}{(}\PYG{n}{lrModel}\PYG{o}{.}\PYG{n}{intercept\PYGZus{}}\PYG{p}{)}
\PYG{n+nb}{print}\PYG{p}{(}\PYG{n}{lrModel}\PYG{o}{.}\PYG{n}{coef\PYGZus{}}\PYG{p}{)}
\end{sphinxVerbatim}

\begin{sphinxVerbatim}[commandchars=\\\{\}]
\PYGZhy{}16.07298072980727
[0.71935219]
\end{sphinxVerbatim}

\begin{sphinxVerbatim}[commandchars=\\\{\}]
\PYG{c+c1}{\PYGZsh{} 4. 根据回归得到的系数,在散点图上绘制回归直线}
\PYG{n}{LR\PYGZus{}data} \PYG{o}{=} \PYG{p}{[}\PYG{n}{lrModel}\PYG{o}{.}\PYG{n}{intercept\PYGZus{}} \PYG{o}{+} \PYG{n}{lrModel}\PYG{o}{.}\PYG{n}{coef\PYGZus{}} \PYG{o}{*} \PYG{n}{i} \PYG{k}{for} \PYG{n}{i} \PYG{o+ow}{in} \PYG{n+nb}{range}\PYG{p}{(}\PYG{l+m+mi}{145}\PYG{p}{,}\PYG{l+m+mi}{165}\PYG{p}{)}\PYG{p}{]}  \PYG{c+c1}{\PYGZsh{} 求解回归直线上的一系列点}
\PYG{n}{plt}\PYG{o}{.}\PYG{n}{plot}\PYG{p}{(}\PYG{n+nb}{range}\PYG{p}{(}\PYG{l+m+mi}{145}\PYG{p}{,} \PYG{l+m+mi}{165}\PYG{p}{)}\PYG{p}{,} \PYG{n}{LR\PYGZus{}data}\PYG{p}{)}  \PYG{c+c1}{\PYGZsh{} 绘制回归直线}
\PYG{n}{plt}\PYG{o}{.}\PYG{n}{scatter}\PYG{p}{(}\PYG{n}{x}\PYG{p}{,} \PYG{n}{y}\PYG{p}{)}   \PYG{c+c1}{\PYGZsh{} 绘制散点图}
\end{sphinxVerbatim}

\begin{sphinxVerbatim}[commandchars=\\\{\}]
\PYGZlt{}matplotlib.collections.PathCollection at 0x12a5f0e10\PYGZgt{}
\end{sphinxVerbatim}

\noindent\sphinxincludegraphics{{prediction_model_11_1}.png}

我们常常用相关系数\(R^2\)描述线性拟合的效果,使用方法如下

\begin{sphinxVerbatim}[commandchars=\\\{\}]
\PYG{c+c1}{\PYGZsh{} 5. 计算R2}
\PYG{n}{lrModel}\PYG{o}{.}\PYG{n}{score}\PYG{p}{(}\PYG{n}{x}\PYG{p}{,}\PYG{n}{y}\PYG{p}{)}
\end{sphinxVerbatim}

\begin{sphinxVerbatim}[commandchars=\\\{\}]
0.928187845952738
\end{sphinxVerbatim}

执行代码可以看到,模型的评分为\sphinxcode{\sphinxupquote{0.928}},是非常不错的一个评分,我们就可以使用这个模型进行未知数据的预测了。(评分等于相关系数\(R^2\)用于表示拟合得到的模型能解释因变量变化的百分比,\(R^2\)越接近于1,表示回归模型拟合效果越好,通常认为\(R^2 > 0.8\) 就是比较满意的回归结果了)

假定两位新同学的身高分别是170cm和163cm,下面使用本文的模型预测他们的身高。

\begin{sphinxVerbatim}[commandchars=\\\{\}]
\PYG{c+c1}{\PYGZsh{} 6. 对新的数据进行预测}
\PYG{n}{lrModel}\PYG{o}{.}\PYG{n}{predict}\PYG{p}{(}\PYG{p}{[}\PYG{p}{[}\PYG{l+m+mi}{170}\PYG{p}{]}\PYG{p}{,}\PYG{p}{[}\PYG{l+m+mi}{163}\PYG{p}{]}\PYG{p}{]}\PYG{p}{)}
\end{sphinxVerbatim}

\begin{sphinxVerbatim}[commandchars=\\\{\}]
array([106.21689217, 101.18142681])
\end{sphinxVerbatim}


\subsection{多元线性回归模型}
\label{\detokenize{docs/prediction_model:id6}}
在回归分析中,如果有两个或两个以上的自变量,就称为多元回归。事实上,一种现象常常是与多个因素相联系的,由多个自变量的最优组合共同来预测或估计因变量,比只用一个自变量进行预测或估计更有效,更符合实际。

在实际经济问题中,一个变量往往受到多个变量的影响。例如,家庭消费支出,除了受家庭可支配收入的影响外,还受诸如家庭所有的财富、物价水平、金融机构存款利息等多种因素的影响,表现在线性回归模型中的解释变量有多个。这样的模型被称为\sphinxstylestrong{多元线性回归模型(multivariable linear regression model)}
\begin{equation*}
\begin{split}
y=\beta_{0}+\beta_{1} x_{1}+\beta_{2} x_{2}+\cdots \beta_{m} x_{m}
\end{split}
\end{equation*}
多元性回归模型的参数估计,同一元线性回归方程一样,也是在要求误差平方和最小的前提下,用最小二乘法求解参数。
\begin{equation*}
\begin{split}
\min \sum_{i=1}^{n} \varepsilon_{i}^{2}=\min \sum_{i=1}^{n}\left(y_{i}-\left(\beta_{0}+\beta_{1} x_{1i}+\beta_{2} x_{2i}+\cdots \beta_{m} x_{mi}\right)\right)^{2}
\end{split}
\end{equation*}
同样地,这里我们暂时不关心回归系数的具体的推导过程,只需要知道如何用Python操作即可。我们通过一个SO2浓度预测的案例进行介绍。

\begin{sphinxVerbatim}[commandchars=\\\{\}]
\PYG{k+kn}{import} \PYG{n+nn}{pandas} \PYG{k}{as} \PYG{n+nn}{pd} \PYG{c+c1}{\PYGZsh{} 导入 pandas库 ,用于数据分析}
\PYG{n}{data} \PYG{o}{=} \PYG{n}{pd}\PYG{o}{.}\PYG{n}{read\PYGZus{}csv}\PYG{p}{(}\PYG{l+s+s1}{\PYGZsq{}}\PYG{l+s+s1}{../\PYGZus{}static/lecture\PYGZus{}specific/prediction\PYGZus{}model/so2.csv}\PYG{l+s+s1}{\PYGZsq{}}\PYG{p}{)} \PYG{c+c1}{\PYGZsh{} 读入数据}
\PYG{n+nb}{print}\PYG{p}{(}\PYG{n}{data}\PYG{p}{)} \PYG{c+c1}{\PYGZsh{} 显示数据}
\end{sphinxVerbatim}

\begin{sphinxVerbatim}[commandchars=\\\{\}]
    SO2(ppm)    R    G    B    S    H
0          0  153  148  157  138   14
1          0  153  147  157  138   16
2          0  153  146  158  137   20
3          0  153  146  158  137   20
4          0  154  145  157  141   19
5         20  144  115  170  135   82
6         20  144  115  169  136   81
7         20  145  115  172  135   83
8         30  145  114  174  135   87
9         30  145  114  176  135   89
10        30  145  114  175  135   89
11        30  146  114  175  135   88
12        50  142   99  175  137  110
13        50  141   99  174  137  109
14        50  142   99  176  136  110
15        80  141   96  181  135  119
16        80  141   96  182  135  119
17        80  140   96  182  135  120
18       100  139   96  175  136  115
19       100  139   96  174  136  114
20       100  139   96  176  136  116
21       150  139   86  178  136  131
22       150  139   87  177  137  129
23       150  138   86  177  137  130
24       150  139   86  178  137  131
\end{sphinxVerbatim}

可以看到,这是一个\sphinxstylestrong{五个变量}的多元线性回归,我们先通过散点图来看一下\sphinxstylestrong{各个变量单独与因变量的关系}。

\begin{sphinxVerbatim}[commandchars=\\\{\}]
\PYG{k+kn}{import} \PYG{n+nn}{seaborn} \PYG{k}{as} \PYG{n+nn}{sns}  \PYG{c+c1}{\PYGZsh{} 导入 seaborn库 ,用于数据可视化}
\PYG{n}{colors} \PYG{o}{=} \PYG{p}{[}\PYG{l+s+s1}{\PYGZsq{}}\PYG{l+s+s1}{r}\PYG{l+s+s1}{\PYGZsq{}}\PYG{p}{,} \PYG{l+s+s1}{\PYGZsq{}}\PYG{l+s+s1}{g}\PYG{l+s+s1}{\PYGZsq{}}\PYG{p}{,} \PYG{l+s+s1}{\PYGZsq{}}\PYG{l+s+s1}{b}\PYG{l+s+s1}{\PYGZsq{}}\PYG{p}{,} \PYG{l+s+s1}{\PYGZsq{}}\PYG{l+s+s1}{m}\PYG{l+s+s1}{\PYGZsq{}}\PYG{p}{,} \PYG{l+s+s1}{\PYGZsq{}}\PYG{l+s+s1}{black}\PYG{l+s+s1}{\PYGZsq{}}\PYG{p}{]}  \PYG{c+c1}{\PYGZsh{} 指定每个绘图的颜色}
\PYG{n}{xlist} \PYG{o}{=} \PYG{p}{[}\PYG{l+s+s1}{\PYGZsq{}}\PYG{l+s+s1}{R}\PYG{l+s+s1}{\PYGZsq{}}\PYG{p}{,} \PYG{l+s+s1}{\PYGZsq{}}\PYG{l+s+s1}{G}\PYG{l+s+s1}{\PYGZsq{}}\PYG{p}{,} \PYG{l+s+s1}{\PYGZsq{}}\PYG{l+s+s1}{B}\PYG{l+s+s1}{\PYGZsq{}}\PYG{p}{,} \PYG{l+s+s1}{\PYGZsq{}}\PYG{l+s+s1}{S}\PYG{l+s+s1}{\PYGZsq{}}\PYG{p}{,} \PYG{l+s+s1}{\PYGZsq{}}\PYG{l+s+s1}{H}\PYG{l+s+s1}{\PYGZsq{}}\PYG{p}{]}  \PYG{c+c1}{\PYGZsh{} 指定每个绘图的变量名}
\PYG{n}{plt}\PYG{o}{.}\PYG{n}{figure}\PYG{p}{(}\PYG{n}{figsize}\PYG{o}{=}\PYG{p}{(}\PYG{l+m+mi}{20}\PYG{p}{,} \PYG{l+m+mi}{13}\PYG{p}{)}\PYG{p}{)}  \PYG{c+c1}{\PYGZsh{} 指定图片大小}
\PYG{k}{for} \PYG{n}{i} \PYG{o+ow}{in} \PYG{n+nb}{range}\PYG{p}{(}\PYG{l+m+mi}{5}\PYG{p}{)}\PYG{p}{:}
    \PYG{n}{plt}\PYG{o}{.}\PYG{n}{subplot}\PYG{p}{(}\PYG{l+m+mi}{2}\PYG{p}{,} \PYG{l+m+mi}{3}\PYG{p}{,} \PYG{n}{i}\PYG{o}{+}\PYG{l+m+mi}{1}\PYG{p}{)}  \PYG{c+c1}{\PYGZsh{} 绘制子图}
    \PYG{n}{title} \PYG{o}{=} \PYG{l+s+s1}{\PYGZsq{}}\PYG{l+s+s1}{r = }\PYG{l+s+s1}{\PYGZsq{}} \PYG{o}{+} \PYG{n+nb}{str}\PYG{p}{(}\PYG{n+nb}{round}\PYG{p}{(}\PYG{n}{np}\PYG{o}{.}\PYG{n}{corrcoef}\PYG{p}{(}\PYG{n}{data}\PYG{p}{[}\PYG{n}{xlist}\PYG{p}{[}\PYG{n}{i}\PYG{p}{]}\PYG{p}{]}\PYG{p}{,}\PYG{n}{data}\PYG{p}{[}\PYG{l+s+s1}{\PYGZsq{}}\PYG{l+s+s1}{SO2(ppm)}\PYG{l+s+s1}{\PYGZsq{}}\PYG{p}{]}\PYG{p}{)}\PYG{p}{[}\PYG{l+m+mi}{1}\PYG{p}{]}\PYG{p}{[}\PYG{l+m+mi}{0}\PYG{p}{]}\PYG{p}{,} \PYG{l+m+mi}{3}\PYG{p}{)}\PYG{p}{)}  \PYG{c+c1}{\PYGZsh{} 定义子图标题}
    \PYG{n}{sns}\PYG{o}{.}\PYG{n}{regplot}\PYG{p}{(}\PYG{n}{x}\PYG{o}{=}\PYG{n}{xlist}\PYG{p}{[}\PYG{n}{i}\PYG{p}{]}\PYG{p}{,} \PYG{n}{y}\PYG{o}{=}\PYG{l+s+s2}{\PYGZdq{}}\PYG{l+s+s2}{SO2(ppm)}\PYG{l+s+s2}{\PYGZdq{}}\PYG{p}{,} \PYG{n}{data}\PYG{o}{=}\PYG{n}{data}\PYG{p}{,} \PYG{n}{color}\PYG{o}{=}\PYG{n}{colors}\PYG{p}{[}\PYG{n}{i}\PYG{p}{]}\PYG{p}{)}  \PYG{c+c1}{\PYGZsh{} 绘制带有回归线和置信区间的线性回归图}
    \PYG{n}{plt}\PYG{o}{.}\PYG{n}{title}\PYG{p}{(}\PYG{n}{title}\PYG{p}{)}  \PYG{c+c1}{\PYGZsh{} 添加子图标题}
\end{sphinxVerbatim}

\noindent\sphinxincludegraphics{{prediction_model_21_0}.png}

可以看到,除了第四个变量\(S\)以外,每一个变量基本上都和因变量有线性关系,因此我们可以删去\(S\),进行多元线性回归。

\begin{sphinxVerbatim}[commandchars=\\\{\}]
\PYG{k+kn}{from} \PYG{n+nn}{sklearn}\PYG{n+nn}{.}\PYG{n+nn}{linear\PYGZus{}model} \PYG{k+kn}{import} \PYG{n}{LinearRegression}  \PYG{c+c1}{\PYGZsh{} 导入线性回归函数LinearRegression}
\PYG{n}{lrModel} \PYG{o}{=} \PYG{n}{LinearRegression}\PYG{p}{(}\PYG{p}{)}     \PYG{c+c1}{\PYGZsh{} 初始化回归模型}
\PYG{n}{lrModel}\PYG{o}{.}\PYG{n}{fit}\PYG{p}{(}\PYG{n}{np}\PYG{o}{.}\PYG{n}{array}\PYG{p}{(}\PYG{n}{data}\PYG{p}{[}\PYG{p}{[}\PYG{l+s+s1}{\PYGZsq{}}\PYG{l+s+s1}{R}\PYG{l+s+s1}{\PYGZsq{}}\PYG{p}{,} \PYG{l+s+s1}{\PYGZsq{}}\PYG{l+s+s1}{G}\PYG{l+s+s1}{\PYGZsq{}}\PYG{p}{,} \PYG{l+s+s1}{\PYGZsq{}}\PYG{l+s+s1}{B}\PYG{l+s+s1}{\PYGZsq{}}\PYG{p}{,} \PYG{l+s+s1}{\PYGZsq{}}\PYG{l+s+s1}{H}\PYG{l+s+s1}{\PYGZsq{}}\PYG{p}{]}\PYG{p}{]}\PYG{p}{)}\PYG{p}{,}
            \PYG{n}{data}\PYG{p}{[}\PYG{l+s+s1}{\PYGZsq{}}\PYG{l+s+s1}{SO2(ppm)}\PYG{l+s+s1}{\PYGZsq{}}\PYG{p}{]}\PYG{p}{)}  \PYG{c+c1}{\PYGZsh{} 输入需要回归的数据}
\PYG{n+nb}{print}\PYG{p}{(}\PYG{l+s+s1}{\PYGZsq{}}\PYG{l+s+s1}{截距为:}\PYG{l+s+s1}{\PYGZsq{}}\PYG{p}{,} \PYG{n}{lrModel}\PYG{o}{.}\PYG{n}{intercept\PYGZus{}}\PYG{p}{)}  \PYG{c+c1}{\PYGZsh{} 输出截距}
\PYG{n+nb}{print}\PYG{p}{(}\PYG{l+s+s1}{\PYGZsq{}}\PYG{l+s+s1}{系数为:}\PYG{l+s+s1}{\PYGZsq{}}\PYG{p}{,} \PYG{n}{lrModel}\PYG{o}{.}\PYG{n}{coef\PYGZus{}}\PYG{p}{)}    \PYG{c+c1}{\PYGZsh{} 输出系数}
\PYG{n}{score} \PYG{o}{=} \PYG{n}{lrModel}\PYG{o}{.}\PYG{n}{score}\PYG{p}{(}\PYG{n}{np}\PYG{o}{.}\PYG{n}{array}\PYG{p}{(}\PYG{n}{data}\PYG{p}{[}\PYG{p}{[}\PYG{l+s+s1}{\PYGZsq{}}\PYG{l+s+s1}{R}\PYG{l+s+s1}{\PYGZsq{}}\PYG{p}{,} \PYG{l+s+s1}{\PYGZsq{}}\PYG{l+s+s1}{G}\PYG{l+s+s1}{\PYGZsq{}}\PYG{p}{,} \PYG{l+s+s1}{\PYGZsq{}}\PYG{l+s+s1}{B}\PYG{l+s+s1}{\PYGZsq{}}\PYG{p}{,} \PYG{l+s+s1}{\PYGZsq{}}\PYG{l+s+s1}{H}\PYG{l+s+s1}{\PYGZsq{}}\PYG{p}{]}\PYG{p}{]}\PYG{p}{)}\PYG{p}{,} \PYG{n}{data}\PYG{p}{[}\PYG{l+s+s1}{\PYGZsq{}}\PYG{l+s+s1}{SO2(ppm)}\PYG{l+s+s1}{\PYGZsq{}}\PYG{p}{]}\PYG{p}{)}
\PYG{n+nb}{print}\PYG{p}{(}\PYG{l+s+s1}{\PYGZsq{}}\PYG{l+s+s1}{R2为:}\PYG{l+s+s1}{\PYGZsq{}}\PYG{p}{,} \PYG{n}{score}\PYG{p}{)}    \PYG{c+c1}{\PYGZsh{} 输出相关系数R2}
\end{sphinxVerbatim}

\begin{sphinxVerbatim}[commandchars=\\\{\}]
截距为: 2044.021593204974
系数为: [ \PYGZhy{}1.38977229 \PYGZhy{}17.5022507    5.68341432  \PYGZhy{}9.34216398]
R2为: 0.8961318172850157
\end{sphinxVerbatim}


\section{非线性拟合}
\label{\detokenize{docs/prediction_model:id7}}

\subsection{一元非线性多项式拟合}
\label{\detokenize{docs/prediction_model:id8}}
尽管上述的线性模型已经能够求解许多实际问题,然而生活中的很多系统的增长趋势往往是非线性的,这时候我们就需要用到非线性拟合。

对于非线性回归问题而言,最简单也是最常见的方法就是\sphinxstylestrong{多项式回归}。多项式(Polynomial)是代数学中的基础概念,是由称为未知数的变量和称为系数的常量通过有限次加法、加减法、乘法以及自然数幂次的乘方运算得到的代数表达式。多项式是整式的一种。未知数只有一个的多项式称为一元多项式;例如\(x^{2}-3 x+4\)就是一个一元多项式。未知数不止一个的多项式称为多元多项式,例如\(x^{3}-2 x y z^{2}+2 y z+1\)就是一个三元多项式。

首先,我们通过一组示例数据来认识多项式回归,示例数据一共有 10 组,分别对应着横坐标和纵坐标。接下来,通过 Matplotlib 绘制数据,查看其变化趋势。

\begin{sphinxVerbatim}[commandchars=\\\{\}]
\PYG{c+c1}{\PYGZsh{} 加载示例数据}
\PYG{n}{x} \PYG{o}{=} \PYG{p}{[}\PYG{l+m+mi}{4}\PYG{p}{,} \PYG{l+m+mi}{8}\PYG{p}{,} \PYG{l+m+mi}{12}\PYG{p}{,} \PYG{l+m+mi}{25}\PYG{p}{,} \PYG{l+m+mi}{32}\PYG{p}{,} \PYG{l+m+mi}{43}\PYG{p}{,} \PYG{l+m+mi}{58}\PYG{p}{,} \PYG{l+m+mi}{63}\PYG{p}{,} \PYG{l+m+mi}{69}\PYG{p}{,} \PYG{l+m+mi}{79}\PYG{p}{]} \PYG{c+c1}{\PYGZsh{} 输入x}
\PYG{n}{y} \PYG{o}{=} \PYG{p}{[}\PYG{l+m+mi}{20}\PYG{p}{,} \PYG{l+m+mi}{33}\PYG{p}{,} \PYG{l+m+mi}{50}\PYG{p}{,} \PYG{l+m+mi}{56}\PYG{p}{,} \PYG{l+m+mi}{42}\PYG{p}{,} \PYG{l+m+mi}{31}\PYG{p}{,} \PYG{l+m+mi}{33}\PYG{p}{,} \PYG{l+m+mi}{46}\PYG{p}{,} \PYG{l+m+mi}{65}\PYG{p}{,} \PYG{l+m+mi}{75}\PYG{p}{]}  \PYG{c+c1}{\PYGZsh{} 输入y }
\PYG{n}{plt}\PYG{o}{.}\PYG{n}{scatter}\PYG{p}{(}\PYG{n}{x}\PYG{p}{,} \PYG{n}{y}\PYG{p}{)}  \PYG{c+c1}{\PYGZsh{} 绘制散点图}
\end{sphinxVerbatim}

\begin{sphinxVerbatim}[commandchars=\\\{\}]
\PYGZlt{}matplotlib.collections.PathCollection at 0x12b592b50\PYGZgt{}
\end{sphinxVerbatim}

\noindent\sphinxincludegraphics{{prediction_model_26_1}.png}

接下来,通过多项式来拟合上面的散点数据。一个标准的一元高阶多项式函数如下所示:
\begin{equation*}
\begin{split}
y(x, w)=w_{0}+w_{1} x+w_{2} x^{2}+\ldots+w_{m} x^{m}=\sum_{j=0}^{m} w_{j} x^{j}
\end{split}
\end{equation*}
其中,\(m\)表示多项式的阶数,\(x^j\)表示\(x\)的\(j\)次幂,\(w\)则代表该多项式的系数。当我们使用上面的多项式去拟合散点时,需要确定两个要素,分别是:
\begin{itemize}
\item {} 
多项式系数\(w\)

\item {} 
多项式阶数\(m\)

\end{itemize}

这也是多项式的两个基本要素。

如果通过手动指定多项式阶数\(m\)的大小,那么就只需要确定多项式系数\(w\)的值是多少。例如,这里首先指定\(m=2\),多项式就变成了:
\$\(
y(x, w)=w_{0}+w_{1} x+w_{2} x^{2}=\sum_{j=0}^{2} w_{j} x^{j}
\)\$

当我们确定\(w\)的值的大小时,就回到了前面线性回归中学习到的内容,具体来说,就是最小化残差平方和(最小二乘法)。

我们先尝试使用二次多项式拟合。

\begin{sphinxVerbatim}[commandchars=\\\{\}]
\PYG{c+c1}{\PYGZsh{} 加载示例数据}
\PYG{n}{x} \PYG{o}{=} \PYG{p}{[}\PYG{l+m+mi}{4}\PYG{p}{,} \PYG{l+m+mi}{8}\PYG{p}{,} \PYG{l+m+mi}{12}\PYG{p}{,} \PYG{l+m+mi}{25}\PYG{p}{,} \PYG{l+m+mi}{32}\PYG{p}{,} \PYG{l+m+mi}{43}\PYG{p}{,} \PYG{l+m+mi}{58}\PYG{p}{,} \PYG{l+m+mi}{63}\PYG{p}{,} \PYG{l+m+mi}{69}\PYG{p}{,} \PYG{l+m+mi}{79}\PYG{p}{]}
\PYG{n}{y} \PYG{o}{=} \PYG{p}{[}\PYG{l+m+mi}{20}\PYG{p}{,} \PYG{l+m+mi}{33}\PYG{p}{,} \PYG{l+m+mi}{50}\PYG{p}{,} \PYG{l+m+mi}{56}\PYG{p}{,} \PYG{l+m+mi}{42}\PYG{p}{,} \PYG{l+m+mi}{31}\PYG{p}{,} \PYG{l+m+mi}{33}\PYG{p}{,} \PYG{l+m+mi}{46}\PYG{p}{,} \PYG{l+m+mi}{65}\PYG{p}{,} \PYG{l+m+mi}{75}\PYG{p}{]}
\PYG{k+kn}{from} \PYG{n+nn}{scipy}\PYG{n+nn}{.}\PYG{n+nn}{optimize} \PYG{k+kn}{import} \PYG{n}{curve\PYGZus{}fit}  \PYG{c+c1}{\PYGZsh{} 导入非线性拟合函数curve\PYGZus{}fit}

\PYG{c+c1}{\PYGZsh{} 定义需要拟合的函数形式,这里使用二次函数的一般式 y = ax\PYGZca{}2 + bx + c}
\PYG{k}{def} \PYG{n+nf}{f2}\PYG{p}{(}\PYG{n}{x}\PYG{p}{,} \PYG{n}{a}\PYG{p}{,} \PYG{n}{b}\PYG{p}{,} \PYG{n}{c}\PYG{p}{)}\PYG{p}{:}
    \PYG{k}{return} \PYG{n}{a} \PYG{o}{*} \PYG{n}{x}\PYG{o}{*}\PYG{o}{*}\PYG{l+m+mi}{2} \PYG{o}{+} \PYG{n}{b}\PYG{o}{*}\PYG{n}{x} \PYG{o}{+} \PYG{n}{c}


\PYG{n}{plt}\PYG{o}{.}\PYG{n}{scatter}\PYG{p}{(}\PYG{n}{x}\PYG{p}{,} \PYG{n}{y}\PYG{p}{)}  \PYG{c+c1}{\PYGZsh{} 绘制散点图}
\PYG{n}{popt}\PYG{p}{,} \PYG{n}{pcov} \PYG{o}{=} \PYG{n}{curve\PYGZus{}fit}\PYG{p}{(}\PYG{n}{f2}\PYG{p}{,} \PYG{n}{x}\PYG{p}{,} \PYG{n}{y}\PYG{p}{)}    \PYG{c+c1}{\PYGZsh{} 执行非线性拟合}
\PYG{c+c1}{\PYGZsh{} popt数组中,三个值分别是待求参数a,b,c}
\PYG{n}{y1} \PYG{o}{=} \PYG{p}{[}\PYG{n}{f2}\PYG{p}{(}\PYG{n}{i}\PYG{p}{,} \PYG{n}{popt}\PYG{p}{[}\PYG{l+m+mi}{0}\PYG{p}{]}\PYG{p}{,} \PYG{n}{popt}\PYG{p}{[}\PYG{l+m+mi}{1}\PYG{p}{]}\PYG{p}{,} \PYG{n}{popt}\PYG{p}{[}\PYG{l+m+mi}{2}\PYG{p}{]}\PYG{p}{)} \PYG{k}{for} \PYG{n}{i} \PYG{o+ow}{in} \PYG{n}{x}\PYG{p}{]}   \PYG{c+c1}{\PYGZsh{} 计算得到拟合曲线上的一系列点}
\PYG{n}{plt}\PYG{o}{.}\PYG{n}{plot}\PYG{p}{(}\PYG{n}{x}\PYG{p}{,} \PYG{n}{y1}\PYG{p}{,} \PYG{l+s+s1}{\PYGZsq{}}\PYG{l+s+s1}{r}\PYG{l+s+s1}{\PYGZsq{}}\PYG{p}{)}   \PYG{c+c1}{\PYGZsh{} 绘制拟合曲线}
\end{sphinxVerbatim}

\begin{sphinxVerbatim}[commandchars=\\\{\}]
[\PYGZlt{}matplotlib.lines.Line2D at 0x12dd7b9d0\PYGZgt{}]
\end{sphinxVerbatim}

\noindent\sphinxincludegraphics{{prediction_model_29_1}.png}

可以看到,其效果并不好,增大次数试一下?

\begin{sphinxVerbatim}[commandchars=\\\{\}]
\PYG{c+c1}{\PYGZsh{} f3为三次多项式}
\PYG{k}{def} \PYG{n+nf}{f3}\PYG{p}{(}\PYG{n}{x}\PYG{p}{,} \PYG{n}{a}\PYG{p}{,} \PYG{n}{b}\PYG{p}{,} \PYG{n}{c}\PYG{p}{,}\PYG{n}{d}\PYG{p}{)}\PYG{p}{:}  
    \PYG{k}{return} \PYG{n}{a} \PYG{o}{*} \PYG{n}{x}\PYG{o}{*}\PYG{o}{*}\PYG{l+m+mi}{3} \PYG{o}{+} \PYG{n}{b}\PYG{o}{*}\PYG{n}{x}\PYG{o}{*}\PYG{o}{*}\PYG{l+m+mi}{2} \PYG{o}{+}\PYG{n}{c} \PYG{o}{*}\PYG{n}{x} \PYG{o}{+}\PYG{n}{d}
\PYG{c+c1}{\PYGZsh{} f4为四次多项式}
\PYG{k}{def} \PYG{n+nf}{f4}\PYG{p}{(}\PYG{n}{x}\PYG{p}{,} \PYG{n}{a}\PYG{p}{,} \PYG{n}{b}\PYG{p}{,} \PYG{n}{c}\PYG{p}{,}\PYG{n}{d}\PYG{p}{,}\PYG{n}{e}\PYG{p}{)}\PYG{p}{:}  
    \PYG{k}{return} \PYG{n}{a} \PYG{o}{*} \PYG{n}{x}\PYG{o}{*}\PYG{o}{*}\PYG{l+m+mi}{4} \PYG{o}{+} \PYG{n}{b}\PYG{o}{*}\PYG{n}{x}\PYG{o}{*}\PYG{o}{*}\PYG{l+m+mi}{3} \PYG{o}{+}\PYG{n}{c} \PYG{o}{*}\PYG{n}{x}\PYG{o}{*}\PYG{o}{*}\PYG{l+m+mi}{2} \PYG{o}{+}\PYG{n}{d}\PYG{o}{*}\PYG{n}{x} \PYG{o}{+} \PYG{n}{e}
\PYG{c+c1}{\PYGZsh{} f5为五次多项式}
\PYG{k}{def} \PYG{n+nf}{f5}\PYG{p}{(}\PYG{n}{x}\PYG{p}{,} \PYG{n}{a}\PYG{p}{,} \PYG{n}{b}\PYG{p}{,} \PYG{n}{c}\PYG{p}{,}\PYG{n}{d}\PYG{p}{,}\PYG{n}{e}\PYG{p}{,}\PYG{n}{f}\PYG{p}{)}\PYG{p}{:}  
    \PYG{k}{return} \PYG{n}{a} \PYG{o}{*} \PYG{n}{x}\PYG{o}{*}\PYG{o}{*}\PYG{l+m+mi}{5} \PYG{o}{+} \PYG{n}{b}\PYG{o}{*}\PYG{n}{x}\PYG{o}{*}\PYG{o}{*}\PYG{l+m+mi}{4} \PYG{o}{+}\PYG{n}{c} \PYG{o}{*}\PYG{n}{x}\PYG{o}{*}\PYG{o}{*}\PYG{l+m+mi}{3} \PYG{o}{+}\PYG{n}{d}\PYG{o}{*}\PYG{n}{x}\PYG{o}{*}\PYG{o}{*}\PYG{l+m+mi}{2} \PYG{o}{+} \PYG{n}{e}\PYG{o}{*}\PYG{n}{x} \PYG{o}{+}\PYG{n}{f}

\PYG{c+c1}{\PYGZsh{} 定义方差计算函数}
\PYG{k}{def} \PYG{n+nf}{error}\PYG{p}{(}\PYG{n}{y1}\PYG{p}{,}\PYG{n}{y2}\PYG{p}{)}\PYG{p}{:}
    \PYG{n}{a} \PYG{o}{=} \PYG{n}{np}\PYG{o}{.}\PYG{n}{array}\PYG{p}{(}\PYG{n}{y1}\PYG{p}{)}
    \PYG{n}{b} \PYG{o}{=} \PYG{n}{np}\PYG{o}{.}\PYG{n}{array}\PYG{p}{(}\PYG{n}{y2}\PYG{p}{)}
    \PYG{k}{return} \PYG{n}{np}\PYG{o}{.}\PYG{n}{dot}\PYG{p}{(}\PYG{n}{a}\PYG{o}{\PYGZhy{}}\PYG{n}{b}\PYG{p}{,}\PYG{n}{a}\PYG{o}{\PYGZhy{}}\PYG{n}{b}\PYG{p}{)}
    

\PYG{n}{plt}\PYG{o}{.}\PYG{n}{figure}\PYG{p}{(}\PYG{n}{figsize} \PYG{o}{=} \PYG{p}{(}\PYG{l+m+mi}{10}\PYG{p}{,}\PYG{l+m+mi}{10}\PYG{p}{)}\PYG{p}{)} \PYG{c+c1}{\PYGZsh{} 定义图片大小}
\PYG{n}{plt}\PYG{o}{.}\PYG{n}{subplot}\PYG{p}{(}\PYG{l+m+mi}{2}\PYG{p}{,}\PYG{l+m+mi}{2}\PYG{p}{,}\PYG{l+m+mi}{1}\PYG{p}{)} \PYG{c+c1}{\PYGZsh{} 开始绘制第1张子图}
\PYG{n}{plt}\PYG{o}{.}\PYG{n}{scatter}\PYG{p}{(}\PYG{n}{x}\PYG{p}{,} \PYG{n}{y}\PYG{p}{)}  \PYG{c+c1}{\PYGZsh{} 绘制(x,y)的散点图}
\PYG{n}{popt}\PYG{p}{,} \PYG{n}{pcov} \PYG{o}{=} \PYG{n}{curve\PYGZus{}fit}\PYG{p}{(}\PYG{n}{f2}\PYG{p}{,} \PYG{n}{x}\PYG{p}{,} \PYG{n}{y}\PYG{p}{)}    \PYG{c+c1}{\PYGZsh{} 执行2次多项式拟合}
\PYG{c+c1}{\PYGZsh{}popt数组中,三个值分别是待求参数a,b,c  }
\PYG{n}{y1} \PYG{o}{=} \PYG{p}{[}\PYG{n}{f2}\PYG{p}{(}\PYG{n}{i}\PYG{p}{,} \PYG{n}{popt}\PYG{p}{[}\PYG{l+m+mi}{0}\PYG{p}{]}\PYG{p}{,}\PYG{n}{popt}\PYG{p}{[}\PYG{l+m+mi}{1}\PYG{p}{]}\PYG{p}{,}\PYG{n}{popt}\PYG{p}{[}\PYG{l+m+mi}{2}\PYG{p}{]}\PYG{p}{)} \PYG{k}{for} \PYG{n}{i} \PYG{o+ow}{in} \PYG{n}{x}\PYG{p}{]}  \PYG{c+c1}{\PYGZsh{} 得到拟合曲线上的一系列点}
\PYG{n}{plt}\PYG{o}{.}\PYG{n}{plot}\PYG{p}{(}\PYG{n}{x}\PYG{p}{,}\PYG{n}{y1}\PYG{p}{,}\PYG{l+s+s1}{\PYGZsq{}}\PYG{l+s+s1}{r\PYGZhy{}\PYGZhy{}}\PYG{l+s+s1}{\PYGZsq{}}\PYG{p}{)}   \PYG{c+c1}{\PYGZsh{} 绘制拟合曲线}
\PYG{n}{plt}\PYG{o}{.}\PYG{n}{title}\PYG{p}{(}\PYG{n+nb}{str}\PYG{p}{(}\PYG{n}{error}\PYG{p}{(}\PYG{n}{y}\PYG{p}{,}\PYG{n}{y1}\PYG{p}{)}\PYG{p}{)}\PYG{p}{)} \PYG{c+c1}{\PYGZsh{} 计算方差,并作为图片的标题}

\PYG{n}{plt}\PYG{o}{.}\PYG{n}{subplot}\PYG{p}{(}\PYG{l+m+mi}{2}\PYG{p}{,}\PYG{l+m+mi}{2}\PYG{p}{,}\PYG{l+m+mi}{2}\PYG{p}{)} \PYG{c+c1}{\PYGZsh{} 开始绘制第2张子图}
\PYG{n}{plt}\PYG{o}{.}\PYG{n}{scatter}\PYG{p}{(}\PYG{n}{x}\PYG{p}{,} \PYG{n}{y}\PYG{p}{)} \PYG{c+c1}{\PYGZsh{} 绘制(x,y)的散点图}
\PYG{n}{popt}\PYG{p}{,} \PYG{n}{pcov} \PYG{o}{=} \PYG{n}{curve\PYGZus{}fit}\PYG{p}{(}\PYG{n}{f3}\PYG{p}{,} \PYG{n}{x}\PYG{p}{,} \PYG{n}{y}\PYG{p}{)}   \PYG{c+c1}{\PYGZsh{} 执行3次多项式拟合}
\PYG{c+c1}{\PYGZsh{}popt数组中,三个值分别是待求参数a,b,c,d  }
\PYG{n}{y1} \PYG{o}{=} \PYG{p}{[}\PYG{n}{f3}\PYG{p}{(}\PYG{n}{i}\PYG{p}{,} \PYG{n}{popt}\PYG{p}{[}\PYG{l+m+mi}{0}\PYG{p}{]}\PYG{p}{,}\PYG{n}{popt}\PYG{p}{[}\PYG{l+m+mi}{1}\PYG{p}{]}\PYG{p}{,}\PYG{n}{popt}\PYG{p}{[}\PYG{l+m+mi}{2}\PYG{p}{]}\PYG{p}{,}\PYG{n}{popt}\PYG{p}{[}\PYG{l+m+mi}{3}\PYG{p}{]}\PYG{p}{)} \PYG{k}{for} \PYG{n}{i} \PYG{o+ow}{in} \PYG{n}{x}\PYG{p}{]}    \PYG{c+c1}{\PYGZsh{} 得到拟合曲线上的一系列点}
\PYG{n}{plt}\PYG{o}{.}\PYG{n}{plot}\PYG{p}{(}\PYG{n}{x}\PYG{p}{,}\PYG{n}{y1}\PYG{p}{,}\PYG{l+s+s1}{\PYGZsq{}}\PYG{l+s+s1}{r\PYGZhy{}\PYGZhy{}}\PYG{l+s+s1}{\PYGZsq{}}\PYG{p}{)}   \PYG{c+c1}{\PYGZsh{} 绘制拟合曲线}
\PYG{n}{plt}\PYG{o}{.}\PYG{n}{title}\PYG{p}{(}\PYG{n+nb}{str}\PYG{p}{(}\PYG{n}{error}\PYG{p}{(}\PYG{n}{y}\PYG{p}{,}\PYG{n}{y1}\PYG{p}{)}\PYG{p}{)}\PYG{p}{)} \PYG{c+c1}{\PYGZsh{} 计算方差,并作为图片的标题}


\PYG{n}{plt}\PYG{o}{.}\PYG{n}{subplot}\PYG{p}{(}\PYG{l+m+mi}{2}\PYG{p}{,}\PYG{l+m+mi}{2}\PYG{p}{,}\PYG{l+m+mi}{3}\PYG{p}{)} \PYG{c+c1}{\PYGZsh{} 开始绘制第3张子图}
\PYG{n}{plt}\PYG{o}{.}\PYG{n}{scatter}\PYG{p}{(}\PYG{n}{x}\PYG{p}{,} \PYG{n}{y}\PYG{p}{)}
\PYG{n}{popt}\PYG{p}{,} \PYG{n}{pcov} \PYG{o}{=} \PYG{n}{curve\PYGZus{}fit}\PYG{p}{(}\PYG{n}{f4}\PYG{p}{,} \PYG{n}{x}\PYG{p}{,} \PYG{n}{y}\PYG{p}{)}  
\PYG{n}{y1} \PYG{o}{=} \PYG{p}{[}\PYG{n}{f4}\PYG{p}{(}\PYG{n}{i}\PYG{p}{,} \PYG{n}{popt}\PYG{p}{[}\PYG{l+m+mi}{0}\PYG{p}{]}\PYG{p}{,}\PYG{n}{popt}\PYG{p}{[}\PYG{l+m+mi}{1}\PYG{p}{]}\PYG{p}{,}\PYG{n}{popt}\PYG{p}{[}\PYG{l+m+mi}{2}\PYG{p}{]}\PYG{p}{,}\PYG{n}{popt}\PYG{p}{[}\PYG{l+m+mi}{3}\PYG{p}{]}\PYG{p}{,}\PYG{n}{popt}\PYG{p}{[}\PYG{l+m+mi}{4}\PYG{p}{]}\PYG{p}{)} \PYG{k}{for} \PYG{n}{i} \PYG{o+ow}{in} \PYG{n}{x}\PYG{p}{]}  
\PYG{n}{plt}\PYG{o}{.}\PYG{n}{plot}\PYG{p}{(}\PYG{n}{x}\PYG{p}{,}\PYG{n}{y1}\PYG{p}{,}\PYG{l+s+s1}{\PYGZsq{}}\PYG{l+s+s1}{r\PYGZhy{}\PYGZhy{}}\PYG{l+s+s1}{\PYGZsq{}}\PYG{p}{)}  
\PYG{n}{plt}\PYG{o}{.}\PYG{n}{title}\PYG{p}{(}\PYG{n+nb}{str}\PYG{p}{(}\PYG{n}{error}\PYG{p}{(}\PYG{n}{y}\PYG{p}{,}\PYG{n}{y1}\PYG{p}{)}\PYG{p}{)}\PYG{p}{)}

\PYG{n}{plt}\PYG{o}{.}\PYG{n}{subplot}\PYG{p}{(}\PYG{l+m+mi}{2}\PYG{p}{,}\PYG{l+m+mi}{2}\PYG{p}{,}\PYG{l+m+mi}{4}\PYG{p}{)} \PYG{c+c1}{\PYGZsh{} 开始绘制第4张子图}
\PYG{n}{plt}\PYG{o}{.}\PYG{n}{scatter}\PYG{p}{(}\PYG{n}{x}\PYG{p}{,} \PYG{n}{y}\PYG{p}{)}
\PYG{n}{popt}\PYG{p}{,} \PYG{n}{pcov} \PYG{o}{=} \PYG{n}{curve\PYGZus{}fit}\PYG{p}{(}\PYG{n}{f5}\PYG{p}{,} \PYG{n}{x}\PYG{p}{,} \PYG{n}{y}\PYG{p}{)}  
\PYG{n}{y1} \PYG{o}{=} \PYG{p}{[}\PYG{n}{f5}\PYG{p}{(}\PYG{n}{i}\PYG{p}{,} \PYG{n}{popt}\PYG{p}{[}\PYG{l+m+mi}{0}\PYG{p}{]}\PYG{p}{,}\PYG{n}{popt}\PYG{p}{[}\PYG{l+m+mi}{1}\PYG{p}{]}\PYG{p}{,}\PYG{n}{popt}\PYG{p}{[}\PYG{l+m+mi}{2}\PYG{p}{]}\PYG{p}{,}\PYG{n}{popt}\PYG{p}{[}\PYG{l+m+mi}{3}\PYG{p}{]}\PYG{p}{,}\PYG{n}{popt}\PYG{p}{[}\PYG{l+m+mi}{4}\PYG{p}{]}\PYG{p}{,}\PYG{n}{popt}\PYG{p}{[}\PYG{l+m+mi}{5}\PYG{p}{]}\PYG{p}{)} \PYG{k}{for} \PYG{n}{i} \PYG{o+ow}{in} \PYG{n}{x}\PYG{p}{]}  
\PYG{n}{plt}\PYG{o}{.}\PYG{n}{plot}\PYG{p}{(}\PYG{n}{x}\PYG{p}{,}\PYG{n}{y1}\PYG{p}{,}\PYG{l+s+s1}{\PYGZsq{}}\PYG{l+s+s1}{r\PYGZhy{}\PYGZhy{}}\PYG{l+s+s1}{\PYGZsq{}}\PYG{p}{)}  
\PYG{n}{plt}\PYG{o}{.}\PYG{n}{title}\PYG{p}{(}\PYG{n+nb}{str}\PYG{p}{(}\PYG{n}{error}\PYG{p}{(}\PYG{n}{y}\PYG{p}{,}\PYG{n}{y1}\PYG{p}{)}\PYG{p}{)}\PYG{p}{)}
\end{sphinxVerbatim}

\begin{sphinxVerbatim}[commandchars=\\\{\}]
Text(0.5, 1.0, \PYGZsq{}38.052046270695655\PYGZsq{})
\end{sphinxVerbatim}

\noindent\sphinxincludegraphics{{prediction_model_31_1}.png}

**思考:**一定是越高的阶次越好吗?

\sphinxincludegraphics{{9605}.png}

注意:兼顾拟合误差和预测误差。防止\sphinxstylestrong{过拟合(overfit)\sphinxstylestrong{和}欠拟合(underfit)}。


\subsection{一元其他非线性函数拟合}
\label{\detokenize{docs/prediction_model:id9}}
这里我们举一个指数拟合的例子与二次函数拟合进行对比。实际上,使用本课程讲到的方法,可以进行任意形式的函数拟合。
\begin{equation*}
\begin{split}
f_1(x) = x^a + b
\end{split}
\end{equation*}\begin{equation*}
\begin{split}
f_2(x) = ax^2 + b
\end{split}
\end{equation*}
\begin{sphinxVerbatim}[commandchars=\\\{\}]
\PYG{c+c1}{\PYGZsh{} 定义f1}
\PYG{k}{def} \PYG{n+nf}{f1}\PYG{p}{(}\PYG{n}{x}\PYG{p}{,} \PYG{n}{a}\PYG{p}{,} \PYG{n}{b}\PYG{p}{)}\PYG{p}{:}  
    \PYG{k}{return} \PYG{n}{x}\PYG{o}{*}\PYG{o}{*}\PYG{n}{a} \PYG{o}{+} \PYG{n}{b}  
\PYG{c+c1}{\PYGZsh{} 定义f2}
\PYG{k}{def} \PYG{n+nf}{f2}\PYG{p}{(}\PYG{n}{x}\PYG{p}{,}\PYG{n}{a}\PYG{p}{,}\PYG{n}{b}\PYG{p}{)}\PYG{p}{:}
    \PYG{k}{return} \PYG{n}{a}\PYG{o}{*}\PYG{n}{x}\PYG{o}{*}\PYG{o}{*}\PYG{l+m+mi}{2} \PYG{o}{+} \PYG{n}{b}
\PYG{c+c1}{\PYGZsh{} 随机生成数据并绘制折线图}
\PYG{n}{xdata} \PYG{o}{=} \PYG{n}{np}\PYG{o}{.}\PYG{n}{linspace}\PYG{p}{(}\PYG{l+m+mi}{0}\PYG{p}{,} \PYG{l+m+mi}{4}\PYG{p}{,} \PYG{l+m+mi}{50}\PYG{p}{)}  
\PYG{n}{y} \PYG{o}{=} \PYG{n}{f1}\PYG{p}{(}\PYG{n}{xdata}\PYG{p}{,} \PYG{l+m+mf}{2.5}\PYG{p}{,} \PYG{l+m+mf}{1.3}\PYG{p}{)}  
\PYG{n}{ydata} \PYG{o}{=} \PYG{n}{y} \PYG{o}{+} \PYG{l+m+mi}{4} \PYG{o}{*} \PYG{n}{np}\PYG{o}{.}\PYG{n}{random}\PYG{o}{.}\PYG{n}{normal}\PYG{p}{(}\PYG{n}{size}\PYG{o}{=}\PYG{n+nb}{len}\PYG{p}{(}\PYG{n}{xdata}\PYG{p}{)}\PYG{p}{)}  
\PYG{n}{plt}\PYG{o}{.}\PYG{n}{plot}\PYG{p}{(}\PYG{n}{xdata}\PYG{p}{,}\PYG{n}{ydata}\PYG{p}{,}\PYG{l+s+s1}{\PYGZsq{}}\PYG{l+s+s1}{b\PYGZhy{}}\PYG{l+s+s1}{\PYGZsq{}}\PYG{p}{)} 

\PYG{c+c1}{\PYGZsh{} 开始拟合}
\PYG{n}{popt1}\PYG{p}{,} \PYG{n}{pcov1} \PYG{o}{=} \PYG{n}{curve\PYGZus{}fit}\PYG{p}{(}\PYG{n}{f1}\PYG{p}{,} \PYG{n}{xdata}\PYG{p}{,} \PYG{n}{ydata}\PYG{p}{)}   \PYG{c+c1}{\PYGZsh{} 使用f1进行拟合}
\PYG{n}{popt2}\PYG{p}{,} \PYG{n}{pcov2} \PYG{o}{=} \PYG{n}{curve\PYGZus{}fit}\PYG{p}{(}\PYG{n}{f2}\PYG{p}{,} \PYG{n}{xdata}\PYG{p}{,} \PYG{n}{ydata}\PYG{p}{)}    \PYG{c+c1}{\PYGZsh{} 使用f2进行拟合}

\PYG{n}{y1} \PYG{o}{=} \PYG{p}{[}\PYG{n}{f1}\PYG{p}{(}\PYG{n}{i}\PYG{p}{,} \PYG{n}{popt1}\PYG{p}{[}\PYG{l+m+mi}{0}\PYG{p}{]}\PYG{p}{,}\PYG{n}{popt1}\PYG{p}{[}\PYG{l+m+mi}{1}\PYG{p}{]}\PYG{p}{)} \PYG{k}{for} \PYG{n}{i} \PYG{o+ow}{in} \PYG{n}{xdata}\PYG{p}{]}
\PYG{n}{y2} \PYG{o}{=} \PYG{p}{[}\PYG{n}{f2}\PYG{p}{(}\PYG{n}{i}\PYG{p}{,} \PYG{n}{popt2}\PYG{p}{[}\PYG{l+m+mi}{0}\PYG{p}{]}\PYG{p}{,}\PYG{n}{popt2}\PYG{p}{[}\PYG{l+m+mi}{1}\PYG{p}{]}\PYG{p}{)} \PYG{k}{for} \PYG{n}{i} \PYG{o+ow}{in} \PYG{n}{xdata}\PYG{p}{]}
\PYG{n}{plt}\PYG{o}{.}\PYG{n}{plot}\PYG{p}{(}\PYG{n}{xdata}\PYG{p}{,}\PYG{n}{y1}\PYG{p}{,}\PYG{l+s+s1}{\PYGZsq{}}\PYG{l+s+s1}{\PYGZhy{}\PYGZhy{}}\PYG{l+s+s1}{\PYGZsq{}}\PYG{p}{,}\PYG{n}{label} \PYG{o}{=} \PYG{l+s+s1}{\PYGZsq{}}\PYG{l+s+s1}{exp}\PYG{l+s+s1}{\PYGZsq{}}\PYG{p}{)}  
\PYG{n}{plt}\PYG{o}{.}\PYG{n}{plot}\PYG{p}{(}\PYG{n}{xdata}\PYG{p}{,}\PYG{n}{y2}\PYG{p}{,}\PYG{l+s+s1}{\PYGZsq{}}\PYG{l+s+s1}{\PYGZhy{}\PYGZhy{}}\PYG{l+s+s1}{\PYGZsq{}}\PYG{p}{,}\PYG{n}{label} \PYG{o}{=} \PYG{l+s+s1}{\PYGZsq{}}\PYG{l+s+s1}{para}\PYG{l+s+s1}{\PYGZsq{}}\PYG{p}{)} 
\PYG{n}{plt}\PYG{o}{.}\PYG{n}{legend}\PYG{p}{(}\PYG{p}{)}
\end{sphinxVerbatim}

\begin{sphinxVerbatim}[commandchars=\\\{\}]
\PYGZlt{}matplotlib.legend.Legend at 0x12e2fc610\PYGZgt{}
\end{sphinxVerbatim}

\noindent\sphinxincludegraphics{{prediction_model_37_1}.png}


\section{时间序列方法}
\label{\detokenize{docs/prediction_model:id10}}

\subsection{移动平均法}
\label{\detokenize{docs/prediction_model:id11}}
移动平均法是根据时间序列资料逐渐推移,依次计算包含一定项数的时序平均数,以反映长期趋势的方法。当时间序列的数值由于受周期变动和不规则变动的影响,起伏较大,不易显示出发展趋势时,可用移动平均法,消除这些因素的影响,分析、预测序列的长期趋势。移动平均法有\sphinxstylestrong{简单移动平均法,加权移动平均法,趋势移动平均法}等。


\subsubsection{简单移动平均法}
\label{\detokenize{docs/prediction_model:id12}}
当预测目标的基本趋势是在某一水平上下波动时,可用一次简单移动平均方法建立预测模型。设观测序列为\(y_{1}, \cdots, y_{T}\),取移动平均的项数 \(N < T\),
\begin{equation*}
\begin{split}
\hat{y}_{t+1}=\frac{1}{N}\left(\hat{y}_{t}+\cdots+\hat{y}_{t-N+1}\right), t=N, N+1, \cdots,
\end{split}
\end{equation*}
即:最近\(N\)期序列值的平均值作为未来各期的预测结果。

一般\(N\)取值范围:\(5 \leq N \leq 200\)。当历史序列的基本趋势变化不大且序列中随机变动成分较多时,\(N\) 的取值应较大一些。否则\(N\)的取值应小一些。在有确定的季节变动周期的资料中,移动平均的项数应取周期长度。选择最佳\(N\)值的一个有效方法是,比较若干模型的预测误差。预测标准误差最小者为好。

**例:**某企业1\sphinxhyphen{}11月份的销售收入时间序列如下表所示。试用一次简单移动平均预测其12月份的销售收入。


\begin{savenotes}\sphinxattablestart
\centering
\begin{tabulary}{\linewidth}[t]{|T|T|T|T|T|T|T|}
\hline
\sphinxstyletheadfamily 
月份
&\sphinxstyletheadfamily 
1
&\sphinxstyletheadfamily 
2
&\sphinxstyletheadfamily 
3
&\sphinxstyletheadfamily 
4
&\sphinxstyletheadfamily 
5
&\sphinxstyletheadfamily 
6
\\
\hline
销售收入
&
533.8
&
574.6
&
606.9
&
649.8
&
705.1
&
772.0
\\
\hline
月份
&
7
&
8
&
9
&
10
&
11
&

\\
\hline
销售收入
&
816.4
&
892.7
&
963.9
&
1015.1
&
1102.7
&

\\
\hline
\end{tabulary}
\par
\sphinxattableend\end{savenotes}

\sphinxstylestrong{解:} 分别取\(N=4,N=5\)的预测公式
\begin{equation*}
\begin{split}
\begin{aligned}
&\hat{y}_{t+1}^{(1)}=\frac{y_{t}+y_{t-1}+y_{t-2}+y_{t-3}}{4}, \quad t=4,5, \cdots, 11\\
&\hat{y}_{t+1}^{(1)}=\frac{y_{t}+y_{t-1}+y_{t-2}+y_{t-3}+y_{t-4}}{5}, \quad t=5, \cdots, 11
\end{aligned}
\end{split}
\end{equation*}
当\(N=4\)时,预测值\(\hat{y}_{12}^{(1)}=993.6\)预测的标准误差为
\$\(
S_{1}=\sqrt{\frac{\sum_{i=5}^{11}\left(\hat{y}_{t}^{(1)}-y_{t}\right)^{2}}{11-4}}=150.5
\)\$

当\(N=5\)时,预测值\(\hat{y}_{12}^{(1)}=958.2\)预测的标准误差为
\$\(
S_{2}=\sqrt{\frac{\sum_{t=6}^{11}\left(\hat{y}_{t}^{(1)}-y_{t}\right)^{2}}{11-5}}=182.4
\)\$

计算结果表明,\(N = 4\)时,预测的标准误差较小,所以选取\(N = 4\)。预测第 12 月份的销售收入为993.6。

\begin{sphinxVerbatim}[commandchars=\\\{\}]
\PYG{n}{x} \PYG{o}{=} \PYG{p}{[}\PYG{l+m+mi}{1}\PYG{p}{,} \PYG{l+m+mi}{2}\PYG{p}{,} \PYG{l+m+mi}{3}\PYG{p}{,} \PYG{l+m+mi}{4}\PYG{p}{,} \PYG{l+m+mi}{5}\PYG{p}{,} \PYG{l+m+mi}{6}\PYG{p}{,} \PYG{l+m+mi}{7}\PYG{p}{,} \PYG{l+m+mi}{8}\PYG{p}{,} \PYG{l+m+mi}{9}\PYG{p}{,} \PYG{l+m+mi}{10}\PYG{p}{,} \PYG{l+m+mi}{11}\PYG{p}{]}  \PYG{c+c1}{\PYGZsh{} 输入x值}
\PYG{n}{y} \PYG{o}{=} \PYG{p}{[}
    \PYG{l+m+mf}{533.8}\PYG{p}{,}
    \PYG{l+m+mf}{574.6}\PYG{p}{,}
    \PYG{l+m+mf}{606.9}\PYG{p}{,}
    \PYG{l+m+mf}{649.8}\PYG{p}{,}
    \PYG{l+m+mf}{705.1}\PYG{p}{,}
    \PYG{l+m+mf}{772.0}\PYG{p}{,}
    \PYG{l+m+mf}{816.4}\PYG{p}{,}
    \PYG{l+m+mf}{892.7}\PYG{p}{,}
    \PYG{l+m+mf}{963.9}\PYG{p}{,}
    \PYG{l+m+mf}{1015.1}\PYG{p}{,}
    \PYG{l+m+mf}{1102.7}\PYG{p}{]}  \PYG{c+c1}{\PYGZsh{} 输入y值}
\PYG{n}{plt}\PYG{o}{.}\PYG{n}{scatter}\PYG{p}{(}\PYG{n}{x}\PYG{p}{,} \PYG{n}{y}\PYG{p}{)}  \PYG{c+c1}{\PYGZsh{} 绘制散点图}
\PYG{n}{N} \PYG{o}{=} \PYG{l+m+mi}{4}  \PYG{c+c1}{\PYGZsh{} 选择N=4}
\PYG{n}{y12} \PYG{o}{=} \PYG{n}{np}\PYG{o}{.}\PYG{n}{mean}\PYG{p}{(}\PYG{n}{y}\PYG{p}{[}\PYG{o}{\PYGZhy{}}\PYG{l+m+mi}{4}\PYG{p}{:}\PYG{p}{]}\PYG{p}{)}  \PYG{c+c1}{\PYGZsh{} 进行移动平均预测}
\PYG{n}{plt}\PYG{o}{.}\PYG{n}{scatter}\PYG{p}{(}\PYG{l+m+mi}{12}\PYG{p}{,} \PYG{n}{y12}\PYG{p}{)}  \PYG{c+c1}{\PYGZsh{} 绘制移动平均结果}
\end{sphinxVerbatim}

\begin{sphinxVerbatim}[commandchars=\\\{\}]
\PYGZlt{}matplotlib.collections.PathCollection at 0x12e34c4d0\PYGZgt{}
\end{sphinxVerbatim}

\noindent\sphinxincludegraphics{{prediction_model_44_1}.png}

简单移动平均法\sphinxstylestrong{只适合做近期预测},而且是预测目标的\sphinxstylestrong{发展趋势变化不大}的情况。如果目标的发展趋势存在其它的变化,采用简单移动平均法就会产生较大的预测偏差和滞后。


\subsubsection{加权移动平均法}
\label{\detokenize{docs/prediction_model:id13}}
在简单移动平均公式中,每期数据在求平均时的作用是等同的。但是,每期数据所包含的信息量不一样,近期数据包含着更多关于未来情况的信心。因此,把各期数据等同看待是不尽合理的,应考虑各期数据的重要性,对近期数据给予较大的权重,这就是加权移动平均法的基本思想。

设观测序列为\(y_{1}, \cdots, y_{T}\),加权移动平均值计算公式为:
\$\(
\hat{y}_{t+1}=\frac{w_{1} y_{t}+w_{2} y_{t-1}+\cdots+w_{N} y_{t-N+1}}{w_{1}+w_{2}+\cdots+w_{N}}, t \geq N
\)\$

式中,\(\hat{y}_{t+1}\)为\(t+1\)期的预测值;\(w_i\)为\(y_{t-i+1}\)的权数,它体现了相应的\(y_t\)在加权平均数中的重要性。

\begin{sphinxVerbatim}[commandchars=\\\{\}]
\PYG{n}{x} \PYG{o}{=} \PYG{p}{[}\PYG{l+m+mi}{1}\PYG{p}{,} \PYG{l+m+mi}{2}\PYG{p}{,} \PYG{l+m+mi}{3}\PYG{p}{,} \PYG{l+m+mi}{4}\PYG{p}{,} \PYG{l+m+mi}{5}\PYG{p}{,} \PYG{l+m+mi}{6}\PYG{p}{,} \PYG{l+m+mi}{7}\PYG{p}{,} \PYG{l+m+mi}{8}\PYG{p}{,} \PYG{l+m+mi}{9}\PYG{p}{,} \PYG{l+m+mi}{10}\PYG{p}{,} \PYG{l+m+mi}{11}\PYG{p}{]}  \PYG{c+c1}{\PYGZsh{} 输入x值}
\PYG{n}{y} \PYG{o}{=} \PYG{p}{[}
    \PYG{l+m+mf}{533.8}\PYG{p}{,}
    \PYG{l+m+mf}{574.6}\PYG{p}{,}
    \PYG{l+m+mf}{606.9}\PYG{p}{,}
    \PYG{l+m+mf}{649.8}\PYG{p}{,}
    \PYG{l+m+mf}{705.1}\PYG{p}{,}
    \PYG{l+m+mf}{772.0}\PYG{p}{,}
    \PYG{l+m+mf}{816.4}\PYG{p}{,}
    \PYG{l+m+mf}{892.7}\PYG{p}{,}
    \PYG{l+m+mf}{963.9}\PYG{p}{,}
    \PYG{l+m+mf}{1015.1}\PYG{p}{,}
    \PYG{l+m+mf}{1102.7}\PYG{p}{]}  \PYG{c+c1}{\PYGZsh{} 输入y值}
\PYG{n}{plt}\PYG{o}{.}\PYG{n}{scatter}\PYG{p}{(}\PYG{n}{x}\PYG{p}{,} \PYG{n}{y}\PYG{p}{)}  \PYG{c+c1}{\PYGZsh{} 绘制散点图}
\PYG{n}{N} \PYG{o}{=} \PYG{l+m+mi}{4}  \PYG{c+c1}{\PYGZsh{} 选择移动平均项数}
\PYG{n}{w} \PYG{o}{=} \PYG{p}{[}\PYG{l+m+mf}{0.1}\PYG{p}{,} \PYG{l+m+mf}{0.2}\PYG{p}{,} \PYG{l+m+mf}{0.3}\PYG{p}{,} \PYG{l+m+mf}{0.4}\PYG{p}{]}  \PYG{c+c1}{\PYGZsh{} 指定权重}
\PYG{n}{y12} \PYG{o}{=} \PYG{n}{np}\PYG{o}{.}\PYG{n}{mean}\PYG{p}{(}\PYG{n}{y}\PYG{p}{[}\PYG{o}{\PYGZhy{}}\PYG{l+m+mi}{4}\PYG{p}{:}\PYG{p}{]}\PYG{p}{)}  \PYG{c+c1}{\PYGZsh{} 简单移动平均预测12月份收入}
\PYG{n}{y12\PYGZus{}weighted} \PYG{o}{=} \PYG{n}{np}\PYG{o}{.}\PYG{n}{dot}\PYG{p}{(}\PYG{n}{y}\PYG{p}{[}\PYG{o}{\PYGZhy{}}\PYG{l+m+mi}{4}\PYG{p}{:}\PYG{p}{]}\PYG{p}{,} \PYG{n}{w}\PYG{p}{)}  \PYG{c+c1}{\PYGZsh{} 加权移动平均预测12月份收入}
\PYG{n}{plt}\PYG{o}{.}\PYG{n}{scatter}\PYG{p}{(}\PYG{l+m+mi}{12}\PYG{p}{,} \PYG{n}{y12}\PYG{p}{,} \PYG{n}{label}\PYG{o}{=}\PYG{l+s+s1}{\PYGZsq{}}\PYG{l+s+s1}{Simple Moving Average}\PYG{l+s+s1}{\PYGZsq{}}\PYG{p}{)}  \PYG{c+c1}{\PYGZsh{} 绘制简单移动平均结果}
\PYG{n}{plt}\PYG{o}{.}\PYG{n}{scatter}\PYG{p}{(}\PYG{l+m+mi}{12}\PYG{p}{,} \PYG{n}{y12\PYGZus{}weighted}\PYG{p}{,} \PYG{n}{label}\PYG{o}{=}\PYG{l+s+s1}{\PYGZsq{}}\PYG{l+s+s1}{Weighted Moving Average}\PYG{l+s+s1}{\PYGZsq{}}\PYG{p}{)}  \PYG{c+c1}{\PYGZsh{} 绘制加权移动平均结果}
\PYG{n}{plt}\PYG{o}{.}\PYG{n}{legend}\PYG{p}{(}\PYG{p}{)}  \PYG{c+c1}{\PYGZsh{} 加入图例}
\end{sphinxVerbatim}

\begin{sphinxVerbatim}[commandchars=\\\{\}]
\PYGZlt{}matplotlib.legend.Legend at 0x12e016a90\PYGZgt{}
\end{sphinxVerbatim}

\noindent\sphinxincludegraphics{{prediction_model_47_1}.png}

在加权移动平均法中, \(w_t\)的选择,同样具有一定的经验性。一般的原则是:近期数据的权数大,远期数据的权数小。至于大到什么程度和小到什么程度,则需要按照预测者对序列的了解和分析来确定。


\subsubsection{趋势移动平均法}
\label{\detokenize{docs/prediction_model:id14}}
下面讨论如何利用移动平均的滞后偏差建立直线趋势预测模型。设时间序列\(\left\{y_{t}\right\}\)从某时期开始具有直线趋势,且认为未来时期也按此直线趋势变化,则可设此直线趋势预测模型为
\$\(
\hat{y}_{t+T}=a_{t}+b_{t} T, \quad T=1,2, \cdots 
\tag{1}
\)\$

其中 \(t\) 为当前时期数;\(T\) 为由\(t\)至预测期的时期数; \(a_t\) 为截距; \(b_t\) 为斜率。两者又称为平滑系数。

现在,我们根据移动平均值来确定平滑系数。由模型(1)可知
\$\(
\begin{aligned}
&a_{t}=y_{t}\\
&y_{t-1}=y_{t}-b_{t}\\
&y_{t-2}=y_{t}-2 b_{t}\\
&y_{t-N+1}=y_{t}-(N-1) b_{t}
\end{aligned}
\)\$

所以
\$\(
\begin{aligned}
M_{t}^{(1)} &=\frac{y_{t}+y_{t-1}+\cdots+y_{t-N+1}}{N}=\frac{y_{t}+\left(y_{t}-b_{t}\right)+\cdots+\left[y_{t}-(N-1) b_{t}\right]}{N} \\
&=\frac{N y_{t}-[1+2+\cdots+(N-1)] b_{t}}{N}=y_{t}-\frac{N-1}{2} b_{t}
\end{aligned}
\)\$

因此
\$\(
y_{t}-M_{t}^{(1)}=\frac{N-1}{2} b_{t}
\tag{2}
\)\$

类似以上的推导,我们还可以得到
\$\(
y_{t-1}-M_{t-1}^{(1)}=\frac{N-1}{2} b_{t}
\)\$

所以
\begin{equation*}
\begin{split}
y_{t}-y_{t-1}=M_{t}^{(1)}-M_{t-1}^{(1)}=b_{t}
\end{split}
\end{equation*}
类似式(2)的推导,可得
\begin{equation*}
\begin{split}
M_{t}^{(1)}-M_{t}^{(2)}=\frac{N-1}{2} b_{t}
\tag{3}
\end{split}
\end{equation*}
于是,由式(2)和式(3)可得平滑系数的计算公式为
\$\(
\left\{\begin{array}{l}
{a_{t}=2 M_{t}^{(1)}-M_{t}^{(2)}} \\
{b_{t}=\dfrac{2}{N-1}\left(M_{t}^{(1)}-M_{t}^{(2)}\right)}
\end{array}\right.
\)\$

继续沿用上面的例子,进行趋势移动平均的计算

\begin{sphinxVerbatim}[commandchars=\\\{\}]
\PYG{c+c1}{\PYGZsh{} 趋势移动平均代码}
\PYG{k}{def} \PYG{n+nf}{moveingAverage}\PYG{p}{(}\PYG{n}{data}\PYG{p}{,}\PYG{n}{N}\PYG{p}{)}\PYG{p}{:}
    \PYG{n}{temp} \PYG{o}{=} \PYG{n}{data}\PYG{o}{.}\PYG{n}{copy}\PYG{p}{(}\PYG{p}{)}
    \PYG{k}{for} \PYG{n}{i} \PYG{o+ow}{in} \PYG{n+nb}{range}\PYG{p}{(}\PYG{n+nb}{len}\PYG{p}{(}\PYG{n}{temp}\PYG{p}{)}\PYG{p}{)}\PYG{p}{:}
        \PYG{k}{if} \PYG{n}{i} \PYG{o}{\PYGZgt{}}\PYG{o}{=}  \PYG{n}{N} \PYG{p}{:}
            \PYG{n}{temp}\PYG{p}{[}\PYG{n}{i}\PYG{p}{]} \PYG{o}{=} \PYG{n}{np}\PYG{o}{.}\PYG{n}{mean}\PYG{p}{(}\PYG{n}{data}\PYG{p}{[}\PYG{n}{i}\PYG{o}{\PYGZhy{}}\PYG{n}{N}\PYG{p}{:}\PYG{n}{i}\PYG{p}{]}\PYG{p}{)}
        \PYG{k}{else}\PYG{p}{:}
            \PYG{n}{temp}\PYG{p}{[}\PYG{n}{i}\PYG{p}{]} \PYG{o}{=} \PYG{l+m+mi}{0}
    \PYG{k}{return} \PYG{n}{temp}
\end{sphinxVerbatim}

\begin{sphinxVerbatim}[commandchars=\\\{\}]
\PYG{n}{x} \PYG{o}{=} \PYG{p}{[}\PYG{l+m+mi}{1}\PYG{p}{,}\PYG{l+m+mi}{2}\PYG{p}{,}\PYG{l+m+mi}{3}\PYG{p}{,}\PYG{l+m+mi}{4}\PYG{p}{,}\PYG{l+m+mi}{5}\PYG{p}{,}\PYG{l+m+mi}{6}\PYG{p}{,}\PYG{l+m+mi}{7}\PYG{p}{,}\PYG{l+m+mi}{8}\PYG{p}{,}\PYG{l+m+mi}{9}\PYG{p}{,}\PYG{l+m+mi}{10}\PYG{p}{,}\PYG{l+m+mi}{11}\PYG{p}{]}
\PYG{n}{y} \PYG{o}{=}  \PYG{p}{[}\PYG{l+m+mf}{533.8}\PYG{p}{,}\PYG{l+m+mf}{574.6}\PYG{p}{,}\PYG{l+m+mf}{606.9}\PYG{p}{,}\PYG{l+m+mf}{649.8}\PYG{p}{,}\PYG{l+m+mf}{705.1}\PYG{p}{,}\PYG{l+m+mf}{772.0} \PYG{p}{,}\PYG{l+m+mf}{816.4}\PYG{p}{,} \PYG{l+m+mf}{892.7}\PYG{p}{,}\PYG{l+m+mf}{963.9}\PYG{p}{,}\PYG{l+m+mf}{1015.1}\PYG{p}{,}\PYG{l+m+mf}{1102.7}\PYG{p}{]}
\PYG{n}{plt}\PYG{o}{.}\PYG{n}{scatter}\PYG{p}{(}\PYG{n}{x}\PYG{p}{,}\PYG{n}{y}\PYG{p}{)}
\PYG{n}{N} \PYG{o}{=} \PYG{l+m+mi}{4}
\PYG{n}{w} \PYG{o}{=} \PYG{p}{[}\PYG{l+m+mf}{0.1}\PYG{p}{,}\PYG{l+m+mf}{0.2}\PYG{p}{,}\PYG{l+m+mf}{0.3}\PYG{p}{,}\PYG{l+m+mf}{0.4}\PYG{p}{]}
\PYG{n}{y12} \PYG{o}{=} \PYG{n}{np}\PYG{o}{.}\PYG{n}{mean}\PYG{p}{(}\PYG{n}{y}\PYG{p}{[}\PYG{o}{\PYGZhy{}}\PYG{l+m+mi}{4}\PYG{p}{:}\PYG{p}{]}\PYG{p}{)}
\PYG{n}{y12\PYGZus{}weighted} \PYG{o}{=} \PYG{n}{np}\PYG{o}{.}\PYG{n}{dot}\PYG{p}{(}\PYG{n}{y}\PYG{p}{[}\PYG{o}{\PYGZhy{}}\PYG{l+m+mi}{4}\PYG{p}{:}\PYG{p}{]}\PYG{p}{,}\PYG{n}{w}\PYG{p}{)}

\PYG{c+c1}{\PYGZsh{} 趋势移动平均代码}
\PYG{n}{N} \PYG{o}{=} \PYG{l+m+mi}{4}
\PYG{n}{MA1} \PYG{o}{=} \PYG{n}{moveingAverage}\PYG{p}{(}\PYG{n}{y}\PYG{p}{,}\PYG{n}{N}\PYG{p}{)}
\PYG{n}{MA2} \PYG{o}{=} \PYG{n}{moveingAverage}\PYG{p}{(}\PYG{n}{MA1}\PYG{p}{,}\PYG{n}{N}\PYG{p}{)}
\PYG{n}{at} \PYG{o}{=} \PYG{l+m+mi}{2} \PYG{o}{*} \PYG{n}{MA1}\PYG{p}{[}\PYG{o}{\PYGZhy{}}\PYG{l+m+mi}{1}\PYG{p}{]} \PYG{o}{\PYGZhy{}} \PYG{n}{MA2}\PYG{p}{[}\PYG{o}{\PYGZhy{}}\PYG{l+m+mi}{1}\PYG{p}{]}
\PYG{n}{bt} \PYG{o}{=} \PYG{l+m+mi}{2}\PYG{o}{/}\PYG{p}{(}\PYG{n}{N}\PYG{o}{\PYGZhy{}}\PYG{l+m+mi}{1}\PYG{p}{)} \PYG{o}{*} \PYG{p}{(}\PYG{n}{MA1}\PYG{p}{[}\PYG{o}{\PYGZhy{}}\PYG{l+m+mi}{1}\PYG{p}{]} \PYG{o}{\PYGZhy{}} \PYG{n}{MA2}\PYG{p}{[}\PYG{o}{\PYGZhy{}}\PYG{l+m+mi}{1}\PYG{p}{]}\PYG{p}{)}

\PYG{n}{y\PYGZus{}hat} \PYG{o}{=} \PYG{n}{at} \PYG{o}{+} \PYG{n}{bt} \PYG{o}{*} \PYG{l+m+mi}{1}

\PYG{n}{plt}\PYG{o}{.}\PYG{n}{scatter}\PYG{p}{(}\PYG{l+m+mi}{12}\PYG{p}{,}\PYG{n}{y12}\PYG{p}{,}\PYG{n}{label} \PYG{o}{=} \PYG{l+s+s1}{\PYGZsq{}}\PYG{l+s+s1}{Simple Moving Average}\PYG{l+s+s1}{\PYGZsq{}}\PYG{p}{)}
\PYG{n}{plt}\PYG{o}{.}\PYG{n}{scatter}\PYG{p}{(}\PYG{l+m+mi}{12}\PYG{p}{,}\PYG{n}{y12\PYGZus{}weighted}\PYG{p}{,}\PYG{n}{label} \PYG{o}{=} \PYG{l+s+s1}{\PYGZsq{}}\PYG{l+s+s1}{Weighted Moving Average}\PYG{l+s+s1}{\PYGZsq{}}\PYG{p}{)}
\PYG{n}{plt}\PYG{o}{.}\PYG{n}{scatter}\PYG{p}{(}\PYG{l+m+mi}{12}\PYG{p}{,}\PYG{n}{y\PYGZus{}hat}\PYG{p}{,}\PYG{n}{label} \PYG{o}{=} \PYG{l+s+s1}{\PYGZsq{}}\PYG{l+s+s1}{Moving Average Concerning Trends}\PYG{l+s+s1}{\PYGZsq{}}\PYG{p}{)}
\PYG{n}{plt}\PYG{o}{.}\PYG{n}{legend}\PYG{p}{(}\PYG{p}{)}
\end{sphinxVerbatim}

\begin{sphinxVerbatim}[commandchars=\\\{\}]
\PYGZlt{}matplotlib.legend.Legend at 0x12de64bd0\PYGZgt{}
\end{sphinxVerbatim}

\noindent\sphinxincludegraphics{{prediction_model_53_1}.png}

趋势移动平均法对于同时存在直线趋势与周期波动的序列,是一种既能反映趋势变化,又可以有效地分离出来周期变动的方法。


\subsubsection{自适应滤波法}
\label{\detokenize{docs/prediction_model:id15}}
自适应滤波法与移动平均法、指数平滑法一样,也是以时间序列的历史观测值进行某种加权平均来预测的,它要寻找一组“最佳”的权数,其办法是先用一组给定的权数来计算一个预测值,然后计算预测误差,再根据预测误差调整权数以减少误差。这样反复进行,直至找出一组“最佳”权数,使误差减少到最低限度。由于这种调整权数的过程与通讯工程中的传输噪声过滤过程极为接近,故称为自适应滤波法。

自适应滤波法的基本预测公式为
\$\(
\hat{y}_{t+1}=w_{1} y_{t}+w_{2} y_{t-1}+\cdots+w_{N} y_{t-N+1}=\sum_{i=1}^{N} w_{i} y_{t-i+1}
\)\$

式中,\(\hat{y}_{t+1}\)为第\(t +1\)期的预测值,\(w_i\)为第\(t-i+1\)期的观测权数。其调整权数的公式为
\begin{equation*}
\begin{split}
w_{i}^{\prime}=w_{i}+2 k \cdot e_{i+1} y_{t-i+1}
\end{split}
\end{equation*}
式中,\(i=1,2, \cdots, N, t=N, N+1, \cdots, n\),n为序列数据的个数,\(w_i\) 为调整前的第\(i\)个权数,\(w_{i}^{\prime}\)为调整后的第\(i\)个权数,\(k\)为学习常数,\(e_{i+1}\)为第\(t +1\)期的预测误差。

上式表明:调整后的一组权数应等于旧的一组权数加上误差调整项,这个调整项包括预测误差、原观测值和学习常数等三个因素。学习常数\(k\)的大小决定权数调整的速度。

下面举一个简单的例子来说明此法的全过程。设有一个时间序列包括10 个观测值,如下表所示。试用自适应滤波法,以两个权数来求第11 期的预测值。


\begin{savenotes}\sphinxattablestart
\centering
\begin{tabulary}{\linewidth}[t]{|T|T|T|T|T|T|T|T|T|T|T|}
\hline
\sphinxstyletheadfamily 
时期
&\sphinxstyletheadfamily 
1
&\sphinxstyletheadfamily 
2
&\sphinxstyletheadfamily 
3
&\sphinxstyletheadfamily 
4
&\sphinxstyletheadfamily 
5
&\sphinxstyletheadfamily 
6
&\sphinxstyletheadfamily 
7
&\sphinxstyletheadfamily 
8
&\sphinxstyletheadfamily 
9
&\sphinxstyletheadfamily 
10
\\
\hline
观测值
&
0.1
&
0.2
&
0.3
&
0.4
&
0.5
&
0.6
&
0.7
&
0.8
&
0.9
&
1
\\
\hline
\end{tabulary}
\par
\sphinxattableend\end{savenotes}

本例中\(N = 2\)。取初始权数 \(w_1 = 0.5\), \(w_2 = 0.5\) ,并设\(k = 0.9\)。\(t\)的取值由\(N = 2\)开始,当\(t=2\)时:
\begin{enumerate}
\sphinxsetlistlabels{\arabic}{enumi}{enumii}{}{.}%
\item {} 
按预测公式,求第\(t +1= 3\)期的预测值。
\$\(
\hat{y}_{t+1}=\hat{y}_{3}=w_{1} y_{2}+w_{2} y_{1}=0.15
\)\$

\item {} 
计算预测误差。
\$\(
e_{t+1}=e_{3}=y_{3}-\hat{y}_{3}=0.3-0.15=0.15
\)\$

\item {} 
更新权重
\$\(
w_{i}^{\prime}=w_{i}+2 k \cdot e_{i+1} y_{t-i+1}
\)\(
展开就是,
\)\(
\begin{aligned}
&w_{1}^{\prime}=w_{1}+2 k e_{3} y_{2}=0.554\\
&w_{2}^{\prime}=w_{2}+2 k e_{3} y_{1}=0.527
\end{aligned}
\)\$

\end{enumerate}

依次类推,我们可以按照同样的规律从4\sphinxhyphen{}10继续调整权重,这时,第一轮调整就此结束。

把现有的新权数作为初始权数,重新开始\(t = 2\)的过程。这样反复进行下去,到预测误差(指新一轮的预测总误差)没有明显改进时,就认为获得了一组“最佳”权数,能实际用来预测第11 期的数值。本例在调整过程中,可使得误差降为零,而权数达到稳定不变,最后得到的“最佳”权数为
\$\(
w_{1}^{\prime}=2.0, \quad w_{2}^{\prime}=-1.0
\)\$

用“最佳”权数预测第 11 期的取值
\begin{equation*}
\begin{split}
\hat{y}_{11}=w_{1}^{\prime} y_{10}+w_{2}^{\prime} y_{9}=1.1
\end{split}
\end{equation*}
在实际应用中,权数调整计算工作量可能很大,必须借助于计算机才能实现。

\begin{sphinxVerbatim}[commandchars=\\\{\}]
\PYG{n}{yt}\PYG{o}{=}\PYG{p}{[}\PYG{l+m+mf}{0.1} \PYG{o}{*} \PYG{n}{i} \PYG{k}{for} \PYG{n}{i} \PYG{o+ow}{in} \PYG{n+nb}{range}\PYG{p}{(}\PYG{l+m+mi}{1}\PYG{p}{,}\PYG{l+m+mi}{11}\PYG{p}{)}\PYG{p}{]}
\PYG{n}{m}\PYG{o}{=}\PYG{n+nb}{len}\PYG{p}{(}\PYG{n}{yt}\PYG{p}{)}
\PYG{n}{k}\PYG{o}{=}\PYG{l+m+mf}{0.9}
\PYG{n}{N}\PYG{o}{=}\PYG{l+m+mi}{2}
\PYG{n}{Terr}\PYG{o}{=}\PYG{l+m+mi}{10000}
\PYG{n}{w}\PYG{o}{=}\PYG{n}{np}\PYG{o}{.}\PYG{n}{ones}\PYG{p}{(}\PYG{n}{N}\PYG{p}{)}\PYG{o}{/}\PYG{n}{N}
\PYG{k}{while} \PYG{n+nb}{abs}\PYG{p}{(}\PYG{n}{Terr}\PYG{p}{)}\PYG{o}{\PYGZgt{}}\PYG{l+m+mf}{0.00001}\PYG{p}{:}
    \PYG{n}{Terr}\PYG{o}{=}\PYG{p}{[}\PYG{p}{]}
    \PYG{k}{for} \PYG{n}{j} \PYG{o+ow}{in} \PYG{n+nb}{range}\PYG{p}{(}\PYG{n}{N}\PYG{o}{+}\PYG{l+m+mi}{1}\PYG{p}{,}\PYG{n}{m}\PYG{p}{)}\PYG{p}{:}
        \PYG{n}{yhat} \PYG{o}{=} \PYG{n}{np}\PYG{o}{.}\PYG{n}{dot}\PYG{p}{(}\PYG{n}{w}\PYG{p}{,}\PYG{n}{yt}\PYG{p}{[}\PYG{n}{j}\PYG{o}{\PYGZhy{}}\PYG{n}{N}\PYG{p}{:}\PYG{n}{j}\PYG{p}{]}\PYG{p}{)}
        \PYG{n}{err} \PYG{o}{=} \PYG{n}{yt}\PYG{p}{[}\PYG{n}{j}\PYG{p}{]} \PYG{o}{\PYGZhy{}} \PYG{n}{yhat}
        \PYG{n}{Terr}\PYG{o}{.}\PYG{n}{append}\PYG{p}{(}\PYG{n}{err}\PYG{p}{)}
        \PYG{n}{w} \PYG{o}{=} \PYG{n}{w} \PYG{o}{+} \PYG{l+m+mi}{2}\PYG{o}{*}\PYG{n}{k}\PYG{o}{*}\PYG{n}{err}\PYG{o}{*}\PYG{n}{np}\PYG{o}{.}\PYG{n}{array}\PYG{p}{(}\PYG{n}{yt}\PYG{p}{[}\PYG{n}{j}\PYG{o}{\PYGZhy{}}\PYG{n}{N}\PYG{p}{:}\PYG{n}{j}\PYG{p}{]}\PYG{p}{)}
    \PYG{n}{Terr} \PYG{o}{=} \PYG{n+nb}{max}\PYG{p}{(}\PYG{n}{Terr}\PYG{p}{)}
    \PYG{n+nb}{print}\PYG{p}{(}\PYG{n}{w}\PYG{p}{)}
\end{sphinxVerbatim}

\begin{sphinxVerbatim}[commandchars=\\\{\}]
[0.55791872 0.59662951]
[0.54979462 0.60536788]
[0.54108544 0.61321874]
[0.53241919 0.6210174 ]
[0.52380161 0.62877213]
[0.51523249 0.63648324]
[0.50671157 0.644151  ]
[0.49823855 0.65177563]
[0.48981319 0.65935739]
[0.48143521 0.66689651]
[0.47310434 0.67439323]
[0.46482032 0.6818478 ]
[0.45658288 0.68926044]
[0.44839177 0.6966314 ]
[0.44024672 0.70396091]
[0.43214748 0.71124921]
[0.42409378 0.71849651]
[0.41608537 0.72570306]
[0.40812199 0.73286909]
[0.4002034  0.73999481]
[0.39232934 0.74708047]
[0.38449956 0.75412628]
[0.37671381 0.76113246]
[0.36897184 0.76809925]
[0.36127341 0.77502686]
[0.35361827 0.78191551]
[0.34600618 0.78876543]
[0.3384369  0.79557682]
[0.33091018 0.80234991]
[0.32342579 0.80908491]
[0.31598349 0.81578203]
[0.30858304 0.8224415 ]
[0.30122421 0.82906351]
[0.29390676 0.83564829]
[0.28663046 0.84219604]
[0.27939508 0.84870696]
[0.27220038 0.85518127]
[0.26504615 0.86161917]
[0.25793214 0.86802087]
[0.25085815 0.87438657]
[0.24382393 0.88071647]
[0.23682927 0.88701078]
[0.22987395 0.89326969]
[0.22295773 0.8994934 ]
[0.21608042 0.90568211]
[0.20924177 0.91183602]
[0.20244159 0.91795533]
[0.19567964 0.92404022]
[0.18895572 0.93009089]
[0.18226961 0.93610754]
[0.17562111 0.94209035]
[0.16900999 0.94803952]
[0.16243604 0.95395523]
[0.15589907 0.95983768]
[0.14939886 0.96568704]
[0.1429352  0.97150352]
[0.13650789 0.97728728]
[0.13011672 0.98303852]
[0.1237615  0.98875741]
[0.11744201 0.99444415]
[0.11115806 1.0000989 ]
[0.10490945 1.00572186]
[0.09869598 1.0113132 ]
[0.09251745 1.01687309]
[0.08637366 1.02240171]
[0.08026442 1.02789925]
[0.07418954 1.03336587]
[0.06814882 1.03880175]
[0.06214208 1.04420706]
[0.05616911 1.04958197]
[0.05022973 1.05492666]
[0.04432375 1.06024129]
[0.03845098 1.06552604]
[0.03261124 1.07078107]
[0.02680434 1.07600654]
[0.02103009 1.08120263]
[0.01528831 1.0863695 ]
[0.00957883 1.09150731]
[0.00390145 1.09661623]
[\PYGZhy{}0.00174401  1.10169642]
[\PYGZhy{}0.00735771  1.10674804]
[\PYGZhy{}0.01293985  1.11177125]
[\PYGZhy{}0.01849059  1.11676622]
[\PYGZhy{}0.02401012  1.1217331 ]
[\PYGZhy{}0.02949861  1.12667204]
[\PYGZhy{}0.03495624  1.13158321]
[\PYGZhy{}0.04038318  1.13646676]
[\PYGZhy{}0.04577959  1.14132285]
[\PYGZhy{}0.05114566  1.14615164]
[\PYGZhy{}0.05648156  1.15095326]
[\PYGZhy{}0.06178745  1.15572789]
[\PYGZhy{}0.0670635   1.16047566]
[\PYGZhy{}0.07230988  1.16519674]
[\PYGZhy{}0.07752675  1.16989127]
[\PYGZhy{}0.08271429  1.17455939]
[\PYGZhy{}0.08787266  1.17920127]
[\PYGZhy{}0.09300202  1.18381704]
[\PYGZhy{}0.09810253  1.18840685]
[\PYGZhy{}0.10317437  1.19297086]
[\PYGZhy{}0.10821768  1.1975092 ]
[\PYGZhy{}0.11323263  1.20202201]
[\PYGZhy{}0.11821937  1.20650945]
[\PYGZhy{}0.12317808  1.21097166]
[\PYGZhy{}0.1281089   1.21540877]
[\PYGZhy{}0.13301199  1.21982093]
[\PYGZhy{}0.13788751  1.22420827]
[\PYGZhy{}0.14273561  1.22857095]
[\PYGZhy{}0.14755644  1.23290909]
[\PYGZhy{}0.15235017  1.23722283]
[\PYGZhy{}0.15711694  1.24151232]
[\PYGZhy{}0.1618569   1.24577769]
[\PYGZhy{}0.16657021  1.25001907]
[\PYGZhy{}0.17125701  1.25423659]
[\PYGZhy{}0.17591746  1.2584304 ]
[\PYGZhy{}0.1805517   1.26260063]
[\PYGZhy{}0.18515988  1.2667474 ]
[\PYGZhy{}0.18974214  1.27087086]
[\PYGZhy{}0.19429864  1.27497113]
[\PYGZhy{}0.19882951  1.27904833]
[\PYGZhy{}0.2033349   1.28310261]
[\PYGZhy{}0.20781496  1.2871341 ]
[\PYGZhy{}0.21226982  1.29114291]
[\PYGZhy{}0.21669963  1.29512917]
[\PYGZhy{}0.22110453  1.29909302]
[\PYGZhy{}0.22548466  1.30303458]
[\PYGZhy{}0.22984015  1.30695398]
[\PYGZhy{}0.23417116  1.31085133]
[\PYGZhy{}0.23847781  1.31472677]
[\PYGZhy{}0.24276023  1.31858041]
[\PYGZhy{}0.24701858  1.32241238]
[\PYGZhy{}0.25125298  1.3262228 ]
[\PYGZhy{}0.25546357  1.3300118 ]
[\PYGZhy{}0.25965048  1.33377948]
[\PYGZhy{}0.26381385  1.33752598]
[\PYGZhy{}0.2679538   1.34125141]
[\PYGZhy{}0.27207047  1.34495589]
[\PYGZhy{}0.27616399  1.34863954]
[\PYGZhy{}0.28023449  1.35230248]
[\PYGZhy{}0.2842821   1.35594481]
[\PYGZhy{}0.28830695  1.35956666]
[\PYGZhy{}0.29230916  1.36316815]
[\PYGZhy{}0.29628887  1.36674938]
[\PYGZhy{}0.3002462   1.37031047]
[\PYGZhy{}0.30418128  1.37385154]
[\PYGZhy{}0.30809422  1.3773727 ]
[\PYGZhy{}0.31198516  1.38087405]
[\PYGZhy{}0.31585422  1.38435571]
[\PYGZhy{}0.31970152  1.38781779]
[\PYGZhy{}0.32352719  1.39126041]
[\PYGZhy{}0.32733134  1.39468366]
[\PYGZhy{}0.3311141   1.39808767]
[\PYGZhy{}0.33487559  1.40147253]
[\PYGZhy{}0.33861593  1.40483835]
[\PYGZhy{}0.34233523  1.40818525]
[\PYGZhy{}0.34603362  1.41151333]
[\PYGZhy{}0.3497112   1.41482269]
[\PYGZhy{}0.35336811  1.41811345]
[\PYGZhy{}0.35700445  1.42138569]
[\PYGZhy{}0.36062035  1.42463954]
[\PYGZhy{}0.36421591  1.42787508]
[\PYGZhy{}0.36779125  1.43109244]
[\PYGZhy{}0.37134648  1.43429169]
[\PYGZhy{}0.37488172  1.43747296]
[\PYGZhy{}0.37839708  1.44063634]
[\PYGZhy{}0.38189267  1.44378193]
[\PYGZhy{}0.38536861  1.44690983]
[\PYGZhy{}0.388825    1.45002014]
[\PYGZhy{}0.39226195  1.45311296]
[\PYGZhy{}0.39567957  1.45618838]
[\PYGZhy{}0.39907797  1.45924651]
[\PYGZhy{}0.40245726  1.46228745]
[\PYGZhy{}0.40581755  1.46531128]
[\PYGZhy{}0.40915895  1.46831811]
[\PYGZhy{}0.41248155  1.47130803]
[\PYGZhy{}0.41578547  1.47428113]
[\PYGZhy{}0.4190708   1.47723752]
[\PYGZhy{}0.42233767  1.48017728]
[\PYGZhy{}0.42558616  1.48310051]
[\PYGZhy{}0.42881638  1.4860073 ]
[\PYGZhy{}0.43202844  1.48889775]
[\PYGZhy{}0.43522243  1.49177194]
[\PYGZhy{}0.43839847  1.49462996]
[\PYGZhy{}0.44155664  1.49747192]
[\PYGZhy{}0.44469705  1.50029789]
[\PYGZhy{}0.44781981  1.50310797]
[\PYGZhy{}0.450925    1.50590225]
[\PYGZhy{}0.45401273  1.50868082]
[\PYGZhy{}0.4570831   1.51144376]
[\PYGZhy{}0.4601362   1.51419116]
[\PYGZhy{}0.46317213  1.51692311]
[\PYGZhy{}0.46619099  1.5196397 ]
[\PYGZhy{}0.46919287  1.52234101]
[\PYGZhy{}0.47217787  1.52502713]
[\PYGZhy{}0.47514608  1.52769815]
[\PYGZhy{}0.4780976   1.53035414]
[\PYGZhy{}0.48103253  1.5329952 ]
[\PYGZhy{}0.48395095  1.53562141]
[\PYGZhy{}0.48685296  1.53823285]
[\PYGZhy{}0.48973865  1.5408296 ]
[\PYGZhy{}0.49260811  1.54341175]
[\PYGZhy{}0.49546143  1.54597938]
[\PYGZhy{}0.49829871  1.54853257]
[\PYGZhy{}0.50112003  1.55107141]
[\PYGZhy{}0.50392549  1.55359596]
[\PYGZhy{}0.50671517  1.55610632]
[\PYGZhy{}0.50948917  1.55860256]
[\PYGZhy{}0.51224756  1.56108477]
[\PYGZhy{}0.51499044  1.56355301]
[\PYGZhy{}0.5177179   1.56600738]
[\PYGZhy{}0.52043002  1.56844794]
[\PYGZhy{}0.52312689  1.57087478]
[\PYGZhy{}0.52580859  1.57328797]
[\PYGZhy{}0.52847521  1.57568759]
[\PYGZhy{}0.53112684  1.57807372]
[\PYGZhy{}0.53376355  1.58044642]
[\PYGZhy{}0.53638544  1.58280579]
[\PYGZhy{}0.53899258  1.58515189]
[\PYGZhy{}0.54158506  1.58748479]
[\PYGZhy{}0.54416296  1.58980457]
[\PYGZhy{}0.54672637  1.59211131]
[\PYGZhy{}0.54927536  1.59440508]
[\PYGZhy{}0.55181001  1.59668595]
[\PYGZhy{}0.55433042  1.59895399]
[\PYGZhy{}0.55683664  1.60120928]
[\PYGZhy{}0.55932878  1.60345188]
[\PYGZhy{}0.5618069   1.60568187]
[\PYGZhy{}0.56427108  1.60789933]
[\PYGZhy{}0.56672141  1.61010431]
[\PYGZhy{}0.56915796  1.61229689]
[\PYGZhy{}0.5715808   1.61447715]
[\PYGZhy{}0.57399002  1.61664514]
[\PYGZhy{}0.5763857   1.61880094]
[\PYGZhy{}0.5787679   1.62094462]
[\PYGZhy{}0.5811367   1.62307624]
[\PYGZhy{}0.58349218  1.62519588]
[\PYGZhy{}0.58583442  1.62730359]
[\PYGZhy{}0.58816349  1.62939946]
[\PYGZhy{}0.59047945  1.63148353]
[\PYGZhy{}0.5927824   1.63355589]
[\PYGZhy{}0.59507239  1.63561659]
[\PYGZhy{}0.59734951  1.63766571]
[\PYGZhy{}0.59961382  1.6397033 ]
[\PYGZhy{}0.60186539  1.64172943]
[\PYGZhy{}0.60410431  1.64374417]
[\PYGZhy{}0.60633063  1.64574758]
[\PYGZhy{}0.60854444  1.64773972]
[\PYGZhy{}0.61074579  1.64972066]
[\PYGZhy{}0.61293477  1.65169046]
[\PYGZhy{}0.61511143  1.65364919]
[\PYGZhy{}0.61727586  1.65559689]
[\PYGZhy{}0.61942811  1.65753365]
[\PYGZhy{}0.62156826  1.65945951]
[\PYGZhy{}0.62369638  1.66137455]
[\PYGZhy{}0.62581253  1.66327881]
[\PYGZhy{}0.62791678  1.66517237]
[\PYGZhy{}0.63000919  1.66705527]
[\PYGZhy{}0.63208984  1.66892759]
[\PYGZhy{}0.63415879  1.67078938]
[\PYGZhy{}0.6362161  1.6726407]
[\PYGZhy{}0.63826184  1.67448161]
[\PYGZhy{}0.64029608  1.67631217]
[\PYGZhy{}0.64231888  1.67813243]
[\PYGZhy{}0.64433031  1.67994246]
[\PYGZhy{}0.64633042  1.68174231]
[\PYGZhy{}0.64831928  1.68353203]
[\PYGZhy{}0.65029696  1.68531169]
[\PYGZhy{}0.65226352  1.68708135]
[\PYGZhy{}0.65421902  1.68884105]
[\PYGZhy{}0.65616353  1.69059085]
[\PYGZhy{}0.6580971   1.69233082]
[\PYGZhy{}0.66001979  1.694061  ]
[\PYGZhy{}0.66193168  1.69578145]
[\PYGZhy{}0.66383281  1.69749223]
[\PYGZhy{}0.66572325  1.69919339]
[\PYGZhy{}0.66760306  1.70088498]
[\PYGZhy{}0.66947229  1.70256705]
[\PYGZhy{}0.67133102  1.70423967]
[\PYGZhy{}0.6731793   1.70590288]
[\PYGZhy{}0.67501718  1.70755674]
[\PYGZhy{}0.67684472  1.7092013 ]
[\PYGZhy{}0.67866199  1.71083661]
[\PYGZhy{}0.68046904  1.71246273]
[\PYGZhy{}0.68226592  1.71407969]
[\PYGZhy{}0.68405271  1.71568757]
[\PYGZhy{}0.68582944  1.71728641]
[\PYGZhy{}0.68759618  1.71887625]
[\PYGZhy{}0.68935299  1.72045715]
[\PYGZhy{}0.69109991  1.72202916]
[\PYGZhy{}0.69283702  1.72359234]
[\PYGZhy{}0.69456435  1.72514672]
[\PYGZhy{}0.69628197  1.72669236]
[\PYGZhy{}0.69798994  1.72822931]
[\PYGZhy{}0.69968829  1.72975761]
[\PYGZhy{}0.7013771   1.73127733]
[\PYGZhy{}0.70305641  1.73278849]
[\PYGZhy{}0.70472628  1.73429116]
[\PYGZhy{}0.70638675  1.73578538]
[\PYGZhy{}0.70803789  1.73727119]
[\PYGZhy{}0.70967974  1.73874865]
[\PYGZhy{}0.71131236  1.7402178 ]
[\PYGZhy{}0.7129358   1.74167869]
[\PYGZhy{}0.71455011  1.74313136]
[\PYGZhy{}0.71615534  1.74457586]
[\PYGZhy{}0.71775154  1.74601225]
[\PYGZhy{}0.71933877  1.74744055]
[\PYGZhy{}0.72091707  1.74886082]
[\PYGZhy{}0.7224865  1.7502731]
[\PYGZhy{}0.7240471   1.75167745]
[\PYGZhy{}0.72559893  1.75307389]
[\PYGZhy{}0.72714202  1.75446248]
[\PYGZhy{}0.72867644  1.75584327]
[\PYGZhy{}0.73020224  1.75721629]
[\PYGZhy{}0.73171945  1.75858158]
[\PYGZhy{}0.73322813  1.7599392 ]
[\PYGZhy{}0.73472832  1.76128919]
[\PYGZhy{}0.73622008  1.76263158]
[\PYGZhy{}0.73770345  1.76396642]
[\PYGZhy{}0.73917848  1.76529376]
[\PYGZhy{}0.74064521  1.76661364]
[\PYGZhy{}0.74210369  1.76792609]
[\PYGZhy{}0.74355398  1.76923116]
[\PYGZhy{}0.74499611  1.77052889]
[\PYGZhy{}0.74643012  1.77181932]
[\PYGZhy{}0.74785608  1.7731025 ]
[\PYGZhy{}0.74927401  1.77437846]
[\PYGZhy{}0.75068397  1.77564725]
[\PYGZhy{}0.752086   1.7769089]
[\PYGZhy{}0.75348015  1.77816345]
[\PYGZhy{}0.75486646  1.77941095]
[\PYGZhy{}0.75624497  1.78065144]
[\PYGZhy{}0.75761573  1.78188495]
[\PYGZhy{}0.75897878  1.78311152]
[\PYGZhy{}0.76033417  1.7843312 ]
[\PYGZhy{}0.76168193  1.78554401]
[\PYGZhy{}0.76302211  1.78675001]
[\PYGZhy{}0.76435476  1.78794922]
[\PYGZhy{}0.76567992  1.78914169]
[\PYGZhy{}0.76699762  1.79032746]
[\PYGZhy{}0.76830791  1.79150655]
[\PYGZhy{}0.76961083  1.79267902]
[\PYGZhy{}0.77090643  1.79384489]
[\PYGZhy{}0.77219474  1.79500421]
[\PYGZhy{}0.77347581  1.796157  ]
[\PYGZhy{}0.77474967  1.79730332]
[\PYGZhy{}0.77601637  1.79844318]
[\PYGZhy{}0.77727594  1.79957664]
[\PYGZhy{}0.77852843  1.80070372]
[\PYGZhy{}0.77977388  1.80182447]
[\PYGZhy{}0.78101232  1.80293891]
[\PYGZhy{}0.7822438   1.80404709]
[\PYGZhy{}0.78346836  1.80514903]
[\PYGZhy{}0.78468603  1.80624478]
[\PYGZhy{}0.78589685  1.80733436]
[\PYGZhy{}0.78710086  1.80841782]
[\PYGZhy{}0.7882981   1.80949519]
[\PYGZhy{}0.78948861  1.81056649]
[\PYGZhy{}0.79067242  1.81163177]
[\PYGZhy{}0.79184958  1.81269106]
[\PYGZhy{}0.79302012  1.8137444 ]
[\PYGZhy{}0.79418407  1.81479181]
[\PYGZhy{}0.79534148  1.81583333]
[\PYGZhy{}0.79649238  1.81686899]
[\PYGZhy{}0.79763681  1.81789883]
[\PYGZhy{}0.7987748   1.81892288]
[\PYGZhy{}0.79990639  1.81994117]
[\PYGZhy{}0.80103162  1.82095373]
[\PYGZhy{}0.80215052  1.8219606 ]
[\PYGZhy{}0.80326313  1.82296181]
[\PYGZhy{}0.80436948  1.82395739]
[\PYGZhy{}0.80546961  1.82494736]
[\PYGZhy{}0.80656356  1.82593177]
[\PYGZhy{}0.80765135  1.82691065]
[\PYGZhy{}0.80873302  1.82788402]
[\PYGZhy{}0.80980862  1.82885192]
[\PYGZhy{}0.81087816  1.82981437]
[\PYGZhy{}0.81194169  1.83077141]
[\PYGZhy{}0.81299924  1.83172307]
[\PYGZhy{}0.81405084  1.83266938]
[\PYGZhy{}0.81509653  1.83361036]
[\PYGZhy{}0.81613633  1.83454606]
[\PYGZhy{}0.81717029  1.83547649]
[\PYGZhy{}0.81819844  1.83640169]
[\PYGZhy{}0.8192208   1.83732169]
[\PYGZhy{}0.82023741  1.83823651]
[\PYGZhy{}0.82124831  1.83914619]
[\PYGZhy{}0.82225352  1.84005075]
[\PYGZhy{}0.82325308  1.84095023]
[\PYGZhy{}0.82424702  1.84184465]
[\PYGZhy{}0.82523537  1.84273404]
[\PYGZhy{}0.82621816  1.84361842]
[\PYGZhy{}0.82719543  1.84449784]
[\PYGZhy{}0.8281672  1.8453723]
[\PYGZhy{}0.8291335   1.84624186]
[\PYGZhy{}0.83009437  1.84710652]
[\PYGZhy{}0.83104984  1.84796631]
[\PYGZhy{}0.83199993  1.84882128]
[\PYGZhy{}0.83294468  1.84967143]
[\PYGZhy{}0.83388412  1.85051681]
[\PYGZhy{}0.83481827  1.85135743]
[\PYGZhy{}0.83574717  1.85219332]
[\PYGZhy{}0.83667085  1.85302451]
[\PYGZhy{}0.83758933  1.85385103]
[\PYGZhy{}0.83850265  1.8546729 ]
[\PYGZhy{}0.83941084  1.85549015]
\end{sphinxVerbatim}

\begin{sphinxVerbatim}[commandchars=\\\{\}]
[\PYGZhy{}0.84031391  1.85630281]
[\PYGZhy{}0.84121191  1.85711089]
[\PYGZhy{}0.84210485  1.85791443]
[\PYGZhy{}0.84299278  1.85871345]
[\PYGZhy{}0.84387571  1.85950797]
[\PYGZhy{}0.84475368  1.86029803]
[\PYGZhy{}0.84562671  1.86108365]
[\PYGZhy{}0.84649483  1.86186485]
[\PYGZhy{}0.84735807  1.86264165]
[\PYGZhy{}0.84821645  1.86341409]
[\PYGZhy{}0.84907001  1.86418218]
[\PYGZhy{}0.84991876  1.86494595]
[\PYGZhy{}0.85076275  1.86570543]
[\PYGZhy{}0.85160199  1.86646064]
[\PYGZhy{}0.8524365  1.8672116]
[\PYGZhy{}0.85326633  1.86795834]
[\PYGZhy{}0.85409149  1.86870087]
[\PYGZhy{}0.85491201  1.86943924]
[\PYGZhy{}0.85572791  1.87017345]
[\PYGZhy{}0.85653923  1.87090353]
[\PYGZhy{}0.85734598  1.8716295 ]
[\PYGZhy{}0.8581482  1.8723514]
[\PYGZhy{}0.8589459   1.87306923]
[\PYGZhy{}0.85973912  1.87378303]
[\PYGZhy{}0.86052788  1.87449281]
[\PYGZhy{}0.8613122  1.8751986]
[\PYGZhy{}0.86209212  1.87590042]
[\PYGZhy{}0.86286764  1.8765983 ]
[\PYGZhy{}0.86363881  1.87729225]
[\PYGZhy{}0.86440564  1.8779823 ]
[\PYGZhy{}0.86516815  1.87866847]
[\PYGZhy{}0.86592638  1.87935078]
[\PYGZhy{}0.86668035  1.88002925]
[\PYGZhy{}0.86743007  1.8807039 ]
[\PYGZhy{}0.86817558  1.88137477]
[\PYGZhy{}0.8689169   1.88204186]
[\PYGZhy{}0.86965404  1.8827052 ]
[\PYGZhy{}0.87038705  1.88336481]
[\PYGZhy{}0.87111593  1.88402071]
[\PYGZhy{}0.87184071  1.88467292]
[\PYGZhy{}0.87256141  1.88532146]
[\PYGZhy{}0.87327806  1.88596636]
[\PYGZhy{}0.87399069  1.88660763]
[\PYGZhy{}0.8746993   1.88724529]
[\PYGZhy{}0.87540393  1.88787937]
[\PYGZhy{}0.8761046   1.88850988]
[\PYGZhy{}0.87680133  1.88913684]
[\PYGZhy{}0.87749414  1.88976029]
[\PYGZhy{}0.87818305  1.89038022]
[\PYGZhy{}0.87886809  1.89099667]
[\PYGZhy{}0.87954928  1.89160965]
[\PYGZhy{}0.88022663  1.89221918]
[\PYGZhy{}0.88090018  1.89282529]
[\PYGZhy{}0.88156994  1.89342799]
[\PYGZhy{}0.88223593  1.8940273 ]
[\PYGZhy{}0.88289818  1.89462324]
[\PYGZhy{}0.8835567   1.89521582]
[\PYGZhy{}0.88421152  1.89580508]
[\PYGZhy{}0.88486266  1.89639102]
[\PYGZhy{}0.88551014  1.89697367]
[\PYGZhy{}0.88615397  1.89755304]
[\PYGZhy{}0.88679419  1.89812915]
[\PYGZhy{}0.8874308   1.89870202]
[\PYGZhy{}0.88806384  1.89927167]
[\PYGZhy{}0.88869331  1.89983812]
[\PYGZhy{}0.88931924  1.90040138]
[\PYGZhy{}0.88994166  1.90096147]
[\PYGZhy{}0.89056057  1.90151842]
[\PYGZhy{}0.89117601  1.90207223]
[\PYGZhy{}0.89178798  1.90262293]
[\PYGZhy{}0.89239651  1.90317053]
[\PYGZhy{}0.89300162  1.90371505]
[\PYGZhy{}0.89360333  1.90425651]
[\PYGZhy{}0.89420165  1.90479493]
[\PYGZhy{}0.89479661  1.90533031]
[\PYGZhy{}0.89538822  1.90586269]
[\PYGZhy{}0.89597651  1.90639207]
[\PYGZhy{}0.89656149  1.90691848]
[\PYGZhy{}0.89714317  1.90744192]
[\PYGZhy{}0.89772159  1.90796242]
[\PYGZhy{}0.89829675  1.90848   ]
[\PYGZhy{}0.89886868  1.90899466]
[\PYGZhy{}0.8994374   1.90950643]
[\PYGZhy{}0.90000291  1.91001532]
[\PYGZhy{}0.90056525  1.91052135]
[\PYGZhy{}0.90112442  1.91102454]
[\PYGZhy{}0.90168045  1.91152489]
[\PYGZhy{}0.90223335  1.91202243]
[\PYGZhy{}0.90278314  1.91251718]
[\PYGZhy{}0.90332984  1.91300914]
[\PYGZhy{}0.90387347  1.91349833]
[\PYGZhy{}0.90441403  1.91398477]
[\PYGZhy{}0.90495156  1.91446848]
[\PYGZhy{}0.90548607  1.91494947]
[\PYGZhy{}0.90601757  1.91542775]
[\PYGZhy{}0.90654608  1.91590335]
[\PYGZhy{}0.90707162  1.91637626]
[\PYGZhy{}0.9075942   1.91684652]
[\PYGZhy{}0.90811385  1.91731414]
[\PYGZhy{}0.90863057  1.91777912]
[\PYGZhy{}0.90914439  1.91824149]
[\PYGZhy{}0.90965532  1.91870126]
[\PYGZhy{}0.91016337  1.91915845]
[\PYGZhy{}0.91066857  1.91961306]
[\PYGZhy{}0.91117093  1.92006512]
[\PYGZhy{}0.91167046  1.92051463]
[\PYGZhy{}0.91216718  1.92096162]
[\PYGZhy{}0.91266111  1.92140609]
[\PYGZhy{}0.91315226  1.92184807]
[\PYGZhy{}0.91364065  1.92228756]
[\PYGZhy{}0.91412629  1.92272457]
[\PYGZhy{}0.9146092   1.92315913]
[\PYGZhy{}0.9150894   1.92359125]
[\PYGZhy{}0.9155669   1.92402093]
[\PYGZhy{}0.91604171  1.9244482 ]
[\PYGZhy{}0.91651385  1.92487307]
[\PYGZhy{}0.91698333  1.92529555]
[\PYGZhy{}0.91745018  1.92571565]
[\PYGZhy{}0.9179144   1.92613339]
[\PYGZhy{}0.91837601  1.92654878]
[\PYGZhy{}0.91883502  1.92696183]
[\PYGZhy{}0.91929145  1.92737256]
[\PYGZhy{}0.91974532  1.92778098]
[\PYGZhy{}0.92019663  1.92818711]
[\PYGZhy{}0.92064541  1.92859095]
[\PYGZhy{}0.92109166  1.92899252]
[\PYGZhy{}0.9215354   1.92939183]
[\PYGZhy{}0.92197665  1.92978889]
[\PYGZhy{}0.92241541  1.93018373]
[\PYGZhy{}0.92285171  1.93057634]
[\PYGZhy{}0.92328555  1.93096674]
[\PYGZhy{}0.92371696  1.93135495]
[\PYGZhy{}0.92414594  1.93174098]
[\PYGZhy{}0.9245725   1.93212484]
[\PYGZhy{}0.92499667  1.93250653]
[\PYGZhy{}0.92541845  1.93288608]
[\PYGZhy{}0.92583786  1.9332635 ]
[\PYGZhy{}0.92625492  1.93363879]
[\PYGZhy{}0.92666962  1.93401198]
[\PYGZhy{}0.927082    1.93438306]
[\PYGZhy{}0.92749205  1.93475206]
[\PYGZhy{}0.9278998   1.93511898]
[\PYGZhy{}0.92830526  1.93548384]
[\PYGZhy{}0.92870843  1.93584665]
[\PYGZhy{}0.92910934  1.93620741]
[\PYGZhy{}0.929508    1.93656615]
[\PYGZhy{}0.92990441  1.93692287]
[\PYGZhy{}0.93029859  1.93727759]
[\PYGZhy{}0.93069056  1.93763031]
[\PYGZhy{}0.93108032  1.93798105]
[\PYGZhy{}0.93146789  1.93832981]
[\PYGZhy{}0.93185328  1.93867661]
[\PYGZhy{}0.93223651  1.93902147]
[\PYGZhy{}0.93261758  1.93936438]
[\PYGZhy{}0.9329965   1.93970536]
[\PYGZhy{}0.9333733   1.94004443]
[\PYGZhy{}0.93374797  1.94038159]
[\PYGZhy{}0.93412054  1.94071686]
[\PYGZhy{}0.93449102  1.94105024]
[\PYGZhy{}0.93485941  1.94138174]
[\PYGZhy{}0.93522573  1.94171138]
[\PYGZhy{}0.93558999  1.94203917]
[\PYGZhy{}0.9359522   1.94236511]
[\PYGZhy{}0.93631237  1.94268922]
[\PYGZhy{}0.93667052  1.94301151]
[\PYGZhy{}0.93702665  1.94333199]
[\PYGZhy{}0.93738078  1.94365066]
[\PYGZhy{}0.93773292  1.94396754]
[\PYGZhy{}0.93808308  1.94428264]
[\PYGZhy{}0.93843127  1.94459597]
[\PYGZhy{}0.93877751  1.94490753]
[\PYGZhy{}0.93912179  1.94521735]
[\PYGZhy{}0.93946414  1.94552542]
[\PYGZhy{}0.93980457  1.94583176]
[\PYGZhy{}0.94014308  1.94613637]
[\PYGZhy{}0.94047968  1.94643927]
[\PYGZhy{}0.94081439  1.94674047]
[\PYGZhy{}0.94114723  1.94703998]
[\PYGZhy{}0.94147819  1.9473378 ]
[\PYGZhy{}0.94180728  1.94763395]
[\PYGZhy{}0.94213453  1.94792843]
[\PYGZhy{}0.94245994  1.94822125]
[\PYGZhy{}0.94278352  1.94851243]
[\PYGZhy{}0.94310527  1.94880197]
[\PYGZhy{}0.94342522  1.94908989]
[\PYGZhy{}0.94374337  1.94937618]
[\PYGZhy{}0.94405973  1.94966086]
[\PYGZhy{}0.94437431  1.94994395]
[\PYGZhy{}0.94468712  1.95022544]
[\PYGZhy{}0.94499818  1.95050534]
[\PYGZhy{}0.94530748  1.95078368]
[\PYGZhy{}0.94561504  1.95106045]
[\PYGZhy{}0.94592088  1.95133566]
[\PYGZhy{}0.94622499  1.95160932]
[\PYGZhy{}0.9465274   1.95188145]
[\PYGZhy{}0.9468281   1.95215204]
[\PYGZhy{}0.94712711  1.95242112]
[\PYGZhy{}0.94742444  1.95268868]
[\PYGZhy{}0.9477201   1.95295473]
[\PYGZhy{}0.9480141   1.95321929]
[\PYGZhy{}0.94830644  1.95348237]
[\PYGZhy{}0.94859714  1.95374396]
[\PYGZhy{}0.94888621  1.95400408]
[\PYGZhy{}0.94917365  1.95426274]
[\PYGZhy{}0.94945947  1.95451994]
[\PYGZhy{}0.94974368  1.9547757 ]
[\PYGZhy{}0.9500263   1.95503002]
[\PYGZhy{}0.95030733  1.95528291]
[\PYGZhy{}0.95058678  1.95553438]
[\PYGZhy{}0.95086465  1.95578443]
[\PYGZhy{}0.95114097  1.95603308]
[\PYGZhy{}0.95141573  1.95628032]
[\PYGZhy{}0.95168894  1.95652618]
[\PYGZhy{}0.95196062  1.95677066]
[\PYGZhy{}0.95223077  1.95701376]
[\PYGZhy{}0.9524994   1.95725549]
[\PYGZhy{}0.95276652  1.95749587]
[\PYGZhy{}0.95303214  1.95773489]
[\PYGZhy{}0.95329626  1.95797257]
[\PYGZhy{}0.9535589   1.95820891]
[\PYGZhy{}0.95382006  1.95844392]
[\PYGZhy{}0.95407976  1.95867761]
[\PYGZhy{}0.95433799  1.95890999]
[\PYGZhy{}0.95459477  1.95914106]
[\PYGZhy{}0.95485011  1.95937083]
[\PYGZhy{}0.95510401  1.95959931]
[\PYGZhy{}0.95535648  1.9598265 ]
[\PYGZhy{}0.95560753  1.96005242]
[\PYGZhy{}0.95585718  1.96027707]
[\PYGZhy{}0.95610541  1.96050045]
[\PYGZhy{}0.95635225  1.96072257]
[\PYGZhy{}0.95659771  1.96094345]
[\PYGZhy{}0.95684178  1.96116309]
[\PYGZhy{}0.95708448  1.96138149]
[\PYGZhy{}0.95732582  1.96159866]
[\PYGZhy{}0.9575658   1.96181461]
[\PYGZhy{}0.95780443  1.96202934]
[\PYGZhy{}0.95804171  1.96224287]
[\PYGZhy{}0.95827767  1.9624552 ]
[\PYGZhy{}0.95851229  1.96266633]
[\PYGZhy{}0.9587456   1.96287628]
[\PYGZhy{}0.95897759  1.96308505]
[\PYGZhy{}0.95920828  1.96329264]
[\PYGZhy{}0.95943768  1.96349906]
[\PYGZhy{}0.95966578  1.96370433]
[\PYGZhy{}0.9598926   1.96390843]
[\PYGZhy{}0.96011814  1.9641114 ]
[\PYGZhy{}0.96034242  1.96431322]
[\PYGZhy{}0.96056544  1.9645139 ]
[\PYGZhy{}0.9607872   1.96471346]
[\PYGZhy{}0.96100771  1.96491189]
[\PYGZhy{}0.96122698  1.96510921]
[\PYGZhy{}0.96144502  1.96530542]
[\PYGZhy{}0.96166184  1.96550052]
[\PYGZhy{}0.96187743  1.96569453]
[\PYGZhy{}0.96209182  1.96588745]
[\PYGZhy{}0.96230499  1.96607928]
[\PYGZhy{}0.96251697  1.96627004]
[\PYGZhy{}0.96272776  1.96645972]
[\PYGZhy{}0.96293736  1.96664833]
[\PYGZhy{}0.96314578  1.96683588]
[\PYGZhy{}0.96335303  1.96702238]
[\PYGZhy{}0.96355912  1.96720783]
[\PYGZhy{}0.96376404  1.96739224]
[\PYGZhy{}0.96396782  1.96757561]
[\PYGZhy{}0.96417044  1.96775795]
[\PYGZhy{}0.96437193  1.96793926]
[\PYGZhy{}0.96457229  1.96811956]
[\PYGZhy{}0.96477151  1.96829884]
[\PYGZhy{}0.96496962  1.96847711]
[\PYGZhy{}0.96516662  1.96865438]
[\PYGZhy{}0.9653625   1.96883065]
[\PYGZhy{}0.96555729  1.96900593]
[\PYGZhy{}0.96575097  1.96918023]
[\PYGZhy{}0.96594357  1.96935354]
[\PYGZhy{}0.96613509  1.96952588]
[\PYGZhy{}0.96632553  1.96969726]
[\PYGZhy{}0.9665149   1.96986766]
[\PYGZhy{}0.9667032   1.97003711]
[\PYGZhy{}0.96689045  1.97020561]
[\PYGZhy{}0.96707664  1.97037316]
[\PYGZhy{}0.96726179  1.97053977]
[\PYGZhy{}0.96744589  1.97070544]
[\PYGZhy{}0.96762896  1.97087017]
[\PYGZhy{}0.967811    1.97103399]
[\PYGZhy{}0.96799201  1.97119688]
[\PYGZhy{}0.96817201  1.97135885]
[\PYGZhy{}0.96835099  1.97151992]
[\PYGZhy{}0.96852897  1.97168007]
[\PYGZhy{}0.96870595  1.97183933]
[\PYGZhy{}0.96888193  1.97199769]
[\PYGZhy{}0.96905693  1.97215516]
[\PYGZhy{}0.96923093  1.97231175]
[\PYGZhy{}0.96940396  1.97246746]
[\PYGZhy{}0.96957602  1.97262228]
[\PYGZhy{}0.96974711  1.97277624]
[\PYGZhy{}0.96991724  1.97292934]
[\PYGZhy{}0.97008641  1.97308157]
[\PYGZhy{}0.97025463  1.97323295]
[\PYGZhy{}0.9704219   1.97338347]
[\PYGZhy{}0.97058824  1.97353315]
[\PYGZhy{}0.97075363  1.97368199]
[\PYGZhy{}0.9709181   1.97382998]
[\PYGZhy{}0.97108164  1.97397715]
[\PYGZhy{}0.97124427  1.97412349]
[\PYGZhy{}0.97140597  1.97426901]
[\PYGZhy{}0.97156677  1.97441371]
[\PYGZhy{}0.97172667  1.97455759]
[\PYGZhy{}0.97188566  1.97470067]
[\PYGZhy{}0.97204376  1.97484294]
[\PYGZhy{}0.97220098  1.97498441]
[\PYGZhy{}0.9723573   1.97512509]
[\PYGZhy{}0.97251275  1.97526497]
[\PYGZhy{}0.97266733  1.97540407]
[\PYGZhy{}0.97282103  1.97554238]
[\PYGZhy{}0.97297388  1.97567992]
[\PYGZhy{}0.97312586  1.97581669]
[\PYGZhy{}0.97327698  1.97595268]
[\PYGZhy{}0.97342726  1.97608791]
[\PYGZhy{}0.97357669  1.97622238]
[\PYGZhy{}0.97372529  1.97635609]
[\PYGZhy{}0.97387304  1.97648906]
[\PYGZhy{}0.97401997  1.97662127]
[\PYGZhy{}0.97416607  1.97675274]
[\PYGZhy{}0.97431134  1.97688347]
[\PYGZhy{}0.9744558   1.97701347]
[\PYGZhy{}0.97459945  1.97714273]
[\PYGZhy{}0.97474229  1.97727127]
[\PYGZhy{}0.97488433  1.97739909]
[\PYGZhy{}0.97502557  1.97752618]
[\PYGZhy{}0.97516601  1.97765257]
[\PYGZhy{}0.97530567  1.97777824]
[\PYGZhy{}0.97544453  1.9779032 ]
[\PYGZhy{}0.97558262  1.97802746]
[\PYGZhy{}0.97571993  1.97815103]
[\PYGZhy{}0.97585647  1.97827389]
[\PYGZhy{}0.97599224  1.97839607]
[\PYGZhy{}0.97612725  1.97851756]
[\PYGZhy{}0.9762615   1.97863837]
[\PYGZhy{}0.976395    1.97875849]
[\PYGZhy{}0.97652774  1.97887795]
[\PYGZhy{}0.97665973  1.97899673]
[\PYGZhy{}0.97679099  1.97911484]
[\PYGZhy{}0.97692151  1.97923229]
[\PYGZhy{}0.97705129  1.97934907]
[\PYGZhy{}0.97718034  1.9794652 ]
[\PYGZhy{}0.97730867  1.97958068]
[\PYGZhy{}0.97743627  1.97969551]
[\PYGZhy{}0.97756316  1.97980969]
[\PYGZhy{}0.97768933  1.97992323]
[\PYGZhy{}0.9778148   1.98003614]
[\PYGZhy{}0.97793956  1.9801484 ]
[\PYGZhy{}0.97806361  1.98026004]
[\PYGZhy{}0.97818697  1.98037105]
[\PYGZhy{}0.97830964  1.98048143]
[\PYGZhy{}0.97843161  1.98059119]
[\PYGZhy{}0.9785529   1.98070034]
[\PYGZhy{}0.97867351  1.98080887]
[\PYGZhy{}0.97879344  1.98091679]
[\PYGZhy{}0.9789127   1.98102411]
[\PYGZhy{}0.97903128  1.98113082]
[\PYGZhy{}0.9791492   1.98123693]
[\PYGZhy{}0.97926645  1.98134244]
[\PYGZhy{}0.97938305  1.98144736]
[\PYGZhy{}0.97949899  1.9815517 ]
[\PYGZhy{}0.97961428  1.98165544]
[\PYGZhy{}0.97972892  1.9817586 ]
[\PYGZhy{}0.97984291  1.98186118]
[\PYGZhy{}0.97995627  1.98196318]
[\PYGZhy{}0.98006898  1.98206462]
[\PYGZhy{}0.98018106  1.98216548]
[\PYGZhy{}0.98029252  1.98226577]
[\PYGZhy{}0.98040334  1.9823655 ]
[\PYGZhy{}0.98051354  1.98246466]
[\PYGZhy{}0.98062313  1.98256327]
[\PYGZhy{}0.98073209  1.98266133]
[\PYGZhy{}0.98084044  1.98275883]
[\PYGZhy{}0.98094819  1.98285579]
[\PYGZhy{}0.98105533  1.9829522 ]
[\PYGZhy{}0.98116186  1.98304807]
[\PYGZhy{}0.9812678  1.9831434]
[\PYGZhy{}0.98137314  1.98323819]
[\PYGZhy{}0.98147789  1.98333245]
[\PYGZhy{}0.98158205  1.98342618]
[\PYGZhy{}0.98168562  1.98351939]
[\PYGZhy{}0.98178861  1.98361206]
[\PYGZhy{}0.98189102  1.98370422]
[\PYGZhy{}0.98199286  1.98379586]
[\PYGZhy{}0.98209412  1.98388699]
[\PYGZhy{}0.98219482  1.9839776 ]
[\PYGZhy{}0.98229495  1.9840677 ]
[\PYGZhy{}0.98239451  1.9841573 ]
[\PYGZhy{}0.98249351  1.98424639]
[\PYGZhy{}0.98259196  1.98433498]
[\PYGZhy{}0.98268986  1.98442307]
[\PYGZhy{}0.9827872   1.98451067]
[\PYGZhy{}0.982884    1.98459777]
[\PYGZhy{}0.98298025  1.98468439]
[\PYGZhy{}0.98307596  1.98477051]
[\PYGZhy{}0.98317113  1.98485616]
[\PYGZhy{}0.98326577  1.98494132]
[\PYGZhy{}0.98335987  1.985026  ]
[\PYGZhy{}0.98345345  1.98511021]
[\PYGZhy{}0.9835465   1.98519394]
[\PYGZhy{}0.98363903  1.9852772 ]
[\PYGZhy{}0.98373103  1.98536   ]
[\PYGZhy{}0.98382252  1.98544232]
[\PYGZhy{}0.9839135   1.98552419]
[\PYGZhy{}0.98400396  1.98560559]
[\PYGZhy{}0.98409391  1.98568654]
[\PYGZhy{}0.98418336  1.98576703]
[\PYGZhy{}0.98427231  1.98584707]
[\PYGZhy{}0.98436075  1.98592666]
[\PYGZhy{}0.9844487  1.9860058]
[\PYGZhy{}0.98453615  1.9860845 ]
[\PYGZhy{}0.98462311  1.98616275]
[\PYGZhy{}0.98470958  1.98624057]
[\PYGZhy{}0.98479557  1.98631794]
[\PYGZhy{}0.98488107  1.98639489]
[\PYGZhy{}0.98496609  1.98647139]
[\PYGZhy{}0.98505064  1.98654747]
[\PYGZhy{}0.98513471  1.98662312]
[\PYGZhy{}0.9852183   1.98669835]
[\PYGZhy{}0.98530143  1.98677315]
[\PYGZhy{}0.98538408  1.98684753]
[\PYGZhy{}0.98546628  1.98692149]
[\PYGZhy{}0.98554801  1.98699504]
[\PYGZhy{}0.98562928  1.98706818]
[\PYGZhy{}0.98571009  1.9871409 ]
[\PYGZhy{}0.98579045  1.98721321]
[\PYGZhy{}0.98587036  1.98728512]
[\PYGZhy{}0.98594982  1.98735662]
[\PYGZhy{}0.98602883  1.98742772]
[\PYGZhy{}0.9861074   1.98749842]
[\PYGZhy{}0.98618552  1.98756872]
[\PYGZhy{}0.98626321  1.98763863]
[\PYGZhy{}0.98634046  1.98770815]
[\PYGZhy{}0.98641727  1.98777727]
[\PYGZhy{}0.98649365  1.987846  ]
[\PYGZhy{}0.98656961  1.98791435]
[\PYGZhy{}0.98664513  1.98798232]
[\PYGZhy{}0.98672023  1.9880499 ]
[\PYGZhy{}0.98679491  1.9881171 ]
[\PYGZhy{}0.98686917  1.98818392]
[\PYGZhy{}0.98694301  1.98825037]
[\PYGZhy{}0.98701644  1.98831644]
[\PYGZhy{}0.98708945  1.98838215]
[\PYGZhy{}0.98716205  1.98844748]
[\PYGZhy{}0.98723425  1.98851245]
[\PYGZhy{}0.98730604  1.98857705]
[\PYGZhy{}0.98737742  1.98864128]
[\PYGZhy{}0.9874484   1.98870516]
[\PYGZhy{}0.98751899  1.98876868]
[\PYGZhy{}0.98758918  1.98883184]
[\PYGZhy{}0.98765897  1.98889464]
[\PYGZhy{}0.98772837  1.98895709]
[\PYGZhy{}0.98779738  1.98901919]
[\PYGZhy{}0.987866    1.98908094]
[\PYGZhy{}0.98793424  1.98914234]
[\PYGZhy{}0.98800209  1.9892034 ]
[\PYGZhy{}0.98806956  1.98926412]
[\PYGZhy{}0.98813665  1.98932449]
[\PYGZhy{}0.98820336  1.98938452]
[\PYGZhy{}0.9882697   1.98944422]
[\PYGZhy{}0.98833567  1.98950358]
[\PYGZhy{}0.98840126  1.98956261]
[\PYGZhy{}0.98846649  1.9896213 ]
[\PYGZhy{}0.98853135  1.98967967]
[\PYGZhy{}0.98859584  1.9897377 ]
[\PYGZhy{}0.98865997  1.98979541]
[\PYGZhy{}0.98872374  1.9898528 ]
[\PYGZhy{}0.98878715  1.98990986]
[\PYGZhy{}0.98885021  1.98996661]
[\PYGZhy{}0.98891291  1.99002303]
[\PYGZhy{}0.98897526  1.99007913]
[\PYGZhy{}0.98903726  1.99013492]
[\PYGZhy{}0.98909891  1.9901904 ]
[\PYGZhy{}0.98916021  1.99024556]
[\PYGZhy{}0.98922117  1.99030042]
[\PYGZhy{}0.98928178  1.99035496]
[\PYGZhy{}0.98934206  1.9904092 ]
[\PYGZhy{}0.98940199  1.99046314]
[\PYGZhy{}0.98946159  1.99051677]
[\PYGZhy{}0.98952085  1.9905701 ]
[\PYGZhy{}0.98957978  1.99062313]
[\PYGZhy{}0.98963838  1.99067586]
\end{sphinxVerbatim}

\begin{sphinxVerbatim}[commandchars=\\\{\}]
[\PYGZhy{}0.98969665  1.99072829]
[\PYGZhy{}0.98975459  1.99078043]
[\PYGZhy{}0.9898122   1.99083228]
[\PYGZhy{}0.9898695   1.99088383]
[\PYGZhy{}0.98992646  1.9909351 ]
[\PYGZhy{}0.98998311  1.99098607]
[\PYGZhy{}0.99003944  1.99103676]
[\PYGZhy{}0.99009546  1.99108717]
[\PYGZhy{}0.99015115  1.99113729]
[\PYGZhy{}0.99020654  1.99118713]
[\PYGZhy{}0.99026161  1.99123669]
[\PYGZhy{}0.99031638  1.99128597]
[\PYGZhy{}0.99037083  1.99133497]
[\PYGZhy{}0.99042498  1.9913837 ]
[\PYGZhy{}0.99047883  1.99143215]
[\PYGZhy{}0.99053237  1.99148034]
[\PYGZhy{}0.99058561  1.99152825]
[\PYGZhy{}0.99063855  1.99157589]
[\PYGZhy{}0.9906912   1.99162326]
[\PYGZhy{}0.99074355  1.99167037]
[\PYGZhy{}0.9907956   1.99171721]
[\PYGZhy{}0.99084736  1.99176379]
[\PYGZhy{}0.99089883  1.9918101 ]
[\PYGZhy{}0.99095001  1.99185616]
[\PYGZhy{}0.9910009   1.99190196]
[\PYGZhy{}0.99105151  1.9919475 ]
[\PYGZhy{}0.99110183  1.99199278]
[\PYGZhy{}0.99115187  1.99203781]
[\PYGZhy{}0.99120163  1.99208258]
[\PYGZhy{}0.99125111  1.99212711]
[\PYGZhy{}0.99130031  1.99217138]
[\PYGZhy{}0.99134923  1.99221541]
[\PYGZhy{}0.99139788  1.99225918]
[\PYGZhy{}0.99144625  1.99230271]
[\PYGZhy{}0.99149435  1.992346  ]
[\PYGZhy{}0.99154218  1.99238904]
[\PYGZhy{}0.99158975  1.99243184]
[\PYGZhy{}0.99163704  1.9924744 ]
[\PYGZhy{}0.99168407  1.99251672]
[\PYGZhy{}0.99173084  1.9925588 ]
[\PYGZhy{}0.99177734  1.99260065]
[\PYGZhy{}0.99182358  1.99264226]
[\PYGZhy{}0.99186956  1.99268364]
[\PYGZhy{}0.99191528  1.99272478]
[\PYGZhy{}0.99196075  1.99276569]
[\PYGZhy{}0.99200595  1.99280637]
[\PYGZhy{}0.99205091  1.99284683]
[\PYGZhy{}0.99209561  1.99288705]
[\PYGZhy{}0.99214006  1.99292705]
[\PYGZhy{}0.99218426  1.99296683]
[\PYGZhy{}0.99222821  1.99300638]
[\PYGZhy{}0.99227192  1.99304571]
[\PYGZhy{}0.99231538  1.99308482]
[\PYGZhy{}0.99235859  1.9931237 ]
[\PYGZhy{}0.99240156  1.99316237]
[\PYGZhy{}0.99244429  1.99320082]
[\PYGZhy{}0.99248678  1.99323906]
[\PYGZhy{}0.99252903  1.99327708]
[\PYGZhy{}0.99257105  1.99331489]
[\PYGZhy{}0.99261282  1.99335248]
[\PYGZhy{}0.99265436  1.99338986]
[\PYGZhy{}0.99269567  1.99342703]
[\PYGZhy{}0.99273675  1.993464  ]
[\PYGZhy{}0.99277759  1.99350075]
[\PYGZhy{}0.99281821  1.9935373 ]
[\PYGZhy{}0.9928586   1.99357364]
[\PYGZhy{}0.99289876  1.99360978]
[\PYGZhy{}0.99293869  1.99364572]
[\PYGZhy{}0.9929784   1.99368145]
[\PYGZhy{}0.99301789  1.99371698]
[\PYGZhy{}0.99305715  1.99375232]
[\PYGZhy{}0.99309619  1.99378745]
[\PYGZhy{}0.99313502  1.99382239]
[\PYGZhy{}0.99317362  1.99385713]
[\PYGZhy{}0.99321201  1.99389167]
[\PYGZhy{}0.99325018  1.99392602]
[\PYGZhy{}0.99328814  1.99396018]
[\PYGZhy{}0.99332588  1.99399414]
[\PYGZhy{}0.99336342  1.99402792]
[\PYGZhy{}0.99340074  1.9940615 ]
[\PYGZhy{}0.99343785  1.9940949 ]
[\PYGZhy{}0.99347475  1.9941281 ]
[\PYGZhy{}0.99351145  1.99416112]
[\PYGZhy{}0.99354793  1.99419396]
[\PYGZhy{}0.99358422  1.99422661]
[\PYGZhy{}0.9936203   1.99425908]
[\PYGZhy{}0.99365617  1.99429136]
[\PYGZhy{}0.99369185  1.99432346]
[\PYGZhy{}0.99372732  1.99435539]
[\PYGZhy{}0.9937626   1.99438713]
[\PYGZhy{}0.99379767  1.99441869]
[\PYGZhy{}0.99383255  1.99445008]
[\PYGZhy{}0.99386723  1.99448129]
[\PYGZhy{}0.99390172  1.99451232]
[\PYGZhy{}0.99393601  1.99454318]
[\PYGZhy{}0.99397012  1.99457387]
[\PYGZhy{}0.99400402  1.99460438]
[\PYGZhy{}0.99403774  1.99463473]
[\PYGZhy{}0.99407127  1.9946649 ]
[\PYGZhy{}0.99410461  1.9946949 ]
[\PYGZhy{}0.99413777  1.99472473]
[\PYGZhy{}0.99417073  1.9947544 ]
[\PYGZhy{}0.99420351  1.9947839 ]
[\PYGZhy{}0.99423611  1.99481323]
[\PYGZhy{}0.99426852  1.9948424 ]
[\PYGZhy{}0.99430075  1.9948714 ]
[\PYGZhy{}0.9943328   1.99490024]
[\PYGZhy{}0.99436467  1.99492892]
[\PYGZhy{}0.99439636  1.99495744]
[\PYGZhy{}0.99442788  1.9949858 ]
[\PYGZhy{}0.99445921  1.99501399]
[\PYGZhy{}0.99449037  1.99504203]
[\PYGZhy{}0.99452135  1.99506991]
[\PYGZhy{}0.99455216  1.99509764]
[\PYGZhy{}0.9945828   1.99512521]
[\PYGZhy{}0.99461326  1.99515262]
[\PYGZhy{}0.99464355  1.99517988]
[\PYGZhy{}0.99467368  1.99520698]
[\PYGZhy{}0.99470363  1.99523394]
[\PYGZhy{}0.99473341  1.99526074]
[\PYGZhy{}0.99476303  1.99528739]
[\PYGZhy{}0.99479248  1.99531389]
[\PYGZhy{}0.99482176  1.99534025]
[\PYGZhy{}0.99485088  1.99536645]
[\PYGZhy{}0.99487984  1.99539251]
[\PYGZhy{}0.99490863  1.99541842]
[\PYGZhy{}0.99493726  1.99544418]
[\PYGZhy{}0.99496573  1.9954698 ]
[\PYGZhy{}0.99499405  1.99549528]
[\PYGZhy{}0.9950222   1.99552061]
[\PYGZhy{}0.99505019  1.9955458 ]
[\PYGZhy{}0.99507802  1.99557085]
[\PYGZhy{}0.9951057   1.99559575]
[\PYGZhy{}0.99513323  1.99562052]
[\PYGZhy{}0.99516059  1.99564515]
[\PYGZhy{}0.99518781  1.99566964]
[\PYGZhy{}0.99521487  1.99569399]
[\PYGZhy{}0.99524178  1.99571821]
[\PYGZhy{}0.99526854  1.99574228]
[\PYGZhy{}0.99529515  1.99576623]
[\PYGZhy{}0.9953216   1.99579004]
[\PYGZhy{}0.99534791  1.99581371]
[\PYGZhy{}0.99537407  1.99583725]
[\PYGZhy{}0.99540009  1.99586066]
[\PYGZhy{}0.99542595  1.99588394]
[\PYGZhy{}0.99545168  1.99590709]
[\PYGZhy{}0.99547725  1.9959301 ]
[\PYGZhy{}0.99550269  1.99595299]
[\PYGZhy{}0.99552798  1.99597575]
[\PYGZhy{}0.99555313  1.99599838]
[\PYGZhy{}0.99557813  1.99602088]
[\PYGZhy{}0.995603    1.99604326]
[\PYGZhy{}0.99562773  1.99606551]
[\PYGZhy{}0.99565231  1.99608764]
[\PYGZhy{}0.99567676  1.99610964]
[\PYGZhy{}0.99570108  1.99613151]
[\PYGZhy{}0.99572525  1.99615327]
[\PYGZhy{}0.99574929  1.9961749 ]
[\PYGZhy{}0.99577319  1.99619641]
[\PYGZhy{}0.99579696  1.9962178 ]
[\PYGZhy{}0.9958206   1.99623907]
[\PYGZhy{}0.9958441   1.99626022]
[\PYGZhy{}0.99586747  1.99628125]
[\PYGZhy{}0.99589071  1.99630216]
[\PYGZhy{}0.99591382  1.99632296]
[\PYGZhy{}0.9959368   1.99634364]
[\PYGZhy{}0.99595965  1.9963642 ]
[\PYGZhy{}0.99598237  1.99638464]
[\PYGZhy{}0.99600496  1.99640497]
[\PYGZhy{}0.99602743  1.99642519]
[\PYGZhy{}0.99604977  1.99644529]
[\PYGZhy{}0.99607198  1.99646528]
[\PYGZhy{}0.99609407  1.99648516]
[\PYGZhy{}0.99611604  1.99650493]
[\PYGZhy{}0.99613788  1.99652458]
[\PYGZhy{}0.9961596   1.99654413]
[\PYGZhy{}0.99618119  1.99656356]
[\PYGZhy{}0.99620267  1.99658288]
[\PYGZhy{}0.99622402  1.9966021 ]
[\PYGZhy{}0.99624526  1.99662121]
[\PYGZhy{}0.99626637  1.99664021]
[\PYGZhy{}0.99628737  1.9966591 ]
[\PYGZhy{}0.99630825  1.99667789]
[\PYGZhy{}0.99632901  1.99669657]
[\PYGZhy{}0.99634965  1.99671515]
[\PYGZhy{}0.99637018  1.99673362]
[\PYGZhy{}0.99639059  1.99675199]
[\PYGZhy{}0.99641089  1.99677026]
[\PYGZhy{}0.99643107  1.99678842]
[\PYGZhy{}0.99645114  1.99680648]
[\PYGZhy{}0.9964711   1.99682444]
[\PYGZhy{}0.99649094  1.9968423 ]
[\PYGZhy{}0.99651068  1.99686005]
[\PYGZhy{}0.9965303   1.99687771]
[\PYGZhy{}0.99654981  1.99689527]
[\PYGZhy{}0.99656921  1.99691273]
[\PYGZhy{}0.99658851  1.99693009]
[\PYGZhy{}0.99660769  1.99694735]
[\PYGZhy{}0.99662677  1.99696452]
[\PYGZhy{}0.99664574  1.99698159]
[\PYGZhy{}0.9966646   1.99699856]
[\PYGZhy{}0.99668336  1.99701544]
[\PYGZhy{}0.99670201  1.99703223]
[\PYGZhy{}0.99672055  1.99704892]
[\PYGZhy{}0.996739    1.99706551]
[\PYGZhy{}0.99675733  1.99708201]
[\PYGZhy{}0.99677557  1.99709842]
[\PYGZhy{}0.9967937   1.99711474]
[\PYGZhy{}0.99681173  1.99713096]
[\PYGZhy{}0.99682966  1.9971471 ]
[\PYGZhy{}0.99684749  1.99716314]
[\PYGZhy{}0.99686522  1.9971791 ]
[\PYGZhy{}0.99688285  1.99719496]
[\PYGZhy{}0.99690038  1.99721073]
[\PYGZhy{}0.99691781  1.99722642]
[\PYGZhy{}0.99693514  1.99724202]
[\PYGZhy{}0.99695238  1.99725753]
[\PYGZhy{}0.99696951  1.99727295]
[\PYGZhy{}0.99698656  1.99728828]
[\PYGZhy{}0.9970035   1.99730353]
[\PYGZhy{}0.99702035  1.9973187 ]
[\PYGZhy{}0.99703711  1.99733377]
[\PYGZhy{}0.99705377  1.99734877]
[\PYGZhy{}0.99707034  1.99736368]
[\PYGZhy{}0.99708681  1.9973785 ]
[\PYGZhy{}0.9971032   1.99739324]
[\PYGZhy{}0.99711949  1.9974079 ]
[\PYGZhy{}0.99713569  1.99742248]
[\PYGZhy{}0.99715179  1.99743698]
[\PYGZhy{}0.99716781  1.99745139]
[\PYGZhy{}0.99718374  1.99746572]
[\PYGZhy{}0.99719957  1.99747997]
[\PYGZhy{}0.99721532  1.99749414]
[\PYGZhy{}0.99723098  1.99750824]
[\PYGZhy{}0.99724655  1.99752225]
[\PYGZhy{}0.99726204  1.99753618]
[\PYGZhy{}0.99727743  1.99755004]
[\PYGZhy{}0.99729274  1.99756381]
[\PYGZhy{}0.99730797  1.99757751]
[\PYGZhy{}0.99732311  1.99759114]
[\PYGZhy{}0.99733816  1.99760468]
[\PYGZhy{}0.99735313  1.99761815]
[\PYGZhy{}0.99736801  1.99763155]
[\PYGZhy{}0.99738282  1.99764487]
[\PYGZhy{}0.99739753  1.99765811]
[\PYGZhy{}0.99741217  1.99767128]
[\PYGZhy{}0.99742672  1.99768438]
[\PYGZhy{}0.99744119  1.9976974 ]
[\PYGZhy{}0.99745558  1.99771035]
[\PYGZhy{}0.99746989  1.99772322]
[\PYGZhy{}0.99748412  1.99773603]
[\PYGZhy{}0.99749827  1.99774876]
[\PYGZhy{}0.99751233  1.99776142]
[\PYGZhy{}0.99752632  1.99777401]
[\PYGZhy{}0.99754024  1.99778652]
[\PYGZhy{}0.99755407  1.99779897]
[\PYGZhy{}0.99756782  1.99781135]
[\PYGZhy{}0.9975815   1.99782366]
[\PYGZhy{}0.9975951  1.9978359]
[\PYGZhy{}0.99760862  1.99784807]
[\PYGZhy{}0.99762207  1.99786017]
[\PYGZhy{}0.99763544  1.9978722 ]
[\PYGZhy{}0.99764874  1.99788417]
[\PYGZhy{}0.99766196  1.99789606]
[\PYGZhy{}0.99767511  1.9979079 ]
[\PYGZhy{}0.99768819  1.99791966]
[\PYGZhy{}0.99770119  1.99793136]
[\PYGZhy{}0.99771411  1.99794299]
[\PYGZhy{}0.99772697  1.99795456]
[\PYGZhy{}0.99773975  1.99796606]
[\PYGZhy{}0.99775246  1.9979775 ]
[\PYGZhy{}0.9977651   1.99798887]
[\PYGZhy{}0.99777767  1.99800018]
[\PYGZhy{}0.99779017  1.99801143]
[\PYGZhy{}0.99780259  1.99802261]
[\PYGZhy{}0.99781495  1.99803373]
[\PYGZhy{}0.99782724  1.99804479]
[\PYGZhy{}0.99783946  1.99805579]
[\PYGZhy{}0.99785161  1.99806672]
[\PYGZhy{}0.99786369  1.99807759]
[\PYGZhy{}0.9978757  1.9980884]
[\PYGZhy{}0.99788765  1.99809915]
[\PYGZhy{}0.99789953  1.99810984]
[\PYGZhy{}0.99791134  1.99812047]
[\PYGZhy{}0.99792308  1.99813104]
[\PYGZhy{}0.99793476  1.99814155]
[\PYGZhy{}0.99794638  1.998152  ]
[\PYGZhy{}0.99795793  1.99816239]
[\PYGZhy{}0.99796941  1.99817273]
[\PYGZhy{}0.99798083  1.998183  ]
[\PYGZhy{}0.99799218  1.99819322]
[\PYGZhy{}0.99800347  1.99820338]
[\PYGZhy{}0.9980147   1.99821348]
[\PYGZhy{}0.99802587  1.99822353]
[\PYGZhy{}0.99803697  1.99823352]
[\PYGZhy{}0.99804801  1.99824345]
[\PYGZhy{}0.99805898  1.99825333]
[\PYGZhy{}0.9980699   1.99826315]
[\PYGZhy{}0.99808075  1.99827292]
[\PYGZhy{}0.99809155  1.99828263]
[\PYGZhy{}0.99810228  1.99829229]
[\PYGZhy{}0.99811295  1.99830189]
[\PYGZhy{}0.99812356  1.99831144]
[\PYGZhy{}0.99813411  1.99832094]
[\PYGZhy{}0.99814461  1.99833038]
[\PYGZhy{}0.99815504  1.99833977]
[\PYGZhy{}0.99816542  1.99834911]
[\PYGZhy{}0.99817573  1.99835839]
[\PYGZhy{}0.99818599  1.99836762]
[\PYGZhy{}0.99819619  1.9983768 ]
[\PYGZhy{}0.99820634  1.99838593]
[\PYGZhy{}0.99821642  1.99839501]
[\PYGZhy{}0.99822645  1.99840403]
[\PYGZhy{}0.99823643  1.99841301]
[\PYGZhy{}0.99824634  1.99842193]
[\PYGZhy{}0.99825621  1.99843081]
[\PYGZhy{}0.99826601  1.99843963]
[\PYGZhy{}0.99827576  1.99844841]
[\PYGZhy{}0.99828546  1.99845713]
[\PYGZhy{}0.9982951   1.99846581]
[\PYGZhy{}0.99830469  1.99847444]
[\PYGZhy{}0.99831422  1.99848301]
[\PYGZhy{}0.9983237   1.99849154]
[\PYGZhy{}0.99833313  1.99850003]
[\PYGZhy{}0.9983425   1.99850846]
[\PYGZhy{}0.99835182  1.99851685]
[\PYGZhy{}0.99836109  1.99852519]
[\PYGZhy{}0.99837031  1.99853348]
[\PYGZhy{}0.99837947  1.99854173]
[\PYGZhy{}0.99838859  1.99854993]
[\PYGZhy{}0.99839765  1.99855809]
[\PYGZhy{}0.99840666  1.9985662 ]
[\PYGZhy{}0.99841562  1.99857426]
[\PYGZhy{}0.99842453  1.99858228]
[\PYGZhy{}0.99843339  1.99859025]
[\PYGZhy{}0.9984422   1.99859818]
[\PYGZhy{}0.99845096  1.99860606]
[\PYGZhy{}0.99845967  1.9986139 ]
[\PYGZhy{}0.99846833  1.99862169]
[\PYGZhy{}0.99847695  1.99862944]
[\PYGZhy{}0.99848551  1.99863715]
[\PYGZhy{}0.99849403  1.99864482]
[\PYGZhy{}0.9985025   1.99865244]
[\PYGZhy{}0.99851092  1.99866001]
[\PYGZhy{}0.99851929  1.99866755]
[\PYGZhy{}0.99852762  1.99867504]
[\PYGZhy{}0.9985359   1.99868249]
[\PYGZhy{}0.99854413  1.9986899 ]
[\PYGZhy{}0.99855232  1.99869727]
[\PYGZhy{}0.99856046  1.9987046 ]
[\PYGZhy{}0.99856855  1.99871188]
[\PYGZhy{}0.9985766   1.99871912]
[\PYGZhy{}0.99858461  1.99872633]
[\PYGZhy{}0.99859257  1.99873349]
[\PYGZhy{}0.99860048  1.99874061]
[\PYGZhy{}0.99860835  1.99874769]
[\PYGZhy{}0.99861618  1.99875474]
[\PYGZhy{}0.99862396  1.99876174]
[\PYGZhy{}0.9986317  1.9987687]
[\PYGZhy{}0.99863939  1.99877563]
[\PYGZhy{}0.99864705  1.99878251]
[\PYGZhy{}0.99865465  1.99878936]
[\PYGZhy{}0.99866222  1.99879617]
[\PYGZhy{}0.99866974  1.99880294]
[\PYGZhy{}0.99867722  1.99880967]
[\PYGZhy{}0.99868466  1.99881636]
[\PYGZhy{}0.99869206  1.99882302]
[\PYGZhy{}0.99869941  1.99882964]
[\PYGZhy{}0.99870673  1.99883622]
[\PYGZhy{}0.998714    1.99884276]
[\PYGZhy{}0.99872123  1.99884927]
[\PYGZhy{}0.99872842  1.99885574]
[\PYGZhy{}0.99873557  1.99886218]
[\PYGZhy{}0.99874268  1.99886858]
[\PYGZhy{}0.99874975  1.99887494]
[\PYGZhy{}0.99875679  1.99888126]
[\PYGZhy{}0.99876378  1.99888756]
[\PYGZhy{}0.99877073  1.99889381]
[\PYGZhy{}0.99877764  1.99890003]
[\PYGZhy{}0.99878452  1.99890622]
[\PYGZhy{}0.99879135  1.99891237]
[\PYGZhy{}0.99879815  1.99891849]
[\PYGZhy{}0.99880491  1.99892457]
[\PYGZhy{}0.99881163  1.99893062]
[\PYGZhy{}0.99881831  1.99893663]
[\PYGZhy{}0.99882495  1.99894261]
[\PYGZhy{}0.99883156  1.99894856]
[\PYGZhy{}0.99883813  1.99895447]
[\PYGZhy{}0.99884467  1.99896035]
[\PYGZhy{}0.99885116  1.99896619]
[\PYGZhy{}0.99885762  1.99897201]
[\PYGZhy{}0.99886405  1.99897779]
[\PYGZhy{}0.99887044  1.99898354]
[\PYGZhy{}0.99887679  1.99898925]
[\PYGZhy{}0.99888311  1.99899494]
[\PYGZhy{}0.99888939  1.99900059]
[\PYGZhy{}0.99889563  1.99900621]
[\PYGZhy{}0.99890184  1.9990118 ]
[\PYGZhy{}0.99890802  1.99901735]
[\PYGZhy{}0.99891416  1.99902288]
[\PYGZhy{}0.99892026  1.99902838]
[\PYGZhy{}0.99892634  1.99903384]
[\PYGZhy{}0.99893237  1.99903927]
[\PYGZhy{}0.99893838  1.99904468]
[\PYGZhy{}0.99894435  1.99905005]
[\PYGZhy{}0.99895028  1.99905539]
[\PYGZhy{}0.99895619  1.9990607 ]
[\PYGZhy{}0.99896206  1.99906598]
[\PYGZhy{}0.99896789  1.99907124]
[\PYGZhy{}0.9989737   1.99907646]
[\PYGZhy{}0.99897947  1.99908165]
[\PYGZhy{}0.99898521  1.99908682]
[\PYGZhy{}0.99899092  1.99909195]
[\PYGZhy{}0.99899659  1.99909706]
[\PYGZhy{}0.99900223  1.99910214]
[\PYGZhy{}0.99900784  1.99910719]
[\PYGZhy{}0.99901342  1.99911221]
[\PYGZhy{}0.99901897  1.9991172 ]
[\PYGZhy{}0.99902449  1.99912216]
[\PYGZhy{}0.99902997  1.9991271 ]
[\PYGZhy{}0.99903543  1.99913201]
[\PYGZhy{}0.99904085  1.99913689]
[\PYGZhy{}0.99904625  1.99914174]
[\PYGZhy{}0.99905161  1.99914657]
[\PYGZhy{}0.99905694  1.99915137]
[\PYGZhy{}0.99906225  1.99915614]
[\PYGZhy{}0.99906752  1.99916089]
[\PYGZhy{}0.99907276  1.99916561]
[\PYGZhy{}0.99907798  1.9991703 ]
[\PYGZhy{}0.99908316  1.99917496]
[\PYGZhy{}0.99908832  1.9991796 ]
[\PYGZhy{}0.99909345  1.99918422]
[\PYGZhy{}0.99909854  1.9991888 ]
[\PYGZhy{}0.99910361  1.99919337]
[\PYGZhy{}0.99910865  1.9991979 ]
[\PYGZhy{}0.99911367  1.99920241]
[\PYGZhy{}0.99911865  1.9992069 ]
[\PYGZhy{}0.99912361  1.99921136]
[\PYGZhy{}0.99912854  1.99921579]
[\PYGZhy{}0.99913344  1.9992202 ]
[\PYGZhy{}0.99913831  1.99922459]
[\PYGZhy{}0.99914316  1.99922895]
[\PYGZhy{}0.99914797  1.99923328]
[\PYGZhy{}0.99915277  1.9992376 ]
[\PYGZhy{}0.99915753  1.99924188]
[\PYGZhy{}0.99916227  1.99924615]
[\PYGZhy{}0.99916698  1.99925039]
[\PYGZhy{}0.99917166  1.9992546 ]
[\PYGZhy{}0.99917632  1.99925879]
[\PYGZhy{}0.99918095  1.99926296]
[\PYGZhy{}0.99918556  1.99926711]
[\PYGZhy{}0.99919014  1.99927123]
[\PYGZhy{}0.99919469  1.99927533]
[\PYGZhy{}0.99919922  1.9992794 ]
[\PYGZhy{}0.99920373  1.99928345]
[\PYGZhy{}0.9992082   1.99928748]
[\PYGZhy{}0.99921266  1.99929149]
[\PYGZhy{}0.99921708  1.99929547]
[\PYGZhy{}0.99922149  1.99929944]
[\PYGZhy{}0.99922586  1.99930338]
[\PYGZhy{}0.99923022  1.99930729]
[\PYGZhy{}0.99923455  1.99931119]
[\PYGZhy{}0.99923885  1.99931506]
[\PYGZhy{}0.99924313  1.99931891]
[\PYGZhy{}0.99924739  1.99932274]
[\PYGZhy{}0.99925162  1.99932655]
[\PYGZhy{}0.99925583  1.99933034]
[\PYGZhy{}0.99926001  1.99933411]
[\PYGZhy{}0.99926417  1.99933785]
[\PYGZhy{}0.99926831  1.99934157]
[\PYGZhy{}0.99927243  1.99934528]
[\PYGZhy{}0.99927652  1.99934896]
[\PYGZhy{}0.99928059  1.99935262]
[\PYGZhy{}0.99928463  1.99935626]
[\PYGZhy{}0.99928866  1.99935988]
[\PYGZhy{}0.99929266  1.99936348]
[\PYGZhy{}0.99929663  1.99936706]
[\PYGZhy{}0.99930059  1.99937062]
[\PYGZhy{}0.99930452  1.99937416]
[\PYGZhy{}0.99930843  1.99937768]
[\PYGZhy{}0.99931232  1.99938118]
[\PYGZhy{}0.99931619  1.99938466]
[\PYGZhy{}0.99932003  1.99938812]
[\PYGZhy{}0.99932386  1.99939156]
[\PYGZhy{}0.99932766  1.99939498]
[\PYGZhy{}0.99933144  1.99939838]
[\PYGZhy{}0.9993352   1.99940177]
[\PYGZhy{}0.99933894  1.99940513]
[\PYGZhy{}0.99934266  1.99940848]
[\PYGZhy{}0.99934635  1.9994118 ]
[\PYGZhy{}0.99935003  1.99941511]
[\PYGZhy{}0.99935368  1.9994184 ]
[\PYGZhy{}0.99935732  1.99942167]
[\PYGZhy{}0.99936093  1.99942492]
[\PYGZhy{}0.99936453  1.99942816]
[\PYGZhy{}0.9993681   1.99943137]
[\PYGZhy{}0.99937165  1.99943457]
[\PYGZhy{}0.99937519  1.99943775]
[\PYGZhy{}0.9993787   1.99944091]
[\PYGZhy{}0.9993822   1.99944405]
[\PYGZhy{}0.99938567  1.99944718]
[\PYGZhy{}0.99938912  1.99945029]
[\PYGZhy{}0.99939256  1.99945338]
[\PYGZhy{}0.99939598  1.99945645]
[\PYGZhy{}0.99939937  1.99945951]
[\PYGZhy{}0.99940275  1.99946255]
[\PYGZhy{}0.99940611  1.99946557]
[\PYGZhy{}0.99940945  1.99946858]
[\PYGZhy{}0.99941277  1.99947157]
[\PYGZhy{}0.99941607  1.99947454]
\end{sphinxVerbatim}

\begin{sphinxVerbatim}[commandchars=\\\{\}]
[\PYGZhy{}0.99941936  1.99947749]
[\PYGZhy{}0.99942262  1.99948043]
[\PYGZhy{}0.99942587  1.99948335]
[\PYGZhy{}0.9994291   1.99948626]
[\PYGZhy{}0.99943231  1.99948915]
[\PYGZhy{}0.9994355   1.99949202]
[\PYGZhy{}0.99943867  1.99949488]
[\PYGZhy{}0.99944183  1.99949772]
[\PYGZhy{}0.99944497  1.99950054]
[\PYGZhy{}0.99944809  1.99950335]
[\PYGZhy{}0.99945119  1.99950614]
[\PYGZhy{}0.99945428  1.99950892]
[\PYGZhy{}0.99945735  1.99951168]
[\PYGZhy{}0.9994604   1.99951443]
[\PYGZhy{}0.99946344  1.99951716]
[\PYGZhy{}0.99946645  1.99951988]
[\PYGZhy{}0.99946945  1.99952258]
[\PYGZhy{}0.99947244  1.99952526]
[\PYGZhy{}0.9994754   1.99952793]
[\PYGZhy{}0.99947835  1.99953058]
[\PYGZhy{}0.99948129  1.99953322]
[\PYGZhy{}0.9994842   1.99953585]
[\PYGZhy{}0.9994871   1.99953846]
[\PYGZhy{}0.99948999  1.99954105]
[\PYGZhy{}0.99949286  1.99954364]
[\PYGZhy{}0.99949571  1.9995462 ]
[\PYGZhy{}0.99949854  1.99954875]
[\PYGZhy{}0.99950136  1.99955129]
[\PYGZhy{}0.99950417  1.99955381]
[\PYGZhy{}0.99950696  1.99955632]
[\PYGZhy{}0.99950973  1.99955882]
[\PYGZhy{}0.99951249  1.9995613 ]
[\PYGZhy{}0.99951523  1.99956377]
[\PYGZhy{}0.99951795  1.99956622]
[\PYGZhy{}0.99952067  1.99956866]
[\PYGZhy{}0.99952336  1.99957109]
[\PYGZhy{}0.99952604  1.9995735 ]
[\PYGZhy{}0.99952871  1.9995759 ]
[\PYGZhy{}0.99953136  1.99957828]
[\PYGZhy{}0.99953399  1.99958065]
[\PYGZhy{}0.99953661  1.99958301]
[\PYGZhy{}0.99953922  1.99958536]
[\PYGZhy{}0.99954181  1.99958769]
[\PYGZhy{}0.99954439  1.99959001]
[\PYGZhy{}0.99954695  1.99959231]
[\PYGZhy{}0.9995495  1.9995946]
[\PYGZhy{}0.99955203  1.99959688]
[\PYGZhy{}0.99955455  1.99959915]
[\PYGZhy{}0.99955705  1.9996014 ]
[\PYGZhy{}0.99955954  1.99960365]
[\PYGZhy{}0.99956202  1.99960588]
[\PYGZhy{}0.99956448  1.99960809]
[\PYGZhy{}0.99956693  1.9996103 ]
[\PYGZhy{}0.99956937  1.99961249]
[\PYGZhy{}0.99957179  1.99961467]
[\PYGZhy{}0.9995742   1.99961683]
[\PYGZhy{}0.99957659  1.99961899]
[\PYGZhy{}0.99957897  1.99962113]
[\PYGZhy{}0.99958134  1.99962326]
[\PYGZhy{}0.9995837   1.99962538]
[\PYGZhy{}0.99958604  1.99962749]
[\PYGZhy{}0.99958837  1.99962958]
[\PYGZhy{}0.99959068  1.99963166]
[\PYGZhy{}0.99959298  1.99963374]
[\PYGZhy{}0.99959527  1.9996358 ]
[\PYGZhy{}0.99959755  1.99963784]
[\PYGZhy{}0.99959981  1.99963988]
[\PYGZhy{}0.99960206  1.9996419 ]
[\PYGZhy{}0.9996043   1.99964392]
[\PYGZhy{}0.99960652  1.99964592]
[\PYGZhy{}0.99960874  1.99964791]
[\PYGZhy{}0.99961094  1.99964989]
[\PYGZhy{}0.99961312  1.99965186]
[\PYGZhy{}0.9996153   1.99965382]
[\PYGZhy{}0.99961746  1.99965577]
[\PYGZhy{}0.99961961  1.9996577 ]
[\PYGZhy{}0.99962175  1.99965963]
[\PYGZhy{}0.99962388  1.99966154]
[\PYGZhy{}0.999626    1.99966344]
[\PYGZhy{}0.9996281   1.99966534]
[\PYGZhy{}0.99963019  1.99966722]
[\PYGZhy{}0.99963227  1.99966909]
[\PYGZhy{}0.99963434  1.99967095]
[\PYGZhy{}0.99963639  1.9996728 ]
[\PYGZhy{}0.99963844  1.99967464]
[\PYGZhy{}0.99964047  1.99967647]
[\PYGZhy{}0.99964249  1.99967829]
[\PYGZhy{}0.9996445  1.9996801]
[\PYGZhy{}0.9996465  1.9996819]
[\PYGZhy{}0.99964849  1.99968369]
[\PYGZhy{}0.99965047  1.99968547]
[\PYGZhy{}0.99965243  1.99968723]
[\PYGZhy{}0.99965439  1.99968899]
[\PYGZhy{}0.99965633  1.99969074]
[\PYGZhy{}0.99965826  1.99969248]
[\PYGZhy{}0.99966019  1.99969421]
[\PYGZhy{}0.9996621   1.99969593]
[\PYGZhy{}0.999664    1.99969764]
[\PYGZhy{}0.99966589  1.99969934]
[\PYGZhy{}0.99966777  1.99970103]
[\PYGZhy{}0.99966963  1.99970271]
[\PYGZhy{}0.99967149  1.99970438]
[\PYGZhy{}0.99967334  1.99970605]
[\PYGZhy{}0.99967518  1.9997077 ]
[\PYGZhy{}0.999677    1.99970934]
[\PYGZhy{}0.99967882  1.99971098]
[\PYGZhy{}0.99968063  1.9997126 ]
[\PYGZhy{}0.99968242  1.99971422]
[\PYGZhy{}0.99968421  1.99971583]
[\PYGZhy{}0.99968598  1.99971742]
[\PYGZhy{}0.99968775  1.99971901]
[\PYGZhy{}0.99968951  1.99972059]
[\PYGZhy{}0.99969125  1.99972217]
[\PYGZhy{}0.99969299  1.99972373]
[\PYGZhy{}0.99969471  1.99972528]
[\PYGZhy{}0.99969643  1.99972683]
[\PYGZhy{}0.99969814  1.99972836]
[\PYGZhy{}0.99969984  1.99972989]
[\PYGZhy{}0.99970152  1.99973141]
[\PYGZhy{}0.9997032   1.99973292]
[\PYGZhy{}0.99970487  1.99973442]
[\PYGZhy{}0.99970653  1.99973591]
[\PYGZhy{}0.99970818  1.9997374 ]
[\PYGZhy{}0.99970982  1.99973888]
[\PYGZhy{}0.99971145  1.99974035]
[\PYGZhy{}0.99971308  1.99974181]
[\PYGZhy{}0.99971469  1.99974326]
[\PYGZhy{}0.99971629  1.9997447 ]
[\PYGZhy{}0.99971789  1.99974614]
[\PYGZhy{}0.99971948  1.99974756]
[\PYGZhy{}0.99972105  1.99974898]
[\PYGZhy{}0.99972262  1.9997504 ]
[\PYGZhy{}0.99972418  1.9997518 ]
[\PYGZhy{}0.99972573  1.99975319]
[\PYGZhy{}0.99972728  1.99975458]
[\PYGZhy{}0.99972881  1.99975596]
[\PYGZhy{}0.99973033  1.99975734]
[\PYGZhy{}0.99973185  1.9997587 ]
[\PYGZhy{}0.99973336  1.99976006]
[\PYGZhy{}0.99973486  1.99976141]
[\PYGZhy{}0.99973635  1.99976275]
[\PYGZhy{}0.99973783  1.99976408]
[\PYGZhy{}0.99973931  1.99976541]
[\PYGZhy{}0.99974077  1.99976673]
[\PYGZhy{}0.99974223  1.99976804]
[\PYGZhy{}0.99974368  1.99976934]
[\PYGZhy{}0.99974512  1.99977064]
[\PYGZhy{}0.99974655  1.99977193]
[\PYGZhy{}0.99974798  1.99977321]
[\PYGZhy{}0.9997494   1.99977449]
[\PYGZhy{}0.99975081  1.99977576]
[\PYGZhy{}0.99975221  1.99977702]
[\PYGZhy{}0.9997536   1.99977827]
[\PYGZhy{}0.99975499  1.99977952]
[\PYGZhy{}0.99975636  1.99978076]
[\PYGZhy{}0.99975773  1.99978199]
[\PYGZhy{}0.9997591   1.99978322]
[\PYGZhy{}0.99976045  1.99978444]
[\PYGZhy{}0.9997618   1.99978565]
[\PYGZhy{}0.99976314  1.99978685]
[\PYGZhy{}0.99976447  1.99978805]
[\PYGZhy{}0.99976579  1.99978925]
[\PYGZhy{}0.99976711  1.99979043]
[\PYGZhy{}0.99976842  1.99979161]
[\PYGZhy{}0.99976972  1.99979278]
[\PYGZhy{}0.99977102  1.99979395]
[\PYGZhy{}0.99977231  1.9997951 ]
[\PYGZhy{}0.99977359  1.99979626]
[\PYGZhy{}0.99977486  1.9997974 ]
[\PYGZhy{}0.99977613  1.99979854]
[\PYGZhy{}0.99977739  1.99979967]
[\PYGZhy{}0.99977864  1.9998008 ]
[\PYGZhy{}0.99977988  1.99980192]
[\PYGZhy{}0.99978112  1.99980304]
[\PYGZhy{}0.99978235  1.99980414]
[\PYGZhy{}0.99978357  1.99980524]
[\PYGZhy{}0.99978479  1.99980634]
[\PYGZhy{}0.999786    1.99980743]
[\PYGZhy{}0.99978721  1.99980851]
[\PYGZhy{}0.9997884   1.99980959]
[\PYGZhy{}0.99978959  1.99981066]
[\PYGZhy{}0.99979077  1.99981172]
[\PYGZhy{}0.99979195  1.99981278]
[\PYGZhy{}0.99979312  1.99981384]
[\PYGZhy{}0.99979428  1.99981488]
[\PYGZhy{}0.99979544  1.99981592]
[\PYGZhy{}0.99979659  1.99981696]
[\PYGZhy{}0.99979774  1.99981799]
[\PYGZhy{}0.99979887  1.99981901]
[\PYGZhy{}0.9998      1.99982003]
[\PYGZhy{}0.99980113  1.99982104]
[\PYGZhy{}0.99980225  1.99982205]
[\PYGZhy{}0.99980336  1.99982305]
[\PYGZhy{}0.99980447  1.99982404]
[\PYGZhy{}0.99980556  1.99982503]
[\PYGZhy{}0.99980666  1.99982602]
[\PYGZhy{}0.99980775  1.999827  ]
[\PYGZhy{}0.99980883  1.99982797]
[\PYGZhy{}0.9998099   1.99982894]
[\PYGZhy{}0.99981097  1.9998299 ]
[\PYGZhy{}0.99981203  1.99983085]
[\PYGZhy{}0.99981309  1.99983181]
[\PYGZhy{}0.99981414  1.99983275]
[\PYGZhy{}0.99981519  1.99983369]
[\PYGZhy{}0.99981623  1.99983463]
[\PYGZhy{}0.99981726  1.99983556]
[\PYGZhy{}0.99981829  1.99983648]
[\PYGZhy{}0.99981931  1.9998374 ]
[\PYGZhy{}0.99982033  1.99983832]
[\PYGZhy{}0.99982134  1.99983923]
[\PYGZhy{}0.99982234  1.99984013]
[\PYGZhy{}0.99982334  1.99984103]
[\PYGZhy{}0.99982433  1.99984192]
[\PYGZhy{}0.99982532  1.99984281]
[\PYGZhy{}0.9998263  1.9998437]
[\PYGZhy{}0.99982728  1.99984457]
[\PYGZhy{}0.99982825  1.99984545]
[\PYGZhy{}0.99982922  1.99984632]
[\PYGZhy{}0.99983018  1.99984718]
[\PYGZhy{}0.99983113  1.99984804]
[\PYGZhy{}0.99983208  1.9998489 ]
[\PYGZhy{}0.99983303  1.99984975]
[\PYGZhy{}0.99983397  1.99985059]
[\PYGZhy{}0.9998349   1.99985143]
[\PYGZhy{}0.99983583  1.99985227]
[\PYGZhy{}0.99983675  1.9998531 ]
[\PYGZhy{}0.99983767  1.99985392]
[\PYGZhy{}0.99983858  1.99985474]
[\PYGZhy{}0.99983949  1.99985556]
[\PYGZhy{}0.99984039  1.99985637]
[\PYGZhy{}0.99984129  1.99985718]
[\PYGZhy{}0.99984218  1.99985798]
[\PYGZhy{}0.99984307  1.99985878]
[\PYGZhy{}0.99984395  1.99985958]
[\PYGZhy{}0.99984483  1.99986037]
[\PYGZhy{}0.9998457   1.99986115]
[\PYGZhy{}0.99984657  1.99986193]
[\PYGZhy{}0.99984743  1.99986271]
[\PYGZhy{}0.99984829  1.99986348]
[\PYGZhy{}0.99984914  1.99986425]
[\PYGZhy{}0.99984999  1.99986501]
[\PYGZhy{}0.99985084  1.99986577]
[\PYGZhy{}0.99985167  1.99986653]
[\PYGZhy{}0.99985251  1.99986728]
[\PYGZhy{}0.99985334  1.99986802]
[\PYGZhy{}0.99985416  1.99986877]
[\PYGZhy{}0.99985498  1.9998695 ]
[\PYGZhy{}0.9998558   1.99987024]
[\PYGZhy{}0.99985661  1.99987097]
[\PYGZhy{}0.99985742  1.99987169]
[\PYGZhy{}0.99985822  1.99987241]
\end{sphinxVerbatim}

\sphinxstylestrong{\(N\), \(k\)值和初始权数的确定}

在开始调整权数时,首先要确定权数个数\(N\)和学习常数\(k\) 。一般说来,当时间序列的观测值呈季节变动时, \(N\) 应取季节性长度值。如序列以一年为周期进行季节变动时,若数据是月度的,则取\(N = 12\),若季节是季度的,则取\(N = 4\)。如果时间序列无明显的周期变动,则可用自相关系数法来确定,即取\(N\)为最高自相关系数的滞后时期。

\(k\)的取值一般可定为\(1/N\) ,也可以用不同的\(k\)值来进行计算,以确定一个能使\(S\)最小的\(k\)值。

初始权数的确定也很重要,如无其它依据,也可用\(1/ N\) 作为初始权系数用,即
\$\(
w_{i}=\frac{1}{N}(i=1,2, \cdots, N)
\)\$

自适应滤波法有两个明显的优点:
\begin{itemize}
\item {} 
一是技术比较简单,可根据预测意图来选择权数的个数和学习常数,以控制预测。也可以由计算机自动选定。

\item {} 
二是它使用了全部历史数据来寻求最佳权系数,并随数据轨迹的变化而不断更新权数,从而不断改进预测。由于自适应滤波法的预测模型简单,又可以在计算机上对数据进行处理,所以这种预测方法应用较为广泛。

\end{itemize}


\section{预测方法总结}
\label{\detokenize{docs/prediction_model:id16}}
\sphinxstylestrong{回归分析法:} 适合中、小样本预测
\sphinxstylestrong{时间序列方法:} 适合中、大样本的随机因素或周期特征的未来趋势未来预测


\section{练习}
\label{\detokenize{docs/prediction_model:id17}}\begin{enumerate}
\sphinxsetlistlabels{\arabic}{enumi}{enumii}{}{.}%
\item {} 
用如下代码生成一个包含了随机变动的正弦函数曲线,请你使用多项式拟合方法,研究用二次,三次,以及更高次函数拟合的情况,给你你认为的最好的拟合方法。

\end{enumerate}

\begin{sphinxVerbatim}[commandchars=\\\{\}]
\PYG{k+kn}{import} \PYG{n+nn}{numpy} \PYG{k}{as} \PYG{n+nn}{np}
\PYG{k+kn}{import} \PYG{n+nn}{matplotlib}\PYG{n+nn}{.}\PYG{n+nn}{pyplot} \PYG{k}{as} \PYG{n+nn}{plt}
\PYG{o}{\PYGZpc{}}\PYG{k}{matplotlib} inline
\PYG{n}{x} \PYG{o}{=} \PYG{p}{[}\PYG{l+m+mf}{0.1} \PYG{o}{*} \PYG{n}{i} \PYG{k}{for} \PYG{n}{i} \PYG{o+ow}{in} \PYG{n+nb}{range}\PYG{p}{(}\PYG{l+m+mi}{100}\PYG{p}{)}\PYG{p}{]}
\PYG{n}{y} \PYG{o}{=} \PYG{p}{[}\PYG{n}{np}\PYG{o}{.}\PYG{n}{sin}\PYG{p}{(}\PYG{n}{t}\PYG{p}{)} \PYG{o}{+} \PYG{n}{np}\PYG{o}{.}\PYG{n}{random}\PYG{o}{.}\PYG{n}{random}\PYG{p}{(}\PYG{p}{)} \PYG{k}{for} \PYG{n}{t} \PYG{o+ow}{in} \PYG{n}{x}\PYG{p}{]}
\PYG{n}{plt}\PYG{o}{.}\PYG{n}{scatter}\PYG{p}{(}\PYG{n}{x}\PYG{p}{,}\PYG{n}{y}\PYG{p}{)}
\end{sphinxVerbatim}

\begin{sphinxVerbatim}[commandchars=\\\{\}]
\PYGZlt{}matplotlib.collections.PathCollection at 0x12af2b810\PYGZgt{}
\end{sphinxVerbatim}

\noindent\sphinxincludegraphics{{prediction_model_65_1}.png}


\chapter{作业解答}
\label{\detokenize{docs/answers:id1}}\label{\detokenize{docs/answers::doc}}

\section{规划模型}
\label{\detokenize{docs/answers:id2}}\begin{itemize}
\item {} 
请使用Python \sphinxcode{\sphinxupquote{scipy}}库 的\sphinxcode{\sphinxupquote{optimize.linprog}}方法,求解以下线性规划问题,并通过图解法验证。

\end{itemize}
\begin{equation*}
\begin{split}
\begin{array}{l}
&{\max z= 4x_{1}+ 3x_{2}} \\
&\text { s.t. }{\quad\left\{\begin{array}{l}
{2x_{1}+ x_{2} \leq 10} \\ 
{x_{1}+ x_{2} \leq 8} \\ 
{x_{1}, x_{2} \geq 0}
\end{array}\right.}\end{array}
\end{split}
\end{equation*}
\begin{sphinxVerbatim}[commandchars=\\\{\}]
\PYG{c+c1}{\PYGZsh{}导入相关库}
\PYG{k+kn}{import} \PYG{n+nn}{numpy} \PYG{k}{as} \PYG{n+nn}{np}
\PYG{k+kn}{from} \PYG{n+nn}{scipy} \PYG{k+kn}{import} \PYG{n}{optimize} \PYG{k}{as} \PYG{n}{op}

\PYG{c+c1}{\PYGZsh{}定义决策变量范围}
\PYG{n}{x1}\PYG{o}{=}\PYG{p}{(}\PYG{l+m+mi}{0}\PYG{p}{,}\PYG{k+kc}{None}\PYG{p}{)}
\PYG{n}{x2}\PYG{o}{=}\PYG{p}{(}\PYG{l+m+mi}{0}\PYG{p}{,}\PYG{k+kc}{None}\PYG{p}{)}

\PYG{c+c1}{\PYGZsh{}定义目标函数系数(请注意这里是求最大值,而linprog默认求最小值,因此我们需要加一个符号)}
\PYG{n}{c}\PYG{o}{=}\PYG{n}{np}\PYG{o}{.}\PYG{n}{array}\PYG{p}{(}\PYG{p}{[}\PYG{o}{\PYGZhy{}}\PYG{l+m+mi}{4}\PYG{p}{,}\PYG{o}{\PYGZhy{}}\PYG{l+m+mi}{3}\PYG{p}{]}\PYG{p}{)} 

\PYG{c+c1}{\PYGZsh{}定义约束条件系数}
\PYG{n}{A\PYGZus{}ub}\PYG{o}{=}\PYG{n}{np}\PYG{o}{.}\PYG{n}{array}\PYG{p}{(}\PYG{p}{[}\PYG{p}{[}\PYG{l+m+mi}{2}\PYG{p}{,}\PYG{l+m+mi}{1}\PYG{p}{]}\PYG{p}{,}\PYG{p}{[}\PYG{l+m+mi}{1}\PYG{p}{,}\PYG{l+m+mi}{1}\PYG{p}{]}\PYG{p}{]}\PYG{p}{)}
\PYG{n}{B\PYGZus{}ub}\PYG{o}{=}\PYG{n}{np}\PYG{o}{.}\PYG{n}{array}\PYG{p}{(}\PYG{p}{[}\PYG{l+m+mi}{10}\PYG{p}{,}\PYG{l+m+mi}{8}\PYG{p}{]}\PYG{p}{)}

\PYG{c+c1}{\PYGZsh{}求解}
\PYG{n}{res}\PYG{o}{=}\PYG{n}{op}\PYG{o}{.}\PYG{n}{linprog}\PYG{p}{(}\PYG{n}{c}\PYG{p}{,}\PYG{n}{A\PYGZus{}ub}\PYG{p}{,}\PYG{n}{B\PYGZus{}ub}\PYG{p}{,}\PYG{n}{bounds}\PYG{o}{=}\PYG{p}{(}\PYG{n}{x1}\PYG{p}{,}\PYG{n}{x2}\PYG{p}{)}\PYG{p}{)}
\PYG{n}{res}
\end{sphinxVerbatim}

\begin{sphinxVerbatim}[commandchars=\\\{\}]
     con: array([], dtype=float64)
     fun: \PYGZhy{}25.99999991344822
 message: \PYGZsq{}Optimization terminated successfully.\PYGZsq{}
     nit: 4
   slack: array([4.63374761e\PYGZhy{}08, 2.01071497e\PYGZhy{}08])
  status: 0
 success: True
       x: array([1.99999997, 6.00000001])
\end{sphinxVerbatim}
\begin{itemize}
\item {} 
请使用Python \sphinxcode{\sphinxupquote{scipy}}库 的\sphinxcode{\sphinxupquote{optimize.minimize}}方法,求解以下非线性规划问题

\end{itemize}
\begin{equation*}
\begin{split}
\begin{array}{l}
&{\min z= x_{1}^2 + x_{2}^2 +x_{3}^2} \\
&\text { s.t. }{\quad\left\{\begin{array}{l}
{x_1+x_2 + x_3\geq9 } \\ 
{ x_{1}, x_{2},x_3 \geq 0}
\end{array}\right.}\end{array}
\end{split}
\end{equation*}
\begin{sphinxVerbatim}[commandchars=\\\{\}]
\PYG{k+kn}{import} \PYG{n+nn}{numpy} \PYG{k}{as} \PYG{n+nn}{np}
\PYG{k+kn}{from} \PYG{n+nn}{scipy}\PYG{n+nn}{.}\PYG{n+nn}{optimize} \PYG{k+kn}{import} \PYG{n}{minimize}

\PYG{c+c1}{\PYGZsh{} 定义目标函数}
\PYG{k}{def} \PYG{n+nf}{objective}\PYG{p}{(}\PYG{n}{x}\PYG{p}{)}\PYG{p}{:}
    \PYG{k}{return} \PYG{p}{(}\PYG{n}{x}\PYG{p}{[}\PYG{l+m+mi}{0}\PYG{p}{]} \PYG{o}{*}\PYG{o}{*} \PYG{l+m+mi}{2} \PYG{o}{+} \PYG{n}{x}\PYG{p}{[}\PYG{l+m+mi}{1}\PYG{p}{]}\PYG{o}{*}\PYG{o}{*}\PYG{l+m+mi}{2} \PYG{o}{+} \PYG{n}{x}\PYG{p}{[}\PYG{l+m+mi}{2}\PYG{p}{]}\PYG{o}{*}\PYG{o}{*}\PYG{l+m+mi}{2}\PYG{p}{)}

\PYG{c+c1}{\PYGZsh{} 定义约束条件}
\PYG{k}{def} \PYG{n+nf}{constraint1}\PYG{p}{(}\PYG{n}{x}\PYG{p}{)}\PYG{p}{:}
    \PYG{k}{return} \PYG{p}{(}\PYG{n}{x}\PYG{p}{[}\PYG{l+m+mi}{0}\PYG{p}{]}  \PYG{o}{+} \PYG{n}{x}\PYG{p}{[}\PYG{l+m+mi}{1}\PYG{p}{]} \PYG{o}{+} \PYG{n}{x}\PYG{p}{[}\PYG{l+m+mi}{2}\PYG{p}{]}  \PYG{o}{\PYGZhy{}} \PYG{l+m+mi}{9}\PYG{p}{)}  \PYG{c+c1}{\PYGZsh{} 不等约束1}

\PYG{c+c1}{\PYGZsh{} 汇总约束条件}
\PYG{n}{con1} \PYG{o}{=} \PYG{p}{\PYGZob{}}\PYG{l+s+s1}{\PYGZsq{}}\PYG{l+s+s1}{type}\PYG{l+s+s1}{\PYGZsq{}}\PYG{p}{:} \PYG{l+s+s1}{\PYGZsq{}}\PYG{l+s+s1}{ineq}\PYG{l+s+s1}{\PYGZsq{}}\PYG{p}{,} \PYG{l+s+s1}{\PYGZsq{}}\PYG{l+s+s1}{fun}\PYG{l+s+s1}{\PYGZsq{}}\PYG{p}{:} \PYG{n}{constraint1}\PYG{p}{\PYGZcb{}}
\PYG{n}{cons} \PYG{o}{=} \PYG{p}{(}\PYG{p}{[}\PYG{n}{con1}\PYG{p}{]}\PYG{p}{)}  

\PYG{c+c1}{\PYGZsh{} 决策变量的符号约束}
\PYG{n}{b} \PYG{o}{=} \PYG{p}{(}\PYG{l+m+mf}{0.0}\PYG{p}{,} \PYG{k+kc}{None}\PYG{p}{)} \PYG{c+c1}{\PYGZsh{}即决策变量的取值范围为大于等于0}
\PYG{n}{bnds} \PYG{o}{=} \PYG{p}{(}\PYG{n}{b}\PYG{p}{,} \PYG{n}{b} \PYG{p}{,}\PYG{n}{b}\PYG{p}{)} 

\PYG{c+c1}{\PYGZsh{}定义初始值}
\PYG{n}{x0}\PYG{o}{=}\PYG{n}{np}\PYG{o}{.}\PYG{n}{array}\PYG{p}{(}\PYG{p}{[}\PYG{l+m+mi}{0}\PYG{p}{,} \PYG{l+m+mi}{0}\PYG{p}{,} \PYG{l+m+mi}{0}\PYG{p}{]}\PYG{p}{)} 

\PYG{c+c1}{\PYGZsh{} 求解}
\PYG{n}{solution} \PYG{o}{=} \PYG{n}{minimize}\PYG{p}{(}\PYG{n}{objective}\PYG{p}{,} \PYG{n}{x0}\PYG{p}{,} \PYG{n}{method}\PYG{o}{=}\PYG{l+s+s1}{\PYGZsq{}}\PYG{l+s+s1}{SLSQP}\PYG{l+s+s1}{\PYGZsq{}}\PYG{p}{,} \PYGZbs{}
                    \PYG{n}{bounds}\PYG{o}{=}\PYG{n}{bnds}\PYG{p}{,} \PYG{n}{constraints}\PYG{o}{=}\PYG{n}{cons}\PYG{p}{)}
                    
\PYG{n}{x} \PYG{o}{=} \PYG{n}{solution}\PYG{o}{.}\PYG{n}{x}

\PYG{c+c1}{\PYGZsh{} 打印结果}
\PYG{n+nb}{print}\PYG{p}{(}\PYG{l+s+s1}{\PYGZsq{}}\PYG{l+s+s1}{目标值: }\PYG{l+s+s1}{\PYGZsq{}} \PYG{o}{+} \PYG{n+nb}{str}\PYG{p}{(}\PYG{n}{objective}\PYG{p}{(}\PYG{n}{x}\PYG{p}{)}\PYG{p}{)}\PYG{p}{)}
\PYG{n+nb}{print}\PYG{p}{(}\PYG{l+s+s1}{\PYGZsq{}}\PYG{l+s+s1}{最优解为}\PYG{l+s+s1}{\PYGZsq{}}\PYG{p}{)}
\PYG{n+nb}{print}\PYG{p}{(}\PYG{l+s+s1}{\PYGZsq{}}\PYG{l+s+s1}{x1 = }\PYG{l+s+s1}{\PYGZsq{}} \PYG{o}{+} \PYG{n+nb}{str}\PYG{p}{(}\PYG{n+nb}{round}\PYG{p}{(}\PYG{n}{x}\PYG{p}{[}\PYG{l+m+mi}{0}\PYG{p}{]}\PYG{p}{,}\PYG{l+m+mi}{2}\PYG{p}{)}\PYG{p}{)}\PYG{p}{)}
\PYG{n+nb}{print}\PYG{p}{(}\PYG{l+s+s1}{\PYGZsq{}}\PYG{l+s+s1}{x2 = }\PYG{l+s+s1}{\PYGZsq{}} \PYG{o}{+} \PYG{n+nb}{str}\PYG{p}{(}\PYG{n+nb}{round}\PYG{p}{(}\PYG{n}{x}\PYG{p}{[}\PYG{l+m+mi}{1}\PYG{p}{]}\PYG{p}{,}\PYG{l+m+mi}{2}\PYG{p}{)}\PYG{p}{)}\PYG{p}{)}
\PYG{n+nb}{print}\PYG{p}{(}\PYG{l+s+s1}{\PYGZsq{}}\PYG{l+s+s1}{x3 = }\PYG{l+s+s1}{\PYGZsq{}} \PYG{o}{+} \PYG{n+nb}{str}\PYG{p}{(}\PYG{n+nb}{round}\PYG{p}{(}\PYG{n}{x}\PYG{p}{[}\PYG{l+m+mi}{2}\PYG{p}{]}\PYG{p}{,}\PYG{l+m+mi}{2}\PYG{p}{)}\PYG{p}{)}\PYG{p}{)}
\PYG{n}{solution}
\end{sphinxVerbatim}

\begin{sphinxVerbatim}[commandchars=\\\{\}]
目标值: 26.99999999999998
最优解为
x1 = 3.0
x2 = 3.0
x3 = 3.0
\end{sphinxVerbatim}

\begin{sphinxVerbatim}[commandchars=\\\{\}]
     fun: 26.99999999999998
     jac: array([6., 6., 6.])
 message: \PYGZsq{}Optimization terminated successfully\PYGZsq{}
    nfev: 13
     nit: 3
    njev: 3
  status: 0
 success: True
       x: array([3., 3., 3.])
\end{sphinxVerbatim}
\begin{itemize}
\item {} 
某农场 I,II,III 等耕地的面积分别为 \(100 hm^2\)、\(300 hm^2\) 和 \(200 hm^2\),计划种植水稻、大豆和玉米,要求三种作物的最低收获量分别为\(190000kg\)、\(130000kg\)和\(350000kg\)。I,II,III 等耕地种植三种作物的单产如下表所示。
若三种作物的售价分别为水稻1.20元/kg,大豆1.50元/kg,玉米0.80元/kg。那么,
\begin{itemize}
\item {} 
如何制订种植计划才能使总产量最大?

\item {} 
如何制订种植计划才能使总产值最大?
\sphinxstylestrong{要求:写出规划问题的标准型,并合理采用本课程学到的知识,进行求解。}

\end{itemize}

\end{itemize}


\begin{savenotes}\sphinxattablestart
\centering
\begin{tabulary}{\linewidth}[t]{|T|T|T|T|}
\hline


&\sphinxstyletheadfamily 
I等耕地
&\sphinxstyletheadfamily 
II等耕地
&\sphinxstyletheadfamily 
III等耕地
\\
\hline
水稻
&
11000
&
9500
&
9000
\\
\hline
大豆
&
8000
&
6800
&
6000
\\
\hline
玉米
&
14000
&
12000
&
10000
\\
\hline
\end{tabulary}
\par
\sphinxattableend\end{savenotes}

\begin{sphinxVerbatim}[commandchars=\\\{\}]
\PYG{c+c1}{\PYGZsh{}导入相关库}
\PYG{k+kn}{import} \PYG{n+nn}{numpy} \PYG{k}{as} \PYG{n+nn}{np}
\PYG{k+kn}{from} \PYG{n+nn}{scipy} \PYG{k+kn}{import} \PYG{n}{optimize} \PYG{k}{as} \PYG{n}{op}

\PYG{c+c1}{\PYGZsh{}定义决策变量范围}
\PYG{n}{x11}\PYG{o}{=}\PYG{p}{(}\PYG{l+m+mi}{0}\PYG{p}{,}\PYG{k+kc}{None}\PYG{p}{)}
\PYG{n}{x12}\PYG{o}{=}\PYG{p}{(}\PYG{l+m+mi}{0}\PYG{p}{,}\PYG{k+kc}{None}\PYG{p}{)}
\PYG{n}{x13}\PYG{o}{=}\PYG{p}{(}\PYG{l+m+mi}{0}\PYG{p}{,}\PYG{k+kc}{None}\PYG{p}{)}

\PYG{n}{x21}\PYG{o}{=}\PYG{p}{(}\PYG{l+m+mi}{0}\PYG{p}{,}\PYG{k+kc}{None}\PYG{p}{)}
\PYG{n}{x22}\PYG{o}{=}\PYG{p}{(}\PYG{l+m+mi}{0}\PYG{p}{,}\PYG{k+kc}{None}\PYG{p}{)}
\PYG{n}{x23}\PYG{o}{=}\PYG{p}{(}\PYG{l+m+mi}{0}\PYG{p}{,}\PYG{k+kc}{None}\PYG{p}{)}

\PYG{n}{x31}\PYG{o}{=}\PYG{p}{(}\PYG{l+m+mi}{0}\PYG{p}{,}\PYG{k+kc}{None}\PYG{p}{)}
\PYG{n}{x32}\PYG{o}{=}\PYG{p}{(}\PYG{l+m+mi}{0}\PYG{p}{,}\PYG{k+kc}{None}\PYG{p}{)}
\PYG{n}{x33}\PYG{o}{=}\PYG{p}{(}\PYG{l+m+mi}{0}\PYG{p}{,}\PYG{k+kc}{None}\PYG{p}{)}
\PYG{n}{bounds}\PYG{o}{=}\PYG{p}{(}\PYG{n}{x11}\PYG{p}{,}\PYG{n}{x12}\PYG{p}{,}\PYG{n}{x13}\PYG{p}{,}\PYG{n}{x21}\PYG{p}{,}\PYG{n}{x22}\PYG{p}{,}\PYG{n}{x23}\PYG{p}{,}\PYG{n}{x31}\PYG{p}{,}\PYG{n}{x32}\PYG{p}{,}\PYG{n}{x33}\PYG{p}{)}

\PYG{c+c1}{\PYGZsh{}定义目标函数系数(请注意这里是求最大值,而linprog默认求最小值,因此我们需要加一个负号)}
\PYG{n}{c}\PYG{o}{=}\PYG{n}{np}\PYG{o}{.}\PYG{n}{array}\PYG{p}{(}\PYG{p}{[}\PYG{o}{\PYGZhy{}}\PYG{l+m+mi}{11}\PYG{o}{*}\PYG{l+m+mf}{1.2}\PYG{p}{,}\PYG{o}{\PYGZhy{}}\PYG{l+m+mi}{8}\PYG{o}{*}\PYG{l+m+mf}{1.5}\PYG{p}{,}\PYG{o}{\PYGZhy{}}\PYG{l+m+mi}{14}\PYG{o}{*}\PYG{l+m+mf}{0.8}\PYG{p}{,}\PYG{o}{\PYGZhy{}}\PYG{l+m+mf}{9.5}\PYG{o}{*}\PYG{l+m+mf}{1.2}\PYG{p}{,}\PYG{o}{\PYGZhy{}}\PYG{l+m+mf}{6.9}\PYG{o}{*}\PYG{l+m+mf}{1.5}\PYG{p}{,}\PYG{o}{\PYGZhy{}}\PYG{l+m+mi}{12}\PYG{o}{*}\PYG{l+m+mf}{0.8}\PYG{p}{,}\PYG{o}{\PYGZhy{}}\PYG{l+m+mi}{9}\PYG{o}{*}\PYG{l+m+mf}{1.2}\PYG{p}{,}\PYG{o}{\PYGZhy{}}\PYG{l+m+mi}{6}\PYG{o}{*}\PYG{l+m+mf}{1.5}\PYG{p}{,}\PYG{o}{\PYGZhy{}}\PYG{l+m+mi}{10}\PYG{o}{*}\PYG{l+m+mf}{0.8}\PYG{p}{]}\PYG{p}{)} 

\PYG{c+c1}{\PYGZsh{}定义不等式约束条件系数}
\PYG{n}{A} \PYG{o}{=} \PYG{o}{\PYGZhy{}} \PYG{n}{np}\PYG{o}{.}\PYG{n}{array}\PYG{p}{(}\PYG{p}{[}\PYG{p}{[}\PYG{l+m+mi}{11}\PYG{p}{,}\PYG{l+m+mi}{0}\PYG{p}{,}\PYG{l+m+mi}{0}\PYG{p}{,}\PYG{l+m+mf}{9.5}\PYG{p}{,}\PYG{l+m+mi}{0}\PYG{p}{,}\PYG{l+m+mi}{0}\PYG{p}{,}\PYG{l+m+mi}{9}\PYG{p}{,}\PYG{l+m+mi}{0}\PYG{p}{,}\PYG{l+m+mi}{0}\PYG{p}{]}\PYG{p}{,}\PYG{p}{[}\PYG{l+m+mi}{0}\PYG{p}{,}\PYG{l+m+mi}{8}\PYG{p}{,}\PYG{l+m+mi}{0}\PYG{p}{,}\PYG{l+m+mi}{0}\PYG{p}{,}\PYG{l+m+mf}{6.9}\PYG{p}{,}\PYG{l+m+mi}{0}\PYG{p}{,}\PYG{l+m+mi}{0}\PYG{p}{,}\PYG{l+m+mi}{6}\PYG{p}{,}\PYG{l+m+mi}{0}\PYG{p}{]}\PYG{p}{,}\PYG{p}{[}\PYG{l+m+mi}{0}\PYG{p}{,}\PYG{l+m+mi}{0}\PYG{p}{,}\PYG{l+m+mi}{14}\PYG{p}{,}\PYG{l+m+mi}{0}\PYG{p}{,}\PYG{l+m+mi}{0}\PYG{p}{,}\PYG{l+m+mi}{12}\PYG{p}{,}\PYG{l+m+mi}{0}\PYG{p}{,}\PYG{l+m+mi}{0}\PYG{p}{,}\PYG{l+m+mi}{10}\PYG{p}{]}\PYG{p}{]}\PYG{p}{)}
\PYG{n}{b} \PYG{o}{=} \PYG{o}{\PYGZhy{}} \PYG{n}{np}\PYG{o}{.}\PYG{n}{array}\PYG{p}{(}\PYG{p}{[}\PYG{l+m+mi}{190}\PYG{p}{,}\PYG{l+m+mi}{130}\PYG{p}{,}\PYG{l+m+mi}{350}\PYG{p}{]}\PYG{p}{)}

\PYG{c+c1}{\PYGZsh{}定义等式约束条件系数}
\PYG{n}{A\PYGZus{}eq} \PYG{o}{=} \PYG{n}{np}\PYG{o}{.}\PYG{n}{array}\PYG{p}{(}\PYG{p}{[}\PYG{p}{[}\PYG{l+m+mi}{1}\PYG{p}{,}\PYG{l+m+mi}{1}\PYG{p}{,}\PYG{l+m+mi}{1}\PYG{p}{,}\PYG{l+m+mi}{0}\PYG{p}{,}\PYG{l+m+mi}{0}\PYG{p}{,}\PYG{l+m+mi}{0}\PYG{p}{,}\PYG{l+m+mi}{0}\PYG{p}{,}\PYG{l+m+mi}{0}\PYG{p}{,}\PYG{l+m+mi}{0}\PYG{p}{]}\PYG{p}{,}\PYG{p}{[}\PYG{l+m+mi}{0}\PYG{p}{,}\PYG{l+m+mi}{0}\PYG{p}{,}\PYG{l+m+mi}{0}\PYG{p}{,}\PYG{l+m+mi}{1}\PYG{p}{,}\PYG{l+m+mi}{1}\PYG{p}{,}\PYG{l+m+mi}{1}\PYG{p}{,}\PYG{l+m+mi}{0}\PYG{p}{,}\PYG{l+m+mi}{0}\PYG{p}{,}\PYG{l+m+mi}{0}\PYG{p}{]}\PYG{p}{,}\PYG{p}{[}\PYG{l+m+mi}{0}\PYG{p}{,}\PYG{l+m+mi}{0}\PYG{p}{,}\PYG{l+m+mi}{0}\PYG{p}{,}\PYG{l+m+mi}{0}\PYG{p}{,}\PYG{l+m+mi}{0}\PYG{p}{,}\PYG{l+m+mi}{0}\PYG{p}{,}\PYG{l+m+mi}{1}\PYG{p}{,}\PYG{l+m+mi}{1}\PYG{p}{,}\PYG{l+m+mi}{1}\PYG{p}{]}\PYG{p}{]}\PYG{p}{)}
\PYG{n}{b\PYGZus{}eq} \PYG{o}{=} \PYG{n}{np}\PYG{o}{.}\PYG{n}{array}\PYG{p}{(}\PYG{p}{[}\PYG{l+m+mi}{100}\PYG{p}{,}\PYG{l+m+mi}{300}\PYG{p}{,}\PYG{l+m+mi}{200}\PYG{p}{]}\PYG{p}{)}


\PYG{c+c1}{\PYGZsh{}求解}
\PYG{n}{res}\PYG{o}{=}\PYG{n}{op}\PYG{o}{.}\PYG{n}{linprog}\PYG{p}{(}\PYG{n}{c}\PYG{p}{,}\PYG{n}{A}\PYG{p}{,}\PYG{n}{b}\PYG{p}{,}\PYG{n}{A\PYGZus{}eq}\PYG{p}{,}\PYG{n}{b\PYGZus{}eq}\PYG{p}{,}\PYG{n}{bounds}\PYG{o}{=}\PYG{n}{bounds}\PYG{p}{)}
\PYG{n}{res}
\end{sphinxVerbatim}

\begin{sphinxVerbatim}[commandchars=\\\{\}]
     con: array([1.00512096e\PYGZhy{}08, 3.07749133e\PYGZhy{}08, 2.04131823e\PYGZhy{}08])
     fun: \PYGZhy{}6830.499999040153
 message: \PYGZsq{}Optimization terminated successfully.\PYGZsq{}
     nit: 9
   slack: array([5.10625000e+03, 2.34581904e\PYGZhy{}07, 3.21025993e\PYGZhy{}07])
  status: 0
 success: True
       x: array([5.87500046e+01, 1.62499956e+01, 2.49999998e+01, 2.99999995e+02,
       5.06481640e\PYGZhy{}06, 2.58879580e\PYGZhy{}07, 2.00000000e+02, 4.09623144e\PYGZhy{}08,
       2.90191689e\PYGZhy{}08])
\end{sphinxVerbatim}







\renewcommand{\indexname}{Index}
\printindex
\end{document}